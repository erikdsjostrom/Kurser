\documentclass{article}

\usepackage[utf8]{inputenc}
\usepackage{amsthm}
\usepackage{amssymb}
\usepackage{mathtools}
\usepackage{graphicx}
\usepackage{mdframed}
\usepackage{float}
\usepackage[top=0.75in, bottom=0.75in, left=0.75in, right=0.75in]{geometry}
\usepackage{gauss}

\usepackage{array}
\allowdisplaybreaks

\makeatletter
\newcounter{elimination@steps}
\newcolumntype{R}[1]{>{\raggedleft\arraybackslash$}p{#1}<{$}}
\def\elimination@num@rights{}
\def\elimination@num@variables{}
\def\elimination@col@width{}
\newenvironment{elimination}[4][0]
{
    \setcounter{elimination@steps}{0}
    \def\elimination@num@rights{#1}
    \def\elimination@num@variables{#2}
    \def\elimination@col@width{#3}
    \renewcommand{\arraystretch}{#4}
    \start@align\@ne\st@rredtrue\m@ne
}
{
    \endalign
    \ignorespacesafterend
}
\newcommand{\step}[2]
{
    \ifnum\value{elimination@steps}>0\sim\quad\fi
    \left[
        \ifnum\elimination@num@rights>0
            \begin{array}
            {@{}*{\elimination@num@variables}{R{\elimination@col@width}}
            |@{}*{\elimination@num@rights}{R{\elimination@col@width}}}
        \else
            \begin{array}
            {@{}*{\elimination@num@variables}{R{\elimination@col@width}}}
        \fi
            #1
        \end{array}
    \right]
    & 
    \begin{array}{l}
        #2
    \end{array}
    \addtocounter{elimination@steps}{1}
}
\makeatother

\DeclarePairedDelimiter{\abs}{\lvert}{\rvert}
\DeclarePairedDelimiter{\norm}{\lvert \lvert}{\rvert \rvert}

\newtheoremstyle{break}% name
  {}%         Space above, empty = `usual value'
  {}%         Space below
  {\itshape}% Body font
  {}%         Indent amount (empty = no indent, \parindent = para indent)
  {\bfseries}% Thm head font
  {.}%        Punctuation after thm head
  {\newline}% Space after thm head: \newline = linebreak
  {}%         Thm head spec

\newtheorem{Def}{Definition}[section]

\theoremstyle{break}

\newtheorem{innerEx}{Exempel}[section]
\newtheorem{sats}{Sats}[section]
\newtheorem{Rem}{Anmärkning}[]

\newenvironment{Ex}
{\begin{mdframed} \begin{innerEx} \vspace{3pt}}
{\vspace{3pt} \end{innerEx} \end{mdframed}}  

\newenvironment{bevis}
{\begin{mdframed} \begin{proof} \vspace{3pt}}
{\vspace{3pt} \end{proof} \end{mdframed}}

\usepackage{color}
\title{
	 Finite Automata Theory and Formal Languages\\
	 Föreläsning 2 - Central concepts of automata theory
    \author{Erik Sjöström}
}
\begin{document}
\maketitle

\section{Alphabets} % (fold)
\label{sec:alphabets}
\begin{Def}
    An alphabet is a finite, non-empty set of symbols, usually denoted by $\Sigma$\\
    The number of symbols in $\Sigma$ is donoted as $\abs{\Sigma}$
\end{Def}

\begin{Not}
We will use $a,b,c,...$ to denote symbols.
\end{Not}

\begin{Rem}
Alphabets will represent the observable events of the automata.
\end{Rem}
\begin{Ex}
    Some alphabets:
    \begin{itemize}
        \item on/off-switch: $\Sigma$ = \{Push\}
        \item simple vending machine: $\Sigma$ = \{5 kr, choc\}
        \item complex vending machine $\Sigma$ = \{5 kr, 10 kr, choc, big choc\}
        \item parity counter $\Sigma$ = \{$p_0, p_1$\}
    \end{itemize}
\end{Ex}
% section alphabets (end)

\section{Strings or Words} % (fold)
\label{sec:strings_or_words}
\begin{Def}
    Strings/Words are finite sequences of symbols from some alphabet.
\end{Def}
\begin{Not}
    We will use $w,x,y,z,...$ to denote words.
\end{Not}
\begin{Rem}
    Words will represent the behaviour of an automaton.
\end{Rem}
\begin{Ex}
    Some behaviours:
    \begin{itemize}
        \item on/off-switch: Push, Push, Push, Push
        \item simple vending machine: 5 kr, choc, 5 kr, choc, 5 kr, choc
        \item parity counter: $p_0p_1$ or $p_0p_0p_0p_1p_1p_0$
    \end{itemize}
\end{Ex}
\begin{Rem}
    Some words do NOT represent \emph{behaviour}
\end{Rem}
\begin{Ex}
    simple vending machine: choc, choc, choc
\end{Ex}
% section strings_or_words (end)

\section{Inductive Definition of $\Sigma^*$} % (fold)
\label{sec:inductive_definition_of_}
\begin{Def}
    $\Sigma^*$ is the set of all words for a given alphabet $\Sigma$.
\end{Def}
This can be described inductively in at least 2 different ways:
\begin{itemize}
    \item Base case: $\epsilon \in \Sigma^*$
    \item Inductive step: if $a \in \Sigma$ and $x \in \Sigma^*$ then $ax \in \Sigma^*$
    \item We will usually work with this definition.
\end{itemize}
Or:
\begin{itemize}
    \item Base case: $\epsilon \in \Sigma^*$
    \item Inductive step: if $a \in \Sigma$ and $x \in \Sigma^*$ then $xa \in \Sigma^*$
\end{itemize}
We can (recursively) \emph{define} functions over $\Sigma^*$ and (inductively) \emph{prove} properties about those functions.
% section inductive_definition_of_ (end)s% section strings_or_words (end)

\section{Concatenation} % (fold)
\label{sec:concatenation}
\begin{Def}
    Given the string x and y, the concatenation xy is defined as:
    \begin{align*}
        &\epsilon y = y\\
        &(ax')y = a(x'y)
    \end{align*}
\end{Def}
\noindent
Observe that in general $xy \neq xy$
\begin{Ex}
    If x = 010, and y = 11, then xy = 01011, and yx = 11010
\end{Ex}
\begin{Lem}
    If $\Sigma$ has more than one symbol then concatenation is not commutative.
\end{Lem}
% section concatenation (end)

\section{Prefix and Suffix} % (fold)
\label{sec:prefix_and_suffix}
\begin{Def}
    Given x and y words over a certain alphabet $\Sigma$:
    \begin{itemize}
        \item x is a {\color{red} prefix} of y iff there exists z such that y = xz
        \item x is a {\color{red} suffix} of y iff there xists z such that y = zx
    \end{itemize}
\end{Def}
\begin{Rem}
    $\forall x$, $\epsilon$ is both a prefix and suffix of x.
\end{Rem}
\begin{Rem}
    $\forall x$, $x$ is both a prefix and suffix of x.
\end{Rem}
% section prefix_and_suffix (end)

\section{Length and Reverse} % (fold)
\label{sec:length_and_reverse}
\begin{Def}
    The {\color{red} length} function $\abs{\mbox{ }}:$ $\Sigma^* \to \mathbb{N}$ is defined as:
    \begin{align*}
        &\abs{\epsilon} = 0\\
        &\abs{ax} = 1 + \abs{x}
    \end{align*}
\end{Def}
\begin{Ex}
    $\abs{01010} = 5$
\end{Ex}
\begin{Def}
    The {\color{red} reverse} function $rev(\mbox{ }):$ $\Sigma^* \to \Sigma^*$ is defined as:
    \begin{align*}
        &rev(\epsilon) = \epsilon\\
        &rev(ax) = rev(x)a
    \end{align*}
\end{Def}
\begin{Ex}
    $rev(a_1...a_n)$ = $a_n...a_1$
\end{Ex}
% section length_and_reverse (end)

\section{Power} % (fold)
\label{sec:power}
\subsection{Of a string} % (fold)
\label{sub:of_a_string}
\begin{Def}
    We define $x^n$ as follows:
    \begin{align*}
        &x^0 = \epsilon\\
        &x^{n+1} = xx^n
    \end{align*}
\end{Def}
\begin{Ex}
    $(010)^3$ = $(010010010)$
\end{Ex}
% subsection of_a_string (end)
\subsection{Of an alphabet} % (fold)
\label{sub:of_an_alphabet}
\begin{Def}
    We define $\Sigma^n$, the set of words over $\Sigma$ with length n, as follows:
    \begin{align*}
        &\Sigma^0 = \{\epsilon\}\\
        &\Sigma^{n+1} = \{ax \mid a \in \Sigma, x \in \Sigma^n\}
    \end{align*}
\end{Def}
\begin{Ex}
    $\{0,1\}^3$ = $\{000,001,010,011,100,101,110,111\}$
\end{Ex}
\begin{Not}
    \begin{align*}
        &\Sigma^* = \Sigma^0 \cup \Sigma^1 \cup \Sigma^2 ... \\
        &\Sigma^+ = \Sigma^1 \cup \Sigma^2 \cup \Sigma^3 ...
    \end{align*}
\end{Not}
% subsection of_an_alphabet (end)
% section power (end)

\section{Some properties} % (fold)
\label{sec:some_properties}
The following properties can be proved by induction:
\begin{Lem}
    Concatenation is associative: $\forall x,y,z$: $x(yz) = (xy)z$
\end{Lem}
\begin{Lem}
    $\forall x$: $x\epsilon = \epsilon x = x$
\end{Lem}
\begin{Lem}
    $\forall x$: $\abs{x^n} = n * \abs{x}$
\end{Lem}
\begin{Lem}
    $\forall \Sigma$: $\abs{\Sigma} = \abs{\Sigma}^n$
\end{Lem}
\begin{Lem}
    $\forall x$: $rev(rev(x)) = x$
\end{Lem}
\begin{Lem}
    $\forall x,y$: $rev(xy) = rev(y)rev(x)$
\end{Lem}
% section some_properties (end)

\section{Languages} % (fold)
\label{sec:languages}
\begin{Def}
    Given an alphabet $\Sigma$, a {\color{red} language} $\mathcal{L}$ is a subset of $\Sigma^*$, that is, $\mathcal{L} \subseteq \Sigma^*$
\end{Def}
\begin{Rem}
    If $\mathcal{L} \subseteq \Sigma^*$ and $\Sigma \subseteq \Delta$ then $\mathcal{L} \subseteq \Delta^*$
\end{Rem}
\begin{Rem}
    A language can be either finite or infinite.
\end{Rem}
\begin{Ex}
    Some languages:
    \begin{itemize}
        \item Swedish, English, French, Spanish...
        \item Any programming language
        \item $\emptyset$, $\{\epsilon\}$, and $\Sigma^*$ are languages over any $\Sigma$
        \item The set of prime Natural numbers: $\{1,3,5,7,11,...\}$
    \end{itemize}
\end{Ex}
% section languages (end)

\section{Some Operations on Languages} % (fold)
\label{sec:some_operations_on_languages}
\begin{Def}
    Given $\mathcal{L}, \mathcal{L}_1, \mathcal{L}_2$ languages, we define the following languages:
\end{Def}
\begin{itemize}
\item \textbf{Union, intersection:} The same as for any set.\\
\item \textbf{Concatenation:} $\mathcal{L}_1 \mathcal{L}_1 = \{x_1x_2 \mid x_1 \in \mathcal{L}_1, x_2 \in \mathcal{L}_2\}$\\
\item \textbf{Closure:} $\mathcal{L}^* = \bigcup_{n \in \mathbb{N}} \mathcal{L}^n$ where $\mathcal{L}^0 = \{\epsilon\}$, $\mathcal{L}^{n+1} = \mathcal{L}^n \mathcal{L}$
\end{itemize}
\begin{Rem}
        $\emptyset^* = \{\epsilon\}$\\
        $\mathcal{L}^* = \mathcal{L}^0 \cup \mathcal{L}^1 \cup \mathcal{L}^2 \cup ... = \left\{ \epsilon \right\} \cup \left\{ x_1 ... x_n \mid n > 0, x_i \in \mathcal{L} \right\}$
\end{Rem}
\begin{Not}
    $\mathcal{L}^* = \mathcal{L}^1 \cup \mathcal{L}^2 \cup \mathcal{L}^3 \cup ...$
\end{Not}
\begin{Ex}
    Let $\mathcal{L} = \left\{ aa, b \right\}$, then
    \begin{align*}
        &\mathcal{L}^0 = \left\{ \epsilon \right\}\\
        &\mathcal{L}^1 = \mathcal{L}\\
        &\mathcal{L}^2 = \mathcal{L}\mathcal{L} = \left\{ aaaa, aab, baa, bb \right\}\\
        &\mathcal{L}^3 = \mathcal{L}^2 \mathcal{L}\\
        &\vdots\\
        &\mathcal{L}^* = \left\{ \epsilon, aa, b, aaaa, aab, baa, bb, ... \right\}
    \end{align*}
\end{Ex}
% section some_operations_on_languages (end)

\section{How to Prove the Equality of Languages?} % (fold)
\label{sec:how_to_prove_the_equality_of_languages_}
Given the languages $\mathcal{L}$ and $\mathcal{M}$, how can we prove that $\mathcal{L} = \mathcal{M}$\\
\bigskip

\noindent
A few possibilities:
\begin{itemize}
    \item Languages are sets so we prive that $\mathcal{L} \subseteq \mathcal{M}$ and $\mathcal{M} \subseteq \mathcal{L}$
    \item Transitivity of equality: $\mathcal{L} = \mathcal{L}_1 = ... = \mathcal{L}_m = \mathcal{M}$
    \item We can reason about the elements in the language:
    \begin{Ex}
        $\left\{ a(ba)^n \mid n \ge 0 \right\}$ = $\left\{ (ab)^na \mid n \ge 0 \right\}$ can be proved by induction on $n.$
    \end{Ex}
\end{itemize}
% section how_to_prove_the_equality_of_languages_ (end)

\section{Algebraic Laws for Languages} % (fold)
\label{sec:algebraic_laws_for_languages}
Laws of concatenation:
\begin{itemize}
    \item Associativity: $\mathcal{L}(\mathcal{M}\mathcal{N}) = (\mathcal{L}\mathcal{M})\mathcal{N}$
    \item Not commutative: $\mathcal{L}\mathcal{M} \neq \mathcal{M}\mathcal{L}$
    \item Distributivity: $\mathcal{L}(\mathcal{M} \cup \mathcal{N}) = \mathcal{L}\mathcal{M} \cup \mathcal{L}\mathcal{N}$
    \item Distributivity: $(\mathcal{M} \cup \mathcal{N})\mathcal{L} = \mathcal{M}\mathcal{L} \cup \mathcal{N}\mathcal{L}$
    \item Identity: $\mathcal{L}\left\{ \epsilon \right\} = \left\{ \epsilon \right\}\mathcal{L} = \mathcal{L}$
    \item Annihilator: $\mathcal{L}\emptyset = \emptyset \mathcal{L} = \emptyset$
    \item Other Rules:
    \begin{itemize}
        \item $\emptyset^* = \left\{ \epsilon \right\}^* = \left\{ \epsilon \right\}$
        \item $\mathcal{L}^+ = \mathcal{L}\mathcal{L}* = \mathcal{L}^*\mathcal{L}$
        \item $(\mathcal{L}^*)^* = \mathcal{L}^*$
    \end{itemize}
\end{itemize}
\begin{Rem}
    While:
    \begin{align*}
        &\mathcal{L}(\mathcal{M} \cap \mathcal{N}) \subseteq \mathcal{L}\mathcal{M} \cap \mathcal{L}\mathcal{N}
        &&\mbox{and}
        &&(\mathcal{M} \cap \mathcal{N})\mathcal{L} \subseteq \mathcal{M}\mathcal{L} \cap \mathcal{N}\mathcal{L}
    \end{align*}
    both hold, in general
    \begin{align*}
        &\mathcal{L}\mathcal{M} \cap \mathcal{L}\mathcal{N} \subseteq \mathcal{L}(\mathcal{M} \cap \mathcal{N})
        &&\mbox{and}
        &&\mathcal{M}\mathcal{L} \cap \mathcal{N}\mathcal{L} \subseteq (\mathcal{M} \cap \mathcal{N})\mathcal{L}
    \end{align*}
    don't.
\end{Rem}
\begin{Ex}
    Consider the case where:
    \[
    \mathcal{L} = \left\{ \epsilon, a \right\},
    \mathcal{M} = \left\{ a \right\},
    \mathcal{N} = \left\{ aa \right\}
    \]
    Then
    \[
     \mathcal{L}\mathcal{M} \cap \mathcal{L}\mathcal{N} = \left\{ aa \right\}
    \] but
    \[
    \mathcal{L}(\mathcal{M} \cap \mathcal{N}) = \mathcal{L}\emptyset = \emptyset
    \]
\end{Ex}
% section algebraic_laws_for_languages (end)

\section{Functions between Languages} % (fold)
\label{sec:functions_between_languages}
\begin{Def}
    A function $f: \Sigma^* \to \Delta^*$ between 2 languages should satisfy:
    \begin{align*}
        &f(\epsilon) = \epsilon\\
        &f(xy) = f(x)f(y)
    \end{align*}
\end{Def}
Intuitively, $f(a_1 ... a_n) = f(a_1)...f(a_n)$
\begin{Rem}
    $f(a) \in \Delta^*$ if $a \in \Sigma$
\end{Rem}
% section functions_between_languages (end)


























\end{document}
