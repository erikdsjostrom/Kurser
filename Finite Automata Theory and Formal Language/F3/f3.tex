\documentclass{article}

\usepackage[utf8]{inputenc}
\usepackage{amsthm}
\usepackage{amssymb}
\usepackage{mathtools}
\usepackage{graphicx}
\usepackage{mdframed}
\usepackage{float}
\usepackage[top=1in, bottom=1.25in, left=1.25in, right=1.25in]{geometry}

\DeclarePairedDelimiter{\abs}{\lvert}{\rvert}
\DeclarePairedDelimiter{\norm}{\lvert \lvert}{\rvert \rvert}

\newtheoremstyle{break}% name
  {}%         Space above, empty = `usual value'
  {}%         Space below
  {\itshape}% Body font
  {}%         Indent amount (empty = no indent, \parindent = para indent)
  {\bfseries}% Thm head font
  {.}%        Punctuation after thm head
  {\newline}% Space after thm head: \newline = linebreak
  {}%         Thm head spec

\newtheorem{Def}{Definition}[section]

\theoremstyle{break}

\newtheorem{innerEx}{Exempel}[section]
\newtheorem{sats}{Sats}[section]
\newtheorem{Rem}{Anmärkning}[section]

\newenvironment{Ex}
{\begin{mdframed} \begin{innerEx} \vspace{3pt}}
{\vspace{3pt} \end{innerEx} \end{mdframed}}  

\newenvironment{bevis}
{\begin{mdframed} \begin{proof} \vspace{3pt}}
{\vspace{3pt} \end{proof} \end{mdframed}}
\usepackage{color}
\usepackage{bussproofs}
\usepackage{listings}
\lstset{language=Haskell,
        mathescape}
\title{
	 Finite Automata Theory and Formal Languages\\
	 Föreläsning 3 - Proofs
    \author{Erik Sjöström}
}
\begin{document}
\maketitle

\section{How Formal Should a Proof Be?} % (fold)
\label{sec:how_formal_should_a_proof_be_}
Depends on the purpose but:
\begin{itemize}
    \item Should be convincing
    \item Should not leave too much out
    \item The validity of each step should be easily understood
\end{itemize}
Valid steps are for example:
\begin{itemize}
    \item Reduction to definition:
    \begin{itemize}
        \item "x is a positive integer" is equivalent to "x > 0"
    \end{itemize}
    \item Use of hypotheses
    \item Combining previous facts in a valid way:
    \begin{itemize}
        \item "Given $A \Rightarrow B$ and $A$ we can conclude $B$ by \textit{modus ponens}"
    \end{itemize}
\end{itemize}
% section how_formal_should_a_proof_be_ (end)

\section{Form of Statements} % (fold)
\label{sec:form_of_statements}
Statements we want to prove are usually of the form:
\begin{center}
    If $H_1$ and $H_2$ ... and $H_n$ then $C_1$ and ... and $C_m$
\end{center}
or:
\begin{center}
    $P_1$ and ... and $P_k$ iff $Q_1$ and ... and $Q_m$
\end{center}
for $n \ge 0$; $m,k \ge 1$
\begin{Rem}
    Observe that one proves the \emph{conclusion} assuming the validity of the \emph{hypotheses}!
\end{Rem}
\begin{Ex}
    We can easily prove "if 0 = 1 then 4 = 2.000"
\end{Ex}
% section form_of_statements (end)

\section{Different Kinds of Proofs} % (fold)
\label{sec:different_kinds_of_proofs}
\textbf{Proofs by contradiction:}
\begin{center}
    \textit{If H then C}
\end{center}
is logically equivalent to
\begin{center}
    \textit{H and not C implies "something known to be false"}
\end{center}
\begin{Ex}
    If $x \neq 0$ then $x^2 \neq 0$,   vs   ,$x \neq 0$ and $x^2 = 0$ is impossible!
\end{Ex}
\textbf{Proofs by Contrapositive:}\\
\emph{"If H then C"} is logically equivalent to \emph{"If not C then not H"}\\
\textbf{Proofs by Counterexample}\\
We find an example that \emph{"breaks"} what we want to prove.
\begin{Ex}
    All Natural numbers are odd.
\end{Ex}
% section different_kinds_of_proofs (end)

\section{Proving a Property over the Natural Numbers} % (fold)
\label{sec:proving_a_property_over_the_natural_numbers}
How to prove a statement over \emph{all} the Natural numbers?
\begin{Ex}
    $\forall n \in \mathbb{N}$, if $n \mid 4$ then $n \mid 2$.
\end{Ex}
First we need to look at what the Natural numbers are:
\begin{Def}
    They are an \emph{inductively defined set} and can be defined by the following \emph{two} rules:
    \begin{center}
        \AxiomC{}
        \UnaryInfC{$0 \in \mathbb{N}$}
        \DisplayProof
        \hspace{3cm}
        \AxiomC{$n \in \mathbb{N}$}
        \UnaryInfC{$n + 1 \in \mathbb{N}$}
        \DisplayProof
    \end{center}
    \end{Def}
% section proving_a_property_over_the_natural_numbers (end)

\section{Mathematical/Simple Induction} % (fold)
\label{sec:mathematical_simple_induction}
\begin{prooftree}
    \AxiomC{$\overbrace{P(0)}^\text{base case}$}
    \AxiomC{$\overbrace{\forall n \in \mathbb{N}\mbox{, } {\color{red} P(n)} \Rightarrow P(n + 1)}^\text{inductive step}$}
    \BinaryInfC{$\underbrace{\forall n \in \mathbb{N}\mbox{, }P(n)}_\text{statement to prove}$}
\end{prooftree}
More generally:
\begin{prooftree}
    \AxiomC{$P(i), P(i+1),...,P(j)$}
    \AxiomC{$\forall i \le j \le n \mbox{, } {\color{red} P(n)} \Rightarrow P(n + 1)$}
    \BinaryInfC{$\forall i \le n \mbox{, } P(n)$}
\end{prooftree}
Hypotheses in red is called {\color{red} \emph{inductive hypotheses.}} (IH).
% section mathematical_simple_induction (end)

\section{Course-of-Values/Strong Induction} % (fold)
\label{sec:course_of_values_strong_induction}
Variant of mathematical induction.
\begin{prooftree}
    \AxiomC{$\overbrace{P(0)}^\text{base case}$}
    \AxiomC{$\overbrace{\forall n \in \mathbb{N}\mbox{, } (\forall m \in \mathbb{N} \mbox{, } 0 \le m \le n \Rightarrow {\color{red} P(m)}) \Rightarrow P(n + 1)}^\text{inductive step}$}
    \BinaryInfC{$\underbrace{\forall n \in \mathbb{N}\mbox{, }P(n)}_\text{statement to prove}$}
\end{prooftree}
Or more generally:
\begin{prooftree}
    \AxiomC{$P(i), P(i + 1), ..., P(j)$}
    \AxiomC{$\forall j < n \mbox{, } (\forall m  \mbox{, } i \le m \le n \Rightarrow {\color{red} P(m)}) \Rightarrow P(n)$}
    \BinaryInfC{$\forall i \le n \mbox{, }P(n)$}
\end{prooftree}
Here we might have {\color{red} several inductive hypotheses} P(m)!
% section course_of_values_strong_induction (end)

\section{Example: Proof by Induction} % (fold)
\label{sec:example_proof_by_induction}
\textbf{Proposition:} Let $f(0) = 0$ and $f(n+1) = f(n) + n + 1$\\
Then $\forall n \in \mathbb{N}$, $f(n) = n(n+1)/2$
\begin{proof}
    By mathematical induction on n\\
    Let $P(n)$ b $f(n) = n(n + 1)/2$
    \begin{itemize}
        \item \textbf{Base case:} We prove that $P(0)$ holds
        \item \textbf{Inductive step:} We prove that if for a given $n \ge 0$ $P(n)$ holds (our IH) then $P(n + 1)$ also holds
        \item \textbf{Closure:} Now we have established that for all $n$, $P(n)$ is true!
    \end{itemize}
\end{proof}
% section example_proof_by_induction (end)

\section{Example: Proof by Induction} % (fold)
\label{sec:example_proof_by_induction2}
\textbf{Proposition:} If $n \ge 8$ then $n$ can be written as a sum of 3's and 5's.
\begin{proof}
    By course-of-values induction on n\\
    Let $P(n)$ be "n can be written as a sum of 3's and 5's"
    \begin{itemize}
        \item \textbf{Base case:} $P(8)$, $P(9)$ and $P(10)$ hold
        \item \textbf{Inductive step:}
        \begin{itemize}
            \item We shall prove that if $P(8)$, $P(9)$, $P(10)$,..., $P(n)$ hold for $n \ge 10$ (our IH) then $P(n+1)$ holds
            \item Observe that if $n \ge 10$ then $n \ge n + 1 - 3 \ge 8$
            \item Hence by inductive hypotheses $P(n + 1 - 3)$ holds
            \item By adding an extra 3 then $P(n + 1)$ holds as well
        \end{itemize}
        \item \textbf{Closure:} $\forall n \ge 8$, $n$ can be written as a sum of 3's and 5's.
    \end{itemize}
\end{proof}
% section example_proof_by_induction2 (end)

\section{Example: All Horses have the Same Colour} % (fold)
\label{sec:example_all_horses_have_the_same_colour}
\textbf{Proposition:} All horses have the same colour.
\begin{proof}
    By mathemathical induction on $n$.\\
    Let $P(n)$ be \emph{"in any set of n horses they all have the same colour"}
    \begin{itemize}
        \item \textbf{Base case:}
        \begin{itemize}
            \item $P(0)$ is not interesting in this example
            \item $P(1)$ is clearly true
        \end{itemize}
        \item \textbf{Inductive step:}
        \begin{itemize}
            \item Let us show that $P(n)$ (our IH) implies $P(n + 1)$
            \item Let $h_1, h_2,..., h_n, h_{n+1}$ be a set of $n + 1$ horses
            \item Take $h_1, h_2,..., h_n$. By IH they all have the same colour.
            \item Now take $h_2, h_3,..., h_n,h_{n+1}$. Again by IH they all have the same colour.
            \item Hence, by transitivity all horses $h_1, h_2,..., h_n, h_{n+1}$, must have the same colour
        \end{itemize}
        \item \textbf{Closure:} $\forall n$, all $n$ horses in the set have the same colour.
    \end{itemize}
\end{proof}
% section example_all_horses_have_the_same_colour (end)

\section{Mutual induction} % (fold)
\label{sec:mutual_induction}
Sometimes we cannot prove a single statement $P(n)$ but rather a group of statements $P_1(n), P_2(n),..., P_k(n)$ simultaneously by induction on $n$.\\
\smallskip

This is very common in automata theory where we need a statement for each of states of the automata.
% section mutual_induction (end)

\section{Example: Proof by Mutual Induction} % (fold)
\label{sec:example_proof_by_mutual_induction}
Let $f,g,h: \mathbb{N} \to \left\{ 0,1 \right\}$ be as follows:
\begin{align*}
    &f(0) = 0 &&g(0) = 1 &&h(0)=0\\
    &f(n+1) = g(n) &&g(n+1) = f(n) && h(n+1) = 1 - h(n)
\end{align*}
\textbf{Proposition:} $\forall n$, $h(n) = f(n)$
\begin{proof}
    If $P(n)$ is \emph{"h(n) = f(n)"} it does no seem possible to prove $P(n) \Rightarrow P(n+1)$ directly.
    \begin{itemize}
        \item We strengthen \emph{P(n)} to \emph{P'(n)}: Let $P'(n)$ be $h(n) = f(n) \land h(n) = 1 - g(n)$
        \item By mathematical induction we prove $P'(0)$: $h(0) = f(0) \land h(0) = 1 - g(0)$
        \item Then we prove that $P'(n) \Rightarrow P'(n+1)$
        \item Since $\forall n$, $P'(n)$ is true then $\forall n$. $P(n)$ is true.
    \end{itemize}
\end{proof}
% section example_proof_by_mutual_induction (end)

\section{Recursive Data Types} % (fold)
\label{sec:recursive_data_types}
What are (the data types of) Natural numbers, lists, trees, ...?\\
\smallskip

This is how you would defined them in Haskell:
\begin{lstlisting}
    data Nat = Zero | Succ Nat
    data List a = Nil | Cons a (List a)
    data BTree a = Leaf a | Node a (BTree a) (BTree a)
\end{lstlisting}
% section recursive_data_types (end)

\section{Inductively Defined Sets} % (fold)
\label{sec:inductively_defined_sets}
\textbf{Natural Numbers:}
\begin{itemize}
    \item \textbf{Base case:} 0 is a Natural number
    \item \textbf{Inductive step:} If \emph{n} is a Natural number then \emph{n + 1} is a Natural number
    \item \textbf{Closure:} There is no other way to construct a Natural number
\end{itemize}
\textbf{Finite Lists:}
\begin{itemize}
    \item \textbf{Base case:} [] is the empty list over any set \emph{A}
    \item \textbf{Inductive step:} If $a \in A$ and \emph{xs} is a list over \emph{A} then \emph{a:xs} is a list over \emph{A}
    \item \textbf{Closure:} There is no other way to construct lists
\end{itemize}
\textbf{Finitely Branching Trees:}
\begin{itemize}
    \item \textbf{Base case:} If $a \in A$ then \emph{(a)} is a tree over any set \emph{A}
    \item \textbf{Inductive step:} If $t_1,..., t_k$ are trees over ther set \emph{A} and $a \in A$ then $(a, t_1,..., t_k)$ is a tree over \emph{A}
    \item \textbf{Closure:} There is no other way to construct trees.
\end{itemize}
To define a set \emph{S} by induction we need to specify:
\begin{itemize}
    \item \textbf{Base case:} $e_1,..., e_m \in S$
    \item \textbf{Inductive steps:} Five $s_1,..., s_n \in S$ then $c_1[s_1,..., s_{n_1}],..., c_k[s_1,...,s_{n_k}]$
    \item \textbf{Closure:} There is no other way to construct elements in \emph{S}. (We will usually omit this part.)
\end{itemize}
\begin{Ex}
    The set of simple Boolean expressions if defined as:
    \begin{itemize}
        \item \textbf{Base cases:} \emph{true} and \emph{false} are Boolean expressions
        \item if \emph{a} and \emph{b} are Boolean expressions then:
            \begin{align*}
                &(a)
                &&\mbox{not }a
                &&a \mbox{ and }b
                &&a \mbox{ or }b
            \end{align*}
            are also Boolean expressions
    \end{itemize}
\end{Ex}
% section inductively_defined_sets (end)

\section{Proofs by Structural Induction} % (fold)
\label{sec:proofs_by_structural_induction}
Generalisation of mathematical induction to other inductively defined sets such as lists, trees, ...\\
{\color{red} Very} useful in computer science: it allows to prove properties over the (finite) elments in a data type!\\
\smallskip

Given an inductively defined set \emph{S}, to prove $\forall s in S$, $P(s)$ then:
\begin{itemize}
    \item \textbf{Base cases:} We prove that $P(e_1),..., P(e_m)$
    \item \textbf{Inductive steps:} Assuming $P(s_1),..., P(s_m)$ (out {\color{red} inductive hypotheses} IH), we prove $P(c_1[s_1,..., s_{n_1}], ..., P(c_k[s_1,..., s_{n_k}])$
    \item \textbf{Closure:} $\forall s \in S$, $P(s)$ (We will usually omit this part)
\end{itemize}
% section proofs_by_structural_induction (end)

\section{Inductive Sets and Structural Induction} % (fold)
\label{sec:inductive_sets_and_structural_induction}
Inductive definition of S:
\begin{center}
    \AxiomC{\mbox{}}
    \UnaryInfC{$e_1 \in S$}
    \DisplayProof
    $\dots$
    \AxiomC{\mbox{}}
    \UnaryInfC{$e_m \in S$}
    \DisplayProof
    \hspace{1cm}
    \AxiomC{$s_1,...,s_{n_1} \in S$}
    \UnaryInfC{$c_1[s_1,...,s_{n_1}] \in S$}
    \DisplayProof
    \dots
    \AxiomC{$s_1,..., s_{n_k} \in S$}
    \UnaryInfC{$c_k[s_1,..., s_{n_k}] \in S$}
    \DisplayProof
\end{center}
Inductive principle associated to S:
\begin{center}
    \[
    \mbox{base cases: }
    \begin{cases}
        P(e_1)\\
        \vdots\\
        P(e_m)
    \end{cases}
    \]
    \[
    \mbox{inductive steps: }
    \begin{cases}
        \forall s_1,..., s_{n_1} \in S, \mbox{ } {\color{red} P(s_1)},..., {\color{red} P(s_{n_1})} \Rightarrow P(c_1[s_1,..., s_{n_1}])\\
        \vdots\\
        \forall s_1,..., s_{n_k} \in S, \mbox{ } {\color{red} P(s_1)},..., {\color{red} P(s_{n_k})} \Rightarrow P(c_k[s_{k_1},..., s_{n_k}])
    \end{cases}
    \hline
    \forall s \in S, \mbox{ } P(s)
    \]
\end{center}
% section inductive_sets_and_structural_induction (end)

\section{Example: Structural Induction over Lists} % (fold)
\label{sec:example_structural_induction_over_lists}
We can now use recursion to define functions over an inductively defined set and the prove properties of these functions by structural induction.\\
\smallskip

Let us (recursively) define the append and length function over lists:
\begin{align*}
    &[] ++ ys = ys &&len [] = 0\\
    &(a:xs) ++ ys = a : (xs ++ ys) && len(a:xs) = 1 + len xs
\end{align*}
\begin{lstlisting}
    [] ++ ys = ys                    len [] = 0
    (a:xs) ++ ys = a:(xs ++ ys)    len(a:xs) = 1 + len xs
\end{lstlisting}\\
\smallskip

\noindent
\textbf{Proposition:} $\forall xs, ys \in \mbox{ List }A$, $\mbox{len }(xs ++ ys) =$ $\mbox{len } xs + \mbox{ len } ys$
\begin{proof}
    By structural induction on $xs in \mbox{ List }A$\\
    $P(xs)$ is $\forall ys \in \mbox{ List }A$, $\mbox{len }(xs ++ ys) =$ $\mbox{len } xs + \mbox{ len } ys$
    \begin{itemize}
        \item \textbf{Base case:} We prove $P[]$
        \item \textbf{Inductive step:} We show $\forall xs \in \mbox{ A }, a \in A$, $P(xs) \Rightarrow P(a:xs)$
        \item \textbf{Closure:} $\forall xs \in \mbox{ List A}$, $P(xs)$
    \end{itemize}
\end{proof}
% section example_structural_induction_over_lists (end)

\section{Example: Structural Induction over Lists} % (fold)
\label{sec:example_structural_induction_over_lists2}
Let us (recursively) define append and reverse function over lists:
\begin{lstlisting}
    [] ++ ys = ys                rev [] = []
    (a:xs) ++ ys = a:(xs ++ ys)  rev(a:xs) = rev xs ++ [a]
\end{lstlisting}
Assume append is associative and that \lstinline|ys ++ [] = ys|\\
\textbf{Proposition:} $\forall xs, ys \in \mbox{ List }A$, $\mbox{rev } (xs ++ ys) = \mbox{ rev } ys ++ xs$
\begin{proof}
    By structural induction on $xs \in \mbox{ List }A$\
    $P(xs)$ is $\forall ys \in \mbox{ List }A$, $\mbox{rev }(xs ++ ys) = \mbox{ rev }ys ++ \mbox{ rev }xs$
    \begin{itemize}
        \item \textbf{Base case:} We prive $P[]$
        \item \textbf{Inductive step:} We show $\forall xs \in \mbox{ List }A$, $a \in A$, $P(xs) \Rightarrow P(a:xs)$
        \item \textbf{Closure:} $\forall xs \in \mbox{ List }A$, $P(xs)$
    \end{itemize}
\end{proof}
% section example_structural_induction_over_lists2 (end)

\section{Example: Structural Induction over Trees} % (fold)
\label{sec:example_structural_induction_over_trees}
Let us (recursively) define functions counting the number of edges and of nodes of a tree:
\begin{lstlisting}
    ne(a) = 0 nn(a) = 1             nn(a) = 1
    ne(a, $t_1$, ..., $t_k$) = k +            nn(a, $t_1$, ..., $t_k$) = 1 +
             ne($t_1$) + ... + ne($t_k$)                nn($t_1$) + ... + nn($t_k$)
\end{lstlisting}
\textbf{Proposition:}
$\forall t \in \mbox{ Tree }A$, \emph{nn(t) = 1 + ne(t)}
\begin{proof}
    By structural induction on $t \in \mbox{ Tree }A$\\
    $P(t)$ is \emph{nn(t) = 1 + ne(t)}
    \begin{itemize}
        \item \textbf{Base case:} We prove \emph{P(a)}
        \item \textbf{Inductive step:} We show $\forall t_1,..., t_k \in \mbox{ Tree }A$, $a \in A$, $P(t_1),..., p(t_k) \Rightarrow P(a,t_1,..., t_k)$
        \item \textbf{Closure:} $\forall t \in \mbox{ Tree }A$, \emph{P(t)}
    \end{itemize}
\end{proof}
% section example_structural_induction_over_trees (end)

\section{Proofs by Induction: Overview of the Steps to Follow} % (fold)
\label{sec:proofs_by_induction_overview_of_the_steps_to_follow}
\begin{itemize}
    \item State property \emph{P} prove by induction. (Might be more general than the actual statement we need to prove)
    \item {\color{red} Determine and state the method to use in the proof!}\\
    \textbf{Example:} Mathematical induction on the length of the list, course-of-values induction on the height of a tree, structural induction on a certain data type
    \item Identify and state base case(s). (Could be more than one! Not always trivial to determine)
    \item Prove base case(s)
    \item Identify and IH! (Will depend on the method to be used)
    \item Prove inductive step(s)
    \item (State closure)
    \item Deduce your statement from \emph{P}, (if not the same)
\end{itemize}
% section proofs_by_induction_overview_of_the_steps_to_follow (end)

\end{document}
