\documentclass{article}

\usepackage[utf8]{inputenc}
\usepackage{amsthm}
\usepackage{amssymb}
\usepackage{mathtools}
\usepackage{graphicx}
\usepackage{mdframed}
\usepackage{float}
\usepackage[top=0.75in, bottom=0.75in, left=0.75in, right=0.75in]{geometry}
\usepackage{gauss}

\usepackage{array}
\allowdisplaybreaks

\makeatletter
\newcounter{elimination@steps}
\newcolumntype{R}[1]{>{\raggedleft\arraybackslash$}p{#1}<{$}}
\def\elimination@num@rights{}
\def\elimination@num@variables{}
\def\elimination@col@width{}
\newenvironment{elimination}[4][0]
{
    \setcounter{elimination@steps}{0}
    \def\elimination@num@rights{#1}
    \def\elimination@num@variables{#2}
    \def\elimination@col@width{#3}
    \renewcommand{\arraystretch}{#4}
    \start@align\@ne\st@rredtrue\m@ne
}
{
    \endalign
    \ignorespacesafterend
}
\newcommand{\step}[2]
{
    \ifnum\value{elimination@steps}>0\sim\quad\fi
    \left[
        \ifnum\elimination@num@rights>0
            \begin{array}
            {@{}*{\elimination@num@variables}{R{\elimination@col@width}}
            |@{}*{\elimination@num@rights}{R{\elimination@col@width}}}
        \else
            \begin{array}
            {@{}*{\elimination@num@variables}{R{\elimination@col@width}}}
        \fi
            #1
        \end{array}
    \right]
    & 
    \begin{array}{l}
        #2
    \end{array}
    \addtocounter{elimination@steps}{1}
}
\makeatother

\DeclarePairedDelimiter{\abs}{\lvert}{\rvert}
\DeclarePairedDelimiter{\norm}{\lvert \lvert}{\rvert \rvert}

\newtheoremstyle{break}% name
  {}%         Space above, empty = `usual value'
  {}%         Space below
  {\itshape}% Body font
  {}%         Indent amount (empty = no indent, \parindent = para indent)
  {\bfseries}% Thm head font
  {.}%        Punctuation after thm head
  {\newline}% Space after thm head: \newline = linebreak
  {}%         Thm head spec

\newtheorem{Def}{Definition}[section]

\theoremstyle{break}

\newtheorem{innerEx}{Exempel}[section]
\newtheorem{sats}{Sats}[section]
\newtheorem{Rem}{Anmärkning}[]

\newenvironment{Ex}
{\begin{mdframed} \begin{innerEx} \vspace{3pt}}
{\vspace{3pt} \end{innerEx} \end{mdframed}}  

\newenvironment{bevis}
{\begin{mdframed} \begin{proof} \vspace{3pt}}
{\vspace{3pt} \end{proof} \end{mdframed}}


\title{
     Analys\\
     Föreläsning 8
    \author{Erik Sjöström}
}
\begin{document}
\maketitle

\section{Repetition} % (fold)
\label{sec:repetition}
\begin{sats}
    $f$ deriverbar i $x_0$ och $x_0$ är en lokal extrempunkt $\Rightarrow$ $f^\prime(x_0) = 0$
\end{sats}
\begin{Rem}
    $f^\prime(x_0) = 0 \nRightarrow x_0$ lokal extrempunkt
\end{Rem}
\begin{sats}
    \textbf{Medelvärdessatsen:} \\
    $f$ kontinuerlig på $[a,b]$\\
    $f$ deriverbar i $(a,b)$\\
    $\Rightarrow$ Finns $\xi \in (a,b)$ så att $f(b)-f(a)$ = $f^\prime(\xi)(b-a)$
\end{sats}
% section repetition (end)
\section{Dagens föreläsning} % (fold)
\label{sec:dagens_f_rel_sning}
\begin{sats}
    \textbf{Cauchys medelvärdessats:}\\
    $f, g$ kontinuerliga på $[a,b]$\\
    $f,g$ deriverbara  i $(a,b)$\\
    $g ^\prime \neq 0$ i $(a,b)$\\
    $\Rightarrow$ Finns $\xi \in (a,b)$ så att $\frac{f(b) - f(a)}{g(b)-g(a)} = \frac{f^\prime(\xi)}{g^\prime(\xi)}$
\end{sats}
\begin{Rem}
    $g(x) \Rightarrow g^\prime(x) = 1 \neq 0$
\end{Rem}
\begin{Rem}
    \begin{align*}
        g(b) - g(a) \neq 0 \mbox{ eftersom } g(b) - g(a) = \overbrace{g(\xi)}^{\neq 0}\overbrace{(b-a)}^{\neq 0}
    \end{align*}
\end{Rem}
\begin{bevis}
    Sätt:
    \[
    h(x) = (f(b) - f(a)) \cdot g(x) - (g(b) - g(a))\cdot f(x)
    \]
    Här gäller:
    \begin{itemize}
        \item $h$ kontinuerlig på $[a,b]$
        \item $h$ deriverbar i $(a,b)$
        \item $h(a) = (f(b) - f(a)) \cdot g(a) - (g(b) - g(a)) \cdot f(a) = f(b)g(a) - g(b)f(a)$
        \item $h(b) = (f(b) - f(a)) \cdot g(b) - (g(b) - g(a)) \cdot f(b) = f(b)g(a) - g(b)f(a)$
    \end{itemize}
    \textbf{Medelvärdessatsen} ger att det finns $\xi \in (a,b)$ så att:
    \[
    h^\prime(\xi) = \frac{h(b)- h(a)}{b-a} = 0
    \]
    Vi får:
    \[
    0 = h^\prime(\xi) = (f(b) - f(a))\cdot g^\prime(\xi) - (g(b)- g(a))f^\prime(\xi)
    \]
    dvs:
    \[
    \frac{f^\prime(\xi)}{g^\prime(\xi)} = \frac{f(b)- f(a)}{g(b)- g(a)}
    \]
    eftersom $g^\prime \neq 0$ och $g(b) - g(a) \neq 0$
\end{bevis}
\textbf{\underline{Tillämpning av medelvärdessatsen:}}\\
Antag $f$ kontinuerlig på $[a,b]$\\
Om:
\begin{enumerate}
    \item $f^\prime(x) \ge 0$ för $x \in (a,b)$ $\Rightarrow$ $f(x)$ är växande på $[a,b]$
    \item $f^\prime(x) > 0$ för $x \in (a,b)$ $\Rightarrow$ $f(x)$ är strängt växande på $[a,b]$
    \item $f^\prime(x) \le 0$ för $x \in (a,b)$ $\Rightarrow$ $f(x)$ är avtagande på $[a,b]$
    \item $f^\prime(x) < 0$ för $x \in (a,b)$ $\Rightarrow$ $f(x)$ är strängt avtagande på $[a,b]$
\end{enumerate}
eftersom:
\begin{enumerate}
    \item Fixera godtyckligt $x, \tilde{x}$ $a \le x < \tilde{x} \le b$
    \begin{align*}
        f(\tilde{x}) - f(x) = \underbrace{f^\prime(\xi_{x, \tilde{x}})}_{\ge 0} \underbrace{(\tilde{x} - x)}_{\ge 0}
    \end{align*}
    vilket följer av medelvärdessatsen tillämpat på $f$ på intervallet $[x, \tilde{x}]$\\
    Vi har $f(x) \le f(\tilde{x})$
    \begin{Rem}
        $f$ kontinuerlig på $[a,b]$\\
        $f^\prime(x) > 0$ på (a,b) utom i ändligt många punkter i (a,b)\\
        $\Rightarrow$ $f$ strängt växande på $[a,b]$
    \end{Rem}
\end{enumerate}
% section dagens_f_rel_sning (end)
\section{Primitv funktion $=$ "antiderivata"} % (fold)
\label{sec:primitv_funktion}
Givet funktion $g(x)$\\
Vi säger att $G(x)$ är en primitiv funktion till $g(x)$ om:
\[
G^\prime(x) = g(x)
\]
Fråga:
\begin{enumerate}
    \item Krav på $g(x)$ för att en $G(x)$ ska existerar?
    \item $G(x)$ entydigt bestämt av $g(x)$?
\end{enumerate}
\textbf{Fråga 2:} Antag att $G(x)$ och $\tilde{G}(x)$ är primitiva funktioner till $g(x)$
\begin{align*}
    &G^\prime(x) = g(x) &&\tilde{G}(x) = g(x)
\end{align*}
Sätt:
\[
h(x) = G(x) - \tilde{G}(x)\mbox{ , } x\in \mathbb{I}
\]
Då gäller:
\begin{itemize}
    \item $h$ är kontinuerlig på $\mathbb{I}$, eftersom $G, \tilde{G}(x)$ deriverbar på $\mathbb{I}$ och alltså kontinuerlig på $\mathbb{I}$
    \item $h$ deriverbar på $\mathbb{I}$
    \item $h^\prime(x) = G^\prime(x) - \tilde{G}^\prime(x) = g(x) - g(x) = 0$, $\forall x \in \mathbb{I}$
\end{itemize}
Fixera $c \in \mathbb{I}$
\begin{align*}
    h(x) - h(c) = h^\prime(\xi)(x-c) = 0 \mbox{ , } \forall x \in \mathbb{I}
\end{align*}
Alltså finns konstant $C$, så att
\[
G(x) = \tilde{G}(x) + C \mbox{ , } \forall x \in \mathbb{I}
\]
Svar på fråga (1):\\
Om $g(x)$ är en kontinuerlig funktion så existerar primitiv funktion till $g(x)$
% section primitv_funktion (end)
\subsection{Beteckning} % (fold)
\label{sub:beteckning}
\[
\underbrace{\int g(x) dx}_\text{obestämd integral} = \underbrace{G(x)}_\text{primitiv funktion} + \underbrace{C}_\text{godtycklig konstant}
\]

% subsection beteckning (end)
\end{document}























