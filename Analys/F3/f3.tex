\documentclass{article}

\usepackage[utf8]{inputenc}
\usepackage{amsthm}
\usepackage{amssymb}
\usepackage{mathtools}
\usepackage{graphicx}
\usepackage{mdframed}
\usepackage{float}
\usepackage[top=1in, bottom=1.25in, left=1.25in, right=1.25in]{geometry}

\DeclarePairedDelimiter{\abs}{\lvert}{\rvert}
\DeclarePairedDelimiter{\norm}{\lvert \lvert}{\rvert \rvert}

\newtheoremstyle{break}% name
  {}%         Space above, empty = `usual value'
  {}%         Space below
  {\itshape}% Body font
  {}%         Indent amount (empty = no indent, \parindent = para indent)
  {\bfseries}% Thm head font
  {.}%        Punctuation after thm head
  {\newline}% Space after thm head: \newline = linebreak
  {}%         Thm head spec

\newtheorem{Def}{Definition}[section]

\theoremstyle{break}

\newtheorem{innerEx}{Exempel}[section]
\newtheorem{sats}{Sats}[section]
\newtheorem{Rem}{Anmärkning}[section]

\newenvironment{Ex}
{\begin{mdframed} \begin{innerEx} \vspace{3pt}}
{\vspace{3pt} \end{innerEx} \end{mdframed}}  

\newenvironment{bevis}
{\begin{mdframed} \begin{proof} \vspace{3pt}}
{\vspace{3pt} \end{proof} \end{mdframed}}

\usepackage{polynom}

\title{
     Analys\\
     Föreläsning 3
    \author{Erik Sjöström}
}
\begin{document}
\maketitle

\section{Polynomdivision} % (fold)
\label{sec:polynomdivision}

\begin{Ex}
    \begin{align*}
        &P(x) = 2x^3 - 3x^2 + 3x + 4 \mbox{ grad(P) = 3}\\
        &Q(x) = x^2 + 1 \mbox{ grad(Q) = 2}
    \end{align*}
    \[
    \polylongdiv{2x^3 - 3x^2 + 3x + 4}{x^2 + 1}
    \]

    Vi har:
    \[
    P(x) = \underbrace{2x - 3}_\text{grad 1 = grad(P) - grad(Q)} Q(x) + \underbrace{x + 7}_\text{grad 1 $<$ grad Q}
    \]
\end{Ex}
\textbf{\underline{Allmänt:}} Givet polynom $P(x), Q(x)$ där grad $Q \le grad(P)$.\\
Då finns polynom $K(x)$ (kvotpolynom) och $R(x)$ (restpolynom), sådana att:
\begin{align*}
	&P(x) = K(x)Q(x) + R(x)\\
	&grad(K) = grad(P) - grad(Q)\\
	&grad(R) < grad(Q)\\
	&\frac{P(x)}{Q(x)} = K(x) + \frac{R(x)}{Q(x)}
\end{align*}
% section polynomdivision (end)
\section{Elementära funktioner} % (fold)
\label{sub:element_ra_funktioner}
\textbf{Polynomfunktioner:}
\begin{align*}
	&f(x) = P(x), &&x \in \mathbb{R} = D_f
\end{align*}
\textbf{Rationella funktioner:}
\begin{align*}
	&f(x) = \frac{P_1(x)}{P_2(x)}, \mbox{ där } P_1(x), P_2(x) \mbox{ är polynom} &&x \in \{x \in \mathbb{R}: P_2(x) \neq 0\} = D_f
\end{align*}
\textbf{Trigonometriska funktioner:}
\begin{align*}
	&cos(x) &&sin(x) &&tan(x) = \frac{sin(x)}{cos(x)}
\end{align*}
\begin{center}
	FIGUR	
\end{center}
Vi mäter vinklar i radianer. En vinkel är x radioner om motsvarande cirkelbåge har längden x. Vi räknar vinklar med tecken (positiv i moturs riktning, negativ i medurs). Omkretsen av enhetscirkeln har längden $2\pi$ per definition.\\
Omvandling mellan grader och radianer. Givet en vinkel av storlek $t$ radianer, $\tilde{t}$ grader ges av:
\[
t = \tilde{t} \cdot \frac{2\pi}{360}
\]
Vi räknar från och med nu med radianer.\\
% subsection element_ra_funktioner (end)
Definition av $\cos(t), \sin(t)$:
\begin{center}
	FIGUR
\end{center}
Egenskaper för $\cos(t), \sin(t)$:
\begin{itemize}
	\item $\cos(t), \sin(t)$ $\in[-1, 1]$, $t \in \mathbb{R}$
	\item $\cos(t), \sin(t)$ $2\pi$-periodiska funktioner på $\mathbb{R}$
	\item $\cos(t)$, jämn funktion ty $\cos(-t) = cos(t)$, $t \in \mathbb{R}$
	\item $\sin(t)$, udda funktion ty $\sin(-t) = -\sin(t)$, $t \in \mathbb{R}$
	\item $\cos(t), \sin(t)$, för $t = 0, \frac{\pi}{6}, \frac{\pi}{4}, \frac{\pi}{3}, \frac{\pi}{2}$
	\begin{itemize}
		\item $\cos(0) = 1$, $\sin(0) = 0$
		\item $\cos(\frac{\pi}{2}) = 0$, $\sin(\frac{\pi}{2}) = 1$
		\item $\cos(\frac{\pi}{4}) = \sin(\frac{\pi}{4}) = \frac{1}{\sqrt{2}}$
		\item $\cos(\frac{\pi}{3}) = \frac{1}{2}$, $\sin(\frac{\pi}{3}) = \frac{\sqrt{3}}{2}$
		\item $\cos(\frac{\pi}{6}) = \frac{\sqrt{3}}{2}$, $\sin(\frac{\pi}{6}) = \frac{1}{2}$
	\end{itemize}
	\item $\cos(\pi + t) = - \cos(t)$
	\item $\sin(\pi + t) = - \sin(t)$
\end{itemize}
\textbf{Graferna för cos(t) och sin(t)}
\begin{center}
	FIGUR
\end{center}
\textbf{Additionsformler för cos, och sin}
\begin{itemize}
	\item $\cos(\alpha - \beta) = \cos(\alpha)\cos(\beta) + \sin(\alpha)\sin(\beta)$ \\
	Eftersom:
	\begin{align*}
		a &= (\cos(\alpha), \sin(\alpha))\\
		b &= (\cos(\beta), \sin(\beta))\\
		a \cdot b &= \abs{a}\abs{b} \cdot \cos(\alpha - \beta)
	\end{align*}
	Kan även uttryckas som:
	\[
	\cos(\alpha) \cdot \cos(\beta) + \sin(\alpha) \cdot \sin(\beta), \mbox{ ty } \abs{a} = \abs{b} = 1
	\]
	\textbf{Trigonometriska ettan:}
	\begin{align*}
		\cos^2(\alpha) + \sin^2(\alpha) &= 1\\
		\cos^2(\beta) + \sin^2(\beta) &= 1 
	\end{align*}
	\begin{equation}
	    \cos(\alpha - \beta) = \cos(\alpha) \cdot \cos(\beta) + \sin(\alpha) \cdot \sin(\beta)
	\end{equation}
	Ersätt $\beta$ med $-\beta$ i (1) $\Rightarrow$
	\item $\cos(\alpha + \beta)$
	\begin{align}
		\cos(\alpha + \beta) &= \cos(\alpha - (-\beta)) = \cos(\alpha)\overbrace{\cos(-\beta)}^\text{jämn} + \sin(\alpha)\overbrace{\sin(-\beta)}^\text{udda} \\
		&= \cos(\alpha)\cos(\beta) - \sin(\alpha)\sin(\beta)
	\end{align}
	\item $\cos(\alpha + \frac{\pi}{2}) = \cos(\alpha) \cdot \overbrace{\cos(\frac{\pi}{2})}^{0} - \sin(\alpha)\overbrace{\sin(\frac{\pi}{2})}^{1} = -\sin(\alpha)$
	\item $\cos(\frac{\pi}{2} - \alpha) = \overbrace{\cos(\frac{\pi}{2})}^{0}\cos(\alpha) + \overbrace{\sin(\frac{\pi}{2})}^{1}\sin(\alpha) = \sin(\alpha)$
	\item $\sin(\alpha + \beta)$
	\begin{align}
		\sin(\alpha + \beta) &= \cos(\frac{\pi}{2} - (\alpha + \beta))\\
		&= \cos((\frac{\pi}{2} - \alpha)- \beta) \\
		&= \underbrace{\cos(\frac{\pi}{2} - a)}_{\sin(\alpha)}\cos(\beta) + \underbrace{\sin(\frac{\pi}{2} - \alpha)}_{\cos(\alpha)}\sin(\beta) \\
		&= \sin(\alpha)\cos(\beta) + \cos(\alpha)\sin(\beta)
	\end{align}
	Ersätt $\beta$ med $-\beta$ i (7) vi får då:
	\item $\sin(\alpha - \beta) = \sin(\alpha)\cos(\beta) - \cos(\alpha)\sin(\beta)$
\end{itemize}
\textbf{Formler för dubbla vinkeln för sin och cos:}
\begin{itemize}
	\item $\cos(2t)$
	\begin{align*}
		\cos(2t) = \cos(t + t) &= \cos(t)\cos(t)- \sin(t)\sin(t)\\
		&= \cos^2(t) - \sin^2(t)\\
		&= \mbox{ trigonometriska ettan }\\
		&= \cos^2(t) - (1 - \cos^2(t))\\
		&= 2\cos^2(t) - 1\\
		&= \mbox{ trigonometriska ettan }\\
		&= 1 - 2\sin^2(t)
	\end{align*}
	\item $\sin(2t) = 2\cos(t)\sin(t)$
\end{itemize}
\textbf{Produktformler}
\begin{itemize}
	\item $\sin(\alpha \pm \beta) = \sin(\alpha)\cos(\beta) \pm \cos(\alpha)\sin(\beta)$
	\item $\sin(\alpha + \beta) + \sin(\alpha - \beta) = 2 \sin(\alpha)\cos(\beta)$
	\item $\sin(\alpha)\cos(\beta) = \frac{1}{2}[\sin(\alpha + \beta) + \sin(\alpha - \beta)]$
\end{itemize}
\textbf{Definition av tan(t)}
\begin{align*}
	&\tan(t) = \frac{\sin(t)}{\cos(t)} &&t \in \{t \in \mathbb{R}: \cos(t) \neq 0 \} = \mathbb{R}\setminus\{\frac{\pi}{2} + m \cdot \pi: \mbox{ m heltal}\}
\end{align*}

\begin{center}
	FIGUR
\end{center}
Triangeln ... och ... är likformiga $\Rightarrow$ $\frac{a}{1} = \frac{\sin(t)}{\cos(t)} = \tan(t)$\\
\textbf{Additionsregel för tan(t)}
\begin{itemize}
	\item $\tan(\alpha + \beta)$
	\begin{align*}
		\tan(\alpha + \beta) &= \frac{\sin(\alpha + \beta)}{\cos(\alpha + \beta)} \\
		&= \frac{\sin(\alpha)\cos(\beta) + \cos(\alpha)\sin(\beta)}{\cos(\alpha)\cos(\beta) - \sin(\alpha)\sin(\beta)}\\
		&= \frac{\tan(\alpha) + \tan(\beta)}{1 - \tan(\alpha)\tan(\beta)}
	\end{align*}
\end{itemize}
\end{document}























