\documentclass{article}

\usepackage[utf8]{inputenc}
\usepackage{amsthm}
\usepackage{amssymb}
\usepackage{mathtools}
\usepackage{graphicx}
\usepackage{mdframed}
\usepackage{float}
\usepackage[top=1in, bottom=1.25in, left=1.25in, right=1.25in]{geometry}

\DeclarePairedDelimiter{\abs}{\lvert}{\rvert}
\DeclarePairedDelimiter{\norm}{\lvert \lvert}{\rvert \rvert}

\newtheoremstyle{break}% name
  {}%         Space above, empty = `usual value'
  {}%         Space below
  {\itshape}% Body font
  {}%         Indent amount (empty = no indent, \parindent = para indent)
  {\bfseries}% Thm head font
  {.}%        Punctuation after thm head
  {\newline}% Space after thm head: \newline = linebreak
  {}%         Thm head spec

\newtheorem{Def}{Definition}[section]

\theoremstyle{break}

\newtheorem{innerEx}{Exempel}[section]
\newtheorem{sats}{Sats}[section]
\newtheorem{Rem}{Anmärkning}[section]

\newenvironment{Ex}
{\begin{mdframed} \begin{innerEx} \vspace{3pt}}
{\vspace{3pt} \end{innerEx} \end{mdframed}}  

\newenvironment{bevis}
{\begin{mdframed} \begin{proof} \vspace{3pt}}
{\vspace{3pt} \end{proof} \end{mdframed}}

\title{
     Analys\\
     Föreläsning 6
    \author{Erik Sjöström}
}
\begin{document}
\maketitle

\section{Derivata} % (fold)
\label{sec:derivata}
$f(x)$ är en reell funktion
\begin{center}
    FIGUR   
\end{center}
\[
f^\prime(a) = \lim\limits_{h \to 0}\frac{f(a + h) - f(a)}{h}
\]
\textbf{Beteckning:} 
\[
f^\prime(a) = \frac{df}{dc}(x) = Df(a) = ...
\]
\begin{Ex}
    \[
    f(x) = \sqrt(x), \mbox{ } D_f = \{x \in \mathbb{R}: x \ge 0\} = [0, \infty)
    \]
    \begin{center}
        FIGUR
    \end{center}
    Fixera $a > 0$. Betrakta:
    \begin{align*}
        \frac{f(a+h) - f(a)}{h} &= \frac{\sqrt{a + g} - \sqrt{a}}{h}\\
        &= \frac{(\sqrt{a+h} - \sqrt{a})(\sqrt{a+h} + \sqrt{a})}{h(\sqrt{a+h} + \sqrt{a})} \\
        &= \frac{(\sqrt{a+h})^2 - (\sqrt{a})^2}{h(\sqrt{a+h} + \sqrt{a})}\\
        &= \frac{a + h - a}{h(\sqrt{a+h} + \sqrt{a})} \\
        &= \frac{1}{\sqrt{a+h} + \sqrt{a}}
        \to \frac{1}{\sqrt{a} + \sqrt{a}}
        =\frac{1}{2\sqrt{a}} \mbox{ då } h \to 0
    \end{align*}
    Alltså:
    \[
    f^\prime(a) = \frac{1}{2\sqrt{a}}, \mbox{ } a > 0
    \]
    \begin{center}
        FIGUR
    \end{center}
\end{Ex}
Vi säger att $f(x)$ är \underline{deriverbar} i $x = a$ om:
\[
\lim\limits_{h \to 0}\frac{f(a+h) - f(a)}{h} = \lim\limits_{x \to a}\frac{f(x)- f(a)}{x-a}
\]
existerar.\\
Vi säger att $f(x)$ är \underline{kontinuerlig} i $x = a$ om:
\[
\lim\limits_{x \to a}f(x) = f(a) \mbox{ dvs } \lim\limits_{x \to a}(f(x) - f(a)) = 0
\]
\begin{sats}
    $f$ deriverbar i $a$ $\Rightarrow$ $f$ kontinuerlig i $a$
\end{sats}
\begin{bevis}
    Betrakta: ($x \neq a$)
    \begin{align*}
        f(x) - f(a) &= \underbrace{\frac{f(x) - f(a)}{x-a}}_{\to f^\prime(a) \mbox{ då } x \to a} \cdot \underbrace{(x - a)}_{\to 0 \mbox{ då } x \to 0}
    \end{align*}
    $f^\prime(a)$ existerar eftersom $f$ deriverbar i $x=a$\\
    \textbf{Slutsats:}
    \[
    \lim\limits_{x \to a}(f(x) - f(a)) = 0
    \]
    Vi har att $f$ kontinuerlig i $x=a$
\end{bevis}
\begin{Ex}
    \[
    f(x) = \abs{x}
    \]
    \begin{center}
        FIGUR
    \end{center}
    Detta exempel visar en kontinuerlig funktion som inte är deriverbar i $x = 0$
\end{Ex}
\subsection{Vänster och högerderivata} % (fold)
\label{sub:v_nster_och_h_gerderivata}
$f(x)$ har vänsterderivatan i $a$ om:
\[
\underbrace{\lim\limits_{h \to 0-}}_{h < 0}\frac{f(a+h)-f(a)}{h} = \lim\limits_{x \to a-}\frac{f(x) f(a)}{x-a}
\]
$f(x)$ har högerderivata i $a$ om:
\[
\underbrace{\lim\limits_{h \to 0+}}_{h > 0}\frac{f(a+h)-f(a)}{h} = \lim\limits_{x \to a+}\frac{f(x) f(a)}{x-a}
\]
Om vänsterderivatan i $a$ = högerderivatan i $a$ så är funktionen deriverbar i $a$
% subsection v_nster_och_h_gerderivata (end)
\subsection{Deriveringsregler} % (fold)
\label{sub:deriveringsregler}
Antag att $f,g$ deriverbara i $a$. Då gäller:
\begin{itemize}
    \item $(f + g)^\prime(a) = f^\prime(a) + g^\prime(a)$
    \item $(f - g)^\prime(a) = f^\prime(a) - g^\prime(a)$
    \item $(cf)^\prime(a) = cf^\prime(a)$
    \item $(f \cdot g)^\prime(a) = f^\prime(a)\cdot g(a) + f(a) \cdot g^\prime(a)$
    \item $(\frac{1}{f})^\prime(a) = \frac{f^\prime(a)}{(f(a))^2}$ om $f(a) \neq 0$
\end{itemize}
\textbf{Kedjeregeln: } $f$ deriverbar i $g(a)$ och $g$ deriverbar i $a$
\begin{align*}
    \frac{d}{dx}f(g(a))|_{x=a} = f^\prime(g(a)) \cdot g^\prime(a)
\end{align*}
\begin{Ex}
    \begin{align*}
        \frac{d}{dx}(x + (\sqrt{x})^3) &= \underbrace{\frac{d}{dx}x}_{=1} + \underbrace{= \frac{d}{dx}(\sqrt{x})^3}_{f(g(x)) \mbox{ där, } f(x)=x^3, g(x)=\sqrt{x}} 
    \end{align*}
    där:
    \begin{align*}
        &f^\prime(x) = 3x^2, &&g^\prime(x) = \frac{1}{2}\frac{1}{\sqrt{x}}
    \end{align*}
    och alltså:
    \begin{align*}
        f^\prime(g(x)) \cdot g(x) &= 3(g(x))^2 \cdot \frac{1}{2}\frac{1}{\sqrt{x}}  \\
        &=3x \cdot \frac{1}{2} \frac{1}{\sqrt{x}} \\
        &=\frac{3}{2}\sqrt{x}
    \end{align*}
\end{Ex}
\textbf{Bevisargument för kedjeregeln:} $f$ deriverbar i $g(a)$ och $g$ deriverbar i $a$
\begin{align*}
    \frac{d}{dx}f(g(a))|_{x=a} = f^\prime(g(a)) \cdot g^\prime(a)
\end{align*}
Definera:
\[
R_u(k) = 
\begin{cases}
    \frac{f(u+k)-f(u)}{k} - f^\prime(u) & k \neq 0\\
    0 & k=0
\end{cases}
\]
om $f^\prime(u)$ existerar så gäller:
\begin{align*}
    R_u(k) \to 0, k \to 0
\end{align*}
för $u$ fixt.\\
Vi har:
\begin{align}
    f(u + k) - f(u) &= (f^\prime(u) + R_u(k)) \cdot k \mbox{ ,gäller för godtyckligt }k
\end{align}
Låt:
\begin{align*}
    &u = g(a)
    &&k= g(a + h) - g(a)
\end{align*}
Insättning i (1) ger:
\begin{align*}
    f(g(a+h)) - f(g(a)) &= (f^\prime(g(a))) + R_{g(a)}(g(a+h)-g(a)) \cdot (g(a+h)-g(a))
\end{align*}
Divdera med $h\neq 0$
\begin{align*}
    \frac{f(g(a+h)-f(g(a)))}{h} &= (f^\prime(g(a)) + R_{g(a)}(g(a+h)-g(a))) \cdot \frac{g(a+h)-g(a)}{h}
\end{align*}
Låt $h \to 0$:
\begin{align*}
    \frac{d}{dx}f(g(x))|_{x=a} &= f^\prime(g(a)) \cdot g^\prime(a), \mbox{ } h \to 0 
\end{align*}
\begin{Ex}
    $2 \times$ kedjeregeln.
    \begin{align*}
        \frac{d}{dx}f_1(f_2(f_3(x))) &= f^\prime_1(f_2(f_3(x))) \cdot f^\prime_2(f_3(x)) \cdot f^\prime_3(x)
    \end{align*}
\end{Ex}
\begin{Ex}
    Antag:
    \[
    f(x) = 
    \begin{cases}
        x^2 \cdot \sin(\frac{1}{x}) & x \neq 0\\
        0 & x = 0
    \end{cases}
    \]
    Vi ser direkt att $f(x)$ kontinuerlig i $x = 0$. \\
    \textbf{Fråga: } Är $f(x)$ deriverbar i $x=0$?\\
    För $x \neq 0$
    \begin{Rem}
        $\frac{d}{dx}\sin(x) = \cos(x)$, $\frac{d}{dx}\frac{1}{x} = - \frac{1}{x^2}$
    \end{Rem}
    \begin{align*}
        f^\prime(x) &= 2x \cdot \sin(\frac{1}{x}) + x^2 \cdot \cos(\frac{1}{x}) \cdot (- \frac{1}{x^2})\\
        &= \underbrace{2x \cdot \sin(\frac{1}{x})}_{\to 0 \mbox{ då } x\to 0} - \underbrace{\cos(\frac{1}{x})}_{\mbox{saknar gränsvärde då } x \to 0}
    \end{align*}
    Betrakta $h \neq 0$
    \begin{align*}
        \frac{f(0 + h)-f(0)}{h} &= \frac{h^2\sin(\frac{1}{h})-0}{h}\\
        &= h \cdot \sin(\frac{1}{h}) \to 0 \mbox{ då } h \to 0
    \end{align*}
\end{Ex}
% subsection deriveringsregler (end)
\textbf{Mål: } $\frac{d}{dx}\sin(x) = \cos(x)$
\begin{align*}
    \frac{d}{dx}\cos(x) &= \frac{d}{dx}(\sin(\frac{\pi}{2} - x)) \\
    &= \cos(\frac{\pi}{2}-x) \cdot (-1)\\
    &= -\cos(\frac{\pi}{2}-x)\\
    &= -\sin(x)
\end{align*}
\textbf{Lemma: }
\[
\lim\limits_{h \to 0}\frac{\sin(h)}{h} = 1
\]
\begin{bevis}
    För $0 < h < \frac{\pi}{2}$
    \begin{center}
        FIGUR
    \end{center}
\end{bevis}


% section derivata (end)

\end{document}























