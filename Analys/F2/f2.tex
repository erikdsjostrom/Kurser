\documentclass{article}

\usepackage[utf8]{inputenc}
\usepackage{amsthm}
\usepackage{amssymb}
\usepackage{mathtools}
\usepackage{graphicx}
\usepackage{mdframed}
\usepackage{float}
\usepackage[top=1in, bottom=1.25in, left=1.25in, right=1.25in]{geometry}

\DeclarePairedDelimiter{\abs}{\lvert}{\rvert}
\DeclarePairedDelimiter{\norm}{\lvert \lvert}{\rvert \rvert}

\newtheoremstyle{break}% name
  {}%         Space above, empty = `usual value'
  {}%         Space below
  {\itshape}% Body font
  {}%         Indent amount (empty = no indent, \parindent = para indent)
  {\bfseries}% Thm head font
  {.}%        Punctuation after thm head
  {\newline}% Space after thm head: \newline = linebreak
  {}%         Thm head spec

\newtheorem{Def}{Definition}[section]

\theoremstyle{break}

\newtheorem{innerEx}{Exempel}[section]
\newtheorem{sats}{Sats}[section]
\newtheorem{Rem}{Anmärkning}[section]

\newenvironment{Ex}
{\begin{mdframed} \begin{innerEx} \vspace{3pt}}
{\vspace{3pt} \end{innerEx} \end{mdframed}}  

\newenvironment{bevis}
{\begin{mdframed} \begin{proof} \vspace{3pt}}
{\vspace{3pt} \end{proof} \end{mdframed}}

\title{
     Analys\\
     Föreläsning 2
    \author{Erik Sjöström}
}
\begin{document}
\maketitle

\section{Absolutbelopp} % (fold)
\label{sec:absolutbelopp}

\begin{Def}
    \[
    \abs{a} =
    \begin{cases}
        a \mbox{ om } a \ge 0\\
        -a \mbox{ om } a < 0
    \end{cases}
    \]
\end{Def}
\subsection{Egenskaper $a,b \in \mathbb{R}$} % (fold)
\label{sub:egenskaper}
\begin{itemize}
    \item $\abs{-a} = \abs{a}$
    \item $a \le \abs{a}$
    \item $\abs{ab} = \abs{a} \cdot \abs{b}$
    \item $\abs{a + b} \le \abs{a} + \abs{b}$
    \item $\sqrt{a^2} = \abs{a}$
    \item $\abs{a} < b \Leftrightarrow -b < a < b$
    \item $\abs{a} \le b \Leftrightarrow -b \le a \le b$
\end{itemize}
% subsection egenskaper (end)
\textbf{Notera:}
\[
\abs{a + b}^2 = (\sqrt{(a + b)^2})^2 = (a + b)^2 = a^2 + 2 \cdot \overbrace{ab}^{\le \abs{ab} = \abs{a}\abs{b}} + b^2 = \abs{a}^2 + 2 \abs{a}\abs{b} + \abs{b}^2 = (\abs{a} + \abs{b})^2
\]
Vi får:
\[
\abs{a + b} \le \abs{a} + \abs{b}
\]

\begin{Ex}
    Bestäm de $x \in \mathbb{R}$ sådana att:
    \[
    \abs{x^2 - 5x + 6} < 1
    \]
    \textbf{\underline{Metod 1:}}
    Sätt $f(x) = \abs{x^2 - 5x + 6}$, $x \in \mathbb{R}$ \\
    Betrakta:
    \begin{align*}
        x^2 - 5x + 6 &= (x - \frac{5}{2})^2 - (\frac{5}{2})^2 + 6 \\
        &= \underbrace{\overbrace{(x - \frac{5}{2})^2 - \frac{1}{2}^2}^{= - (\frac{1}{2})^2 \mbox{ om } x = \frac{5}{2}}}_{\mbox{minimum}} \\
        &= (x - \frac{5}{2} + \frac{1}{2})(x - \frac{5}{2} - \frac{1}{2}) \\
        &= \overbrace{(x - 2)(x - 3)}^{\mbox{Avläs nollställen}}
    \end{align*}

\textbf{Rita grafen för $f(x)$}

\begin{center}
    FIGUR
\end{center}
\begin{center}
    FIGUR
\end{center}
\end{Ex}

Beräkna $x \in \mathbb{R}$ sådan att $f(x) = 1$
\begin{align*}
    f(x) = 1 &\Leftrightarrow x^2 - 5x + 6 \pm 1\\
    &\Leftrightarrow x^2 - 5x + 6 = 1\\
    &\Leftrightarrow x^2 - 5x + 5 = 0\\
    &\Leftrightarrow (x - \frac{5}{2})^2 - (\frac{5}{2})^2 + 5 = 0 \\
    &\Leftrightarrow (x - \frac{5}{2})^2 - (\frac{\sqrt{5}}{2})^2 = 0 \\
    &\Leftrightarrow (x - \frac{5}{2} + \frac{\sqrt{5}}{2})(x - \frac{5}{2} - \frac{\sqrt{5}}{2}) = 0\\
    &\Leftrightarrow x = \frac{5}{2} \pm \frac{\sqrt{5}}{2}
\end{align*}
\textbf{Svar:}
\[
\{x \in \mathbb{R}: f(x) < 1\} = (\frac{5}{2} - \frac{\sqrt{5}}{2}, \frac{5}{2} + \frac{\sqrt{5}}{2})
\]
\textbf{Metod 2:}
\[
\abs{x^2 - 5x + 6} < 1 \Leftrightarrow
\begin{cases}
    A: & -1 < x^2 - 5x + 6 \\
    B: & x^2 - 5x + 6 < 1
\end{cases}
\]
\textbf{Svar:}
\[
\{x \in \mathbb{R}: \abs{x^2 - 5x + 6}\} = \{x \in \mathbb{R}: \mbox{ Både A och B är uppfyllda}\}
\]
\textbf{Metod 3:}
Dela upp i olika fall där man kan "ta bort beloppstecknet"
\begin{itemize}
    \item Fall 1: $x < 2$
    \item Fall 2: $2 \le x < 3$
    \item Fall 3: $3 \le x$
\end{itemize}
% section absolutbelopp (end)

\section{Koordinatsystem} % (fold)
\label{sec:koordinatsystem}
\begin{center}
    FIGUR
\end{center}
\subsection{Räta linjens ekvation} % (fold)
\label{sub:r_ta_linjens_ekvation}
Givet en punkt $(x_0, y_0)$ på linjen, och linjen lutning $m$.
\[
y = m(x - x_0) + y_0
\]
Givet två punkter $(x_0, y_0)$, $(x_1, y_1)$ på linjen:
\[
y = \frac{y_1 - y_0}{x_1 - x_0}(x - x_0) + y_0
\]
% subsection r_ta_linjens_ekvation (end)
\subsection{Avståndet mellan två punkter} % (fold)
\label{sub:avst_ndet_mellan_tv_punkter}
\begin{center}
    FIGUR
\end{center}
\textbf{Pythagoras sats:} 
\begin{align*}
    d^2 &= \abs{x_0 - x_1}^2 + \abs{y_0 - y_1}^2 = (x_0 - x_1)^2 + (y_0 - y_1)^2\\ 
    \Leftrightarrow d &= \sqrt{(x_0 - x_1)^2 + (y_0 - y_1)^2}
\end{align*}
% subsection avst_ndet_mellan_tv_punkter (end)
\subsection{Cirkelns ekvation} % (fold)
\label{sub:cirkelns_ekvation}

Givet medelpunkten $(x_0, y_0)$ och radie $r > 0$:
\begin{center}
    FIGUR
\end{center}
\[
r = \sqrt{(x-x_0)^2 + (y-y_0)^2}
\]
Oftast skriver man:
\[
r^2 = (x-x_0)^2 + (y-y_0)^2
\]
% subsection cirkelns_ekvation (end)
% section koordinatsystem (end)

\section{Reella funktioner} % (fold)
\label{sec:reella_funktioner}

\begin{center}
    FIGUR
\end{center}
\textbf{Definitionsmängden} för $f: D_f$\\
\textbf{Värdemängden} för $f: V_f = \{f(x): x \in D_f\}$ \\
\textbf{Reella funktioner:}
\begin{itemize}
    \item $D_f \subset \mathbb{R}$
    \item $V_f \subset \mathbb{R}$
\end{itemize}
$A \subset B$ betyder att A är en delmängd av B, ej nödvändigtvis äkta delmängd.
\begin{Ex}
    \begin{align}
        &f(x) = \sqrt{x + 1} && D_f = \{x \in \mathbb{R}: x+1 \ge 0\} = [-1,\infty)
    \end{align}
    Grafen för $f$ är mängden:
    \[
    \{(x,y) \in \mathbb{R}^2: x \in D_f \mbox{ och } y=f(x)\}
    \]
    \begin{center}
        FIGUR, eller kanske inte...
    \end{center}
\end{Ex}
\textbf{Definitionsmängdskonvention:} Om $D_f$ ej anges antas att $D_f$ är största möjliga mängd.

\subsection{Egenskaper} % (fold)
\label{sub:egeenskaper}
$f,g$ reella funktioner med $D_f = D_g$
\begin{itemize}
    \item $(f + g)(x) = f(x) + g(x), \mbox{ }x \in D_f = D_g$
    \item $(f - g)(x) = f(x) - g(x), \mbox{ }x \in D_f = D_g$
    \item $(f \cdot g)(x) = f(x) \cdot g(x), \mbox{ }x \in D_f = D_g$
    \item $(\frac{f}{g})(x) = \frac{f(x)}{g(x)}, \mbox{ } x \in D_f \cap \{x \in \mathbb{R}: g(x) \neq 0\}$
\end{itemize}
% subsection egeenskaper (end)
\subsection{Sammansatta funktioner} % (fold)
\label{sub:sammansatta_funktioner}
\begin{align*}
    &f \circ g(x) = f(g(x)), && x \in D_g \cap \{x \in \mathbb{R}: g(x) \in D_f\}
\end{align*}
\begin{center}
    FIGUR
\end{center}
\textbf{OBS:} Generellt så gäller: 
\[
f \circ g \neq g \circ f
\]
\begin{Ex}
    \begin{align}
        &f(x) = x^2 && g(x) = x + 1
    \end{align}
    \[
    f \circ g(x) = f(g(x)) = f(x + 1) = (x + 1)^2
    \]
    \[
    g \circ f(x) = g(f(x)) = g(x^2) = x^2 + 1
    \]
\end{Ex}

% subsection sammansatta_funktioner (end)
\subsection{Jämn/udda funktion} % (fold)
\label{sub:j_mn_udda_funktion}
Antag att $D_f = \mathbb{R}$. \\
$f$ kallas jämn funktion om:
\[
f(-x) = f(x), \mbox{ } \forall x \in \mathbb{R}
\]
$f$ kallas udda funktion om:
\[
f(-x) = -f(x), \mbox{ } \forall x \in \mathbb{R}
\]
\textbf{Jämn funktion}:
\begin{center}
    FIGUR
\end{center}
grafen är symmetrisk runt y-axeln.\\
\textbf{Udda funktion:}
\begin{center}
    FIGUR
\end{center}
grafen är symmetrisk kring origo.
\begin{Ex}
    Jämna funktioner: $1, \abs{x}, x^2, x^4, x^6, cos(x)$ \\
    Udda funktioner: $x, x^3, x^5, sin(x)$
\end{Ex}
\textbf{Men} $f(x) = x^2 + x^3$ är varken udda eller jämn.
% subsection j_mn_udda_funktion (end)
% section reella_funktioner (end)

\section{Polynom} % (fold)
\label{sec:polynom}
\[
P(x) = a_nx^n + a_{n-1}x^{n-1} + ... + a_1x^1 + a_0
\]
där
\[
a_n, a_{n-1}, ..., a_0 \in \mathbb{R} \mbox{ och } x \in \mathbb{R}
\]
Om $a_n \neq 0$, så säger vi att $P(x)$ har graden $n$, $grad(P) = n$\\
Om $P(r) = 0$ så säger vi att $r$ är ett nollställe till $P(x)$.\\
Vi säger också att $r$ är en rot till ekvationen $P(x) = 0$\\
Vi säger att polynomet $\widetilde{P}(x)$ är en faktor i polynomet $P(x)$ om det finns ett polynom $Q(x)$ sådant att:
\[
P(x) = \widetilde{P}(x) \cdot Q(x)
\]
\subsection{Polynomdivision} % (fold)
\label{sub:polynomdivision}
\textbf{\underline{Faktorsatsen:}} Givet ett polynom $P(x)$ av grad $\ge 1$. Då gäller:
\begin{center}
    $r$ är ett nollställe till $P(x) \Longleftrightarrow x -r$ är en faktor i $P(x)$
\end{center}
\textbf{Division:} Givet $r \in \mathbb{R}$ finns polynom $Q(x)$ och $a \in \mathbb{R}$ så att:
\[
P(x) = (x-r)Q(x) + a
\]
\begin{center}
    $r$ nollställe till $P(x) \Longleftrightarrow a = 0 \Longleftrightarrow (x - r)$ faktor till $P(x)$
\end{center}
% subsection polynomdivision (end)
% section polynom (end)

\end{document}























