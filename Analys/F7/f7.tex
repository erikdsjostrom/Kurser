\documentclass{article}

\usepackage[utf8]{inputenc}
\usepackage{amsthm}
\usepackage{amssymb}
\usepackage{mathtools}
\usepackage{graphicx}
\usepackage{mdframed}
\usepackage{float}
\usepackage[top=0.75in, bottom=0.75in, left=0.75in, right=0.75in]{geometry}
\usepackage{gauss}

\usepackage{array}
\allowdisplaybreaks

\makeatletter
\newcounter{elimination@steps}
\newcolumntype{R}[1]{>{\raggedleft\arraybackslash$}p{#1}<{$}}
\def\elimination@num@rights{}
\def\elimination@num@variables{}
\def\elimination@col@width{}
\newenvironment{elimination}[4][0]
{
    \setcounter{elimination@steps}{0}
    \def\elimination@num@rights{#1}
    \def\elimination@num@variables{#2}
    \def\elimination@col@width{#3}
    \renewcommand{\arraystretch}{#4}
    \start@align\@ne\st@rredtrue\m@ne
}
{
    \endalign
    \ignorespacesafterend
}
\newcommand{\step}[2]
{
    \ifnum\value{elimination@steps}>0\sim\quad\fi
    \left[
        \ifnum\elimination@num@rights>0
            \begin{array}
            {@{}*{\elimination@num@variables}{R{\elimination@col@width}}
            |@{}*{\elimination@num@rights}{R{\elimination@col@width}}}
        \else
            \begin{array}
            {@{}*{\elimination@num@variables}{R{\elimination@col@width}}}
        \fi
            #1
        \end{array}
    \right]
    & 
    \begin{array}{l}
        #2
    \end{array}
    \addtocounter{elimination@steps}{1}
}
\makeatother

\DeclarePairedDelimiter{\abs}{\lvert}{\rvert}
\DeclarePairedDelimiter{\norm}{\lvert \lvert}{\rvert \rvert}

\newtheoremstyle{break}% name
  {}%         Space above, empty = `usual value'
  {}%         Space below
  {\itshape}% Body font
  {}%         Indent amount (empty = no indent, \parindent = para indent)
  {\bfseries}% Thm head font
  {.}%        Punctuation after thm head
  {\newline}% Space after thm head: \newline = linebreak
  {}%         Thm head spec

\newtheorem{Def}{Definition}[section]

\theoremstyle{break}

\newtheorem{innerEx}{Exempel}[section]
\newtheorem{sats}{Sats}[section]
\newtheorem{Rem}{Anmärkning}[]

\newenvironment{Ex}
{\begin{mdframed} \begin{innerEx} \vspace{3pt}}
{\vspace{3pt} \end{innerEx} \end{mdframed}}  

\newenvironment{bevis}
{\begin{mdframed} \begin{proof} \vspace{3pt}}
{\vspace{3pt} \end{proof} \end{mdframed}}


\title{
     Analys\\
     Föreläsning 7
    \author{Erik Sjöström}
}
\begin{document}
\maketitle

\section{Kedjeregeln} % (fold)
\label{sec:kedjeregeln}
\[
f \circ g(x) = f(g(x))
\]
\[
\frac{d}{dx}f(x) = f^\prime(g(x))g^\prime(x)
\]
\begin{sats}
    \[
    \lim\limits_{h \to o}\frac{\sin(h)}{h}=1
    \]
\end{sats}
\begin{sats}
    \[
    \frac{d}{dx}\sin(x) = \cos(x)
    \]
\end{sats}
\begin{bevis}
	\begin{align*}
		\frac{\sin(x + h)-\sin(h)}{h} &= \{\sin(a + b) = \sin(a)\cos(b) + \sin(b)\cos(a)\}\\
		&= \frac{1}{h}(\sin(x)\cos(h) + \sin(h)\cos(x) - \sin(x))\\
		&= \underbrace{\frac{\sin(h)}{h}}_{\to 1} \cdot \cos(x) + \frac{1}{h}(\underbrace{\cos(h)}_{-2\sin^2(\frac{h}{2})} - 1) \cdot \sin(x) \\
		&= \frac{\sin(h)}{h}\cos(x) \overbrace{- 2 \cdot \underbrace{\frac{\sin^2(\frac{h}{2})}{(\frac{h}{2})^2}}_{\to 1} \cdot \underbrace{h}_{\to 0} \cdot \frac{1}{4}}^{\to 0} \cdot \sin(x)
	\end{align*}
	Eftersom $\cos(a + b) = \cos(a)\cos(b) - \sin(a)\sin(b)$\\
	och $\cos(2a) = \cos^2(a) - \sin^2(a) = $ {trigettan} $= 1 - 2\sin^2(a)$\\
	\textbf{Alltså:} \[
	\frac{d}{dx}\sin(x) = \cos(x)
	\]
\end{bevis}
\begin{sats}
    \[
    \frac{d}{dx}\cos(x) = -\sin(x)
    \]
    \textbf{Bevis:} Se förra föreläsningen
\end{sats}
\begin{sats}
    \[
    \frac{d}{dx}\tan(x) = 1 + \tan^2(x) = \frac{1}{\cos^2(x)}
    \]
\end{sats}
\begin{bevis}
	\begin{align*}
		\frac{d}{dx}\tan(x) &= \frac{d}{dx}\frac{\sin(x)}{\cos(x)}\\
		&= \frac{(\frac{d}{dx}\sin(x))\cos(x) - \sin(x)(\frac{d}{dx}\cos(x))}{\cos^2(x)} \\
		&= \frac{\cos^2(x) + \sin^2(x)}{\cos^2(x)} \\
		&= \begin{cases}
			\frac{1}{\cos^2(x)}\\
			1 + (\frac{\sin(x)}{\cos(x)})^2 = 1 + (\tan(x))^2
		\end{cases}\\
	\end{align*}
\end{bevis}
\begin{Ex}
    \[
    f(x) = 3 \cdot \sin^2(4\tan(\cos(x)))
    \]
    \begin{align*}
    	\frac{d}{dx}f(x) &= 3 \cdot \frac{d}{dx}(\sin^2(4\tan(\cos(x)))) \\
    	&= 3 \cdot 2 \cdot \sin(4\tan(\cos(x))) \cdot \underbrace{\frac{d}{dx}\sin(4\tan(\cos(x)))}_{\cos(4\tan(\cos(x))) \cdot \underbrace{\frac{d}{dx}(3\tan(\cos(x))}_{4(1 + \tan^2(\cos(x))) \cdot \underbrace{\frac{d}{dx}\cos(x)}_{-\sin(x)}}} \\
    	&= -24 \sin(4\tan(\cos(x))) \cdot \cos(4\tan(\cos(x))) \cdot (1 + \tan^2(\cos(x))) \cdot \sin(x)
    \end{align*}
\end{Ex}
% section kedjeregeln (end)
\section{Implicit derivering} % (fold)
\label{sec:implicit_derivering}
\begin{Ex}
    \[
    \begin{cases}
    	y(x)\sin(x) = x^3 + \cos(y(x))\\
    	y(0) = \frac{\pi}{2}
    \end{cases}
    \]
    Beräkna $y^\prime (0)$\\
    Vi får:
    \[
    \frac{d}{dx}(y(x)\sin(x)) = \frac{d}{dx}(x^3 + \cos(y(x)))
    \]
    \begin{align*}
    	y ^\prime(x)\sin(x) + y(x)\cos(x) = 3x^2 + (-\sin(y(x))) \cdot y ^\prime(x)
    \end{align*}
    Sätt $y x = 0$. Vi får:
    \begin{align*}
    	y ^\prime(0) \cdot 0 + \underbrace{y(0) \cdot 1}_{\frac{\pi}{2}} = 0 - \underbrace{\sin(y(0))}_{1} \cdot y ^\prime(0) \\
    \end{align*}
    Dvs: 
    \[
    y ^\prime(0) = - \frac{\pi}{2}
    \]
\end{Ex}
\begin{Ex}
    \[
    \frac{d}{dx}x^{\frac{1}{n}}
    \]
    $n$ är ett positivt heltal.\\
    \textbf{Notera: }
    \[
    (x^{\frac{1}{n}})^n = x
    \]
    Implicit derivering:
    \begin{align*}
    	n \cdot (x^{\frac{1}{n}})^{n-1} \cdot \frac{d}{dx}x^{\frac{1}{n}} = 1
    \end{align*}
    Vi får:
    \begin{align*}
    	\frac{d}{dx}x^{\frac{1}{n}} = \frac{1}{n}(x^{\frac{1}{n}(n -1)})^{-1} &= \frac{1}{n}(x^{1- \frac{1}{n}})^{-1} \\
    	&= \frac{1}{n}x^{\frac{1}{n}-1}
    \end{align*}
\end{Ex}
\begin{Ex}
    \[
    \frac{d}{dx}x^{\frac{m}{n}}
    \]
    Dubbelkolla!   asdålawd
    \begin{align*}
    	\frac{d}{dx}(x^{\frac{1}{n}})^m &= m \cdot (x^{\frac{1}{m}})^{m-1} \cdot \frac{d}{dx} x^{\frac{1}{n}} \\
    	&= \frac{m}{n}x^{\frac{m}{n} - \frac{1}{m} + \frac{1}{m} -1} \\
    	&= \frac{m}{n}x^{\frac{m}{n}-1}
    \end{align*}
\end{Ex}

Antag att $f:[a,b] \rightarrow \mathbb{R}$\\
Vi säger att: $f$ växande på $[a,b]$ om $f(x) \le f(\tilde{x})$ för alla $a \le x \le \tilde{x} \le b$
\begin{center}
	FIGUR
\end{center}
$f$ \underline{strängt växande} på $[a,b]$ om $f(x) < f(\tilde{x}$ för alla $a \le x < \tilde{x} \le b$
\begin{center}
	FIGUR
\end{center}
\underline{Avtagande} ... om $f(x) \ge f(\tilde{x})$ för alla $a \le x < \tilde{x} \le b$ osv fyll i asfafasf\\
Om $f$ är växande är $-f$ avtagande och omvänt.\\
Om $f$ är strängt växande är $-f$ strängt avtagande och omvänt.\\
Vi säger att $f$ har ett heltal max i $x_0$ om det finns ett $\epsilon > 0$ sådant att:
\[
f(x) \le f(x_0) \mbox{ för all } x\in(x_0 - \epsilon, x_0 + \epsilon)
\]
\begin{center}
	FIGUR
\end{center}
$x_0$ kallas för lokal maxpunkt.\\
Vi säger att ... min ...
\[
f(x) \ge f(x_0) \mbox{ för all } x\in(x_0 - \epsilon, x_0 + \epsilon)
\]
...kallas för lokal minpunkt.\\
Vi säger att att $x_0$ är en lokal extrempunkt om $x_0$ är en lokal maxpunkt eller en lokal minpunkt.
\begin{sats}
    Antag att $f:[a,b] \rightarrow \mathbb{R}$ är en kontinuerlig funktion.\\
    Antag att $x_0 \in (a,b)$ är en lokal extrempunkt, och att $f$ är deriverbar i $x_0$.\\
    Då gäller:
    \[
    f ^\prime(x_0) = 0
    \]
\end{sats}
\begin{bevis}
	Antag att $x_0$ är en lokal maxpunkt.
	\begin{align*}
		h > 0: \mbox{ } \frac{f(x_0 + h) - f(x_0)}{h} \rightarrow \underbrace{f ^\prime(x_0)}_{\le 0} \mbox{ då } h \to 0+
	\end{align*}
	Samma differeanskvot:
	\begin{align*}
		h < 0: \mbox{ } \frac{f(x_0 + h)- f(x_0)}{h} \rightarrow \underbrace{f ^\prime(x_0)}_{\ge 0} \mbox{ då } h \to 0-
	\end{align*}
	Analogt bevisargument för $x_0$ lokal minpunkt.
\end{bevis}
\begin{sats}
    Antag att $f[a,b] \to \mathbb{R}$ är en kontinuerlig funktion.\\
    Antag att $f$ är deriverbar för alla $x \in (a,b)$, och att $f(a) = f(b)$\\
    Då finns:
    \[
    \xi \in (a,b) \mbox{ så att } f ^\prime(\xi) = 0
    \]
\end{sats}
\begin{bevis}
	\begin{center}
		FIGURER
	\end{center}
	\begin{center}
		FIGURER
	\end{center}
	\begin{center}
		FIGURER
	\end{center}
	\begin{center}
		FIGURER
	\end{center}
\end{bevis}
\begin{sats}
    \textbf{Medelvärdessatsen:}\\
    Antag att $f:[a,b] \to \mathbb{R}$ är en kontinuerlig funktion. \\
    Antag att $f$ är deriverbar i $(a,b)$\\
    Då finns:
    \[
    \xi \in (a,b) \mbox{ så att } f(b) - f(a) = f ^\prime(\xi)(b-a)
    \]
    \begin{center}
    	FIGUR
    \end{center}
\end{sats}
\begin{Ex}
    \begin{align*}
    	f(x) = x^3\\
    	f ^\prime(x) = 3x^2\\
    	f ^\prime(0) = 0
    \end{align*}
    \begin{center}
    	
    \end{center}
\end{Ex}

% section implicit_derivering (end)


\end{document}























