\documentclass{article}

\usepackage[utf8]{inputenc}
\usepackage{amsthm}
\usepackage{amssymb}
\usepackage{mathtools}
\usepackage{graphicx}
\usepackage{mdframed}
\usepackage{float}
\usepackage[top=0.75in, bottom=0.75in, left=0.75in, right=0.75in]{geometry}
\usepackage{gauss}

\usepackage{array}
\allowdisplaybreaks

\makeatletter
\newcounter{elimination@steps}
\newcolumntype{R}[1]{>{\raggedleft\arraybackslash$}p{#1}<{$}}
\def\elimination@num@rights{}
\def\elimination@num@variables{}
\def\elimination@col@width{}
\newenvironment{elimination}[4][0]
{
    \setcounter{elimination@steps}{0}
    \def\elimination@num@rights{#1}
    \def\elimination@num@variables{#2}
    \def\elimination@col@width{#3}
    \renewcommand{\arraystretch}{#4}
    \start@align\@ne\st@rredtrue\m@ne
}
{
    \endalign
    \ignorespacesafterend
}
\newcommand{\step}[2]
{
    \ifnum\value{elimination@steps}>0\sim\quad\fi
    \left[
        \ifnum\elimination@num@rights>0
            \begin{array}
            {@{}*{\elimination@num@variables}{R{\elimination@col@width}}
            |@{}*{\elimination@num@rights}{R{\elimination@col@width}}}
        \else
            \begin{array}
            {@{}*{\elimination@num@variables}{R{\elimination@col@width}}}
        \fi
            #1
        \end{array}
    \right]
    & 
    \begin{array}{l}
        #2
    \end{array}
    \addtocounter{elimination@steps}{1}
}
\makeatother

\DeclarePairedDelimiter{\abs}{\lvert}{\rvert}
\DeclarePairedDelimiter{\norm}{\lvert \lvert}{\rvert \rvert}

\newtheoremstyle{break}% name
  {}%         Space above, empty = `usual value'
  {}%         Space below
  {\itshape}% Body font
  {}%         Indent amount (empty = no indent, \parindent = para indent)
  {\bfseries}% Thm head font
  {.}%        Punctuation after thm head
  {\newline}% Space after thm head: \newline = linebreak
  {}%         Thm head spec

\newtheorem{Def}{Definition}[section]

\theoremstyle{break}

\newtheorem{innerEx}{Exempel}[section]
\newtheorem{sats}{Sats}[section]
\newtheorem{Rem}{Anmärkning}[]

\newenvironment{Ex}
{\begin{mdframed} \begin{innerEx} \vspace{3pt}}
{\vspace{3pt} \end{innerEx} \end{mdframed}}  

\newenvironment{bevis}
{\begin{mdframed} \begin{proof} \vspace{3pt}}
{\vspace{3pt} \end{proof} \end{mdframed}}


\title{
     Analys\\
     Föreläsning 4
    \author{Erik Sjöström}
}
\begin{document}
\maketitle

\section{Gränsvärden} % (fold)
\label{sec:gr_nsv_rden}

$f(x)$ reell funktion, $a \in \mathbb{R}$, \\
"$f(x)$ har gränsvärdet $L \in \mathbb{R}$ i x = a"

\begin{center}
	FIGURER	
\end{center}
\textbf{Beteckning:} $\lim\limits_{x \rightarrow a} f(x) = L$
\begin{center}
	FIGUR
\end{center}
$\lim\limits_{x \rightarrow a} f(x)$ existerar ej.
\begin{center}
	FIGUR
\end{center}
$\lim\limits_{x \rightarrow a} f(x)$ existerar ej.

\begin{Ex}
    \[
    f(x) = \frac{x^2 - 1}{x - 1}, \mbox{ } D_f = \{x \in \mathbb{R}: x \neq 1\}
    \]
    Vi ser:
    \[
    \frac{x^2 -1}{x-1} = \frac{(x-1)(x+1)}{x-1} = x+1
    \]
    Detta ger att $\lim\limits_{x \rightarrow 1}f(x) = 2$
\end{Ex}

\begin{Ex}
    \[
    f(x) = \frac{x}{\abs{x}}, \mbox{ } D_f = \{x \in \mathbb{R}: x \neq 0\}
    \]
    \textbf{Fråga: } Existerar $\lim\limits_{x \rightarrow 0}f(x)$ ?
    \[
    f(x) = 
    \begin{cases}
    	\frac{x}{x} & x > 0\\
    	\frac{x}{-x} & x < 0
    \end{cases}
    = 
    \begin{cases}
    	1 & x > 0\\
    	-1 & x < 0
    \end{cases}
    \]
    \textbf{Slutsats: } $\lim\limits_{x \to 0}f(x)$, existerar ej.
    \begin{center}
    	FIGUR
    \end{center}
\end{Ex}

\subsection{Formell definition av gränsvärde} % (fold)
\label{sub:formell_definition_av_gr_nsv_rde}

Givet en reell funktion $f(x)$ och $a \in \mathbb{R}$. Antag:
\[
D_f \cap \{x \in \mathbb{R}: 0 < \abs{x - a} < \delta \} \neq \phi, \mbox{ för alla } \delta > 0
\]
\begin{center}
	FIGUR
\end{center}
Vi säger att $f(x)$ har gränsvärdet $L \in \mathbb{R}$ i $x = a$, betecknat:
\[
\lim_{x \to a}f(x) = L, 
\]
eller:
\[
f(x) \rightarrow L, x \rightarrow a
\]
\textbf{om} för varje $\epsilon > 0$ finns $\delta = \delta(\epsilon) > 0$ sådant att $x \in D_f$ och $ 0 < \abs{x - a} < \delta \Longrightarrow \abs{f(x) - L} < \epsilon$
\begin{center}
	FIGUR
\end{center}
% subsection formell_definition_av_gr_nsv_rde (end)

\subsection{Ensidiga gränsvärden} % (fold)
\label{sub:ensidiga_gr_nsv_rden}

\[
\lim\limits_{x \to a+}f(x) = L
\]
Om för varje $\epsilon > 0$ finns $\delta(\epsilon) > 0$ så att:
\[
a \in D_f \mbox{ och } \overbrace{a < x < a + \delta}^{0 < \abs{x - a} < \delta \mbox{ och }x > a} \Longrightarrow \abs{f(x) - L} < \epsilon
\]

\[
\lim\limits_{x \to a-}f(x) = L
\]
Om för varje $\epsilon > 0$ finns $\delta(\epsilon) > 0$ så att:
\[
a \in D_f \mbox{ och } \overbrace{a - \delta < x < a}^{0 < \abs{x-a} < \delta \mbox{ och }x < a} \Longrightarrow \abs{f(x) - L} < \epsilon
\]

% subsection ensidiga_gr_nsv_rden (end)
\subsection{Gränsvärden som går mot oändligheten} % (fold)
\label{sec:gr_nsv_rden_som_g_r_mot_o_ndligheten}

% section gr_nsv_rden_som_g_r_mot_o_ndligheten (end)
\[
\lim\limits_{x \to \infty} f(x) = L
\]

\textbf{om} för varje $\epsilon > 0$ finns $\omega \in \mathbb{R}$ sådant att $x \in D_f$ och $x > \omega$ $\Longrightarrow$ $\abs{f(x) - L} < \epsilon$

\[
\lim\limits_{x \to -\infty} = L
\]
På samma sätt

\begin{Ex}
    \[
    f(x) = \sqrt{x^2 + x} - x 
    \]
    \textbf{Fråga: } Existerar $\lim\limits_{x\to \infty}f(x)$?\\
    \textbf{Notera:}
    \begin{align*}
    	\sqrt{x^2 + x} -x &= \frac{(\sqrt{x^2+x}-x)(\sqrt{x^2+x}+x)}{\sqrt{x^2+x}+x}\\
    	&= \frac{(\sqrt{x^2+x})^2 -x^2}{\sqrt{x^2+x}+x}\\
    	&= \frac{x^2+x-x^2}{\sqrt{x^2+x}+x}\\
    	&= \frac{x}{\sqrt{x^2+x}+x}\\
    	&= \{x > 0: \sqrt{x^2 +x} = \sqrt{x^2(1 + \frac{1}{x})} = \sqrt{x^2}\sqrt{1 + \frac{1}{x}} = x \underbrace{\sqrt{1 + \frac{1}{x}}}_{\rightarrow 1, x \rightarrow \infty}\} \\
    	&= \frac{1}{\sqrt{1 + \frac{1}{x}} + 1} \rightarrow \frac{1}{2}, x \rightarrow \infty
    \end{align*}
    \textbf{Slutsats:}
    \[
    \lim\limits_{x \to \infty}(\sqrt{x^2 + x} - x) = \frac{1}{2}
    \]
\end{Ex}
\begin{Ex}
    \[
    \lim\limits_{x \to \infty}\frac{3x^2 + x -2}{x^3-1}
    \]
    \textbf{Notera:}
    \begin{align*}
    	\frac{3x^2+x-2}{x^3-1} &= \frac{x^2(3 + \frac{1}{x} - \frac{2}{x^2})}{x^3(1 - \frac{1}{x^3})}\\
    	&= \frac{1}{x} \cdot \frac{3 + \frac{1}{x} - \frac{2}{x^2}}{1 - \frac{1}{x^3}} \longrightarrow 0, x \rightarrow \infty
    \end{align*}
\end{Ex}

\subsection{Oegentliga gränsvärden} % (fold)
\label{sub:oegentliga_gr_nsv_rden}
Vi skriver ($f(x)$ reell funktion, $a \in \mathbb{R}$):
\[
f(x) \rightarrow \infty, \mbox{ } x \rightarrow a
\]
\textbf{om} för varje $\omega \in \mathbb{R}$ finns $\delta = \delta(\omega) > 0$ sådant att $x \in D_f$ och $0 < \abs{x - a} < \delta \Longrightarrow f(x) > \omega$
\begin{center}
	FIGUR
\end{center}
% subsection oegentliga_gr_nsv_rden (end)

\subsection{Gränsvärdesregler} % (fold)
\label{sub:gr_nsv_rdesregler}
Antag:
\[
\lim\limits_{x \to a}f(x) = L
\]
och:
\[
\lim\limits_{x \to a}g(x) = M
\]
Då gäller:
\begin{itemize}
	\item $\lim\limits_{x \to a}(f(x) \pm g(x)) = L \pm M$
	\item $\lim\limits_{x \to a}(f(x) \cdot g(x)) = L \cdot M$
	\item $\lim\limits_{x \to a}(\frac{f(x)}{g(x)}) = \frac{L}{M}, \mbox{ om }M \neq 0$
	\item $f(x) < g(x)$ alla x $\rightarrow L \le M$
\end{itemize}

% subsection gr_nsv_rdesregler (end)
\subsection{Squeezing lemma} % (fold)
\label{sub:squeesing_lemma}

Om $f(x) \le h(x) \le g(x)$ för alla x, och $\lim\limits_{x \to a}f(x) = \lim\limits_{x \to a}g(x)$, så gäller:
\[
\lim\limits_{x \to a}h(x)
\]
existerar, och är lika med:
\[
\lim\limits_{x \to a}f(x) = \lim\limits_{x \to a}g(x)
\]
\begin{center}
	FIGUR
\end{center}
% subsection squeesing_lemma (end)
\begin{Ex}
    \[
    f(x) = 3x^2 + x - 2, \mbox{ } a = 1
    \]
    Visa att:
    \[
    \lim\limits_{x \to 1}f(x) = 2
    \]
    Utgående från definitionen av gränsvärdet.
    \[
    D_f = \mathbb{R}
    \]
    Fixera godtyckligt $\epsilon > 0$ 
    \begin{align*}
    	\abs{f(x)-2} &= \abs{(3x^2 + x -2) -2} \\
    	&= \abs{3x^2 + x - 4} \\
    	&= \abs{3(x-1)^2 + x -1 -3(-2x + 1) + (x-1) + 1 - 4}\\
    	&= \abs{3(x-1)^2 + (x-1) + 7x -7} \\
    	&= \abs{3(x-1)^2 + 8(x - 1)} \\
    	&\le \abs{3(x-1)^2} + \abs{8(x-1)} (triangelolikheten)\\
    	&= 3 \abs{x-1} \abs{x-1} + 8 \abs{x-1}\\
    	&= \underbrace{(3 \abs{x-1} + 8)}_{\le 11} \cdot \abs{x-1} < \epsilon \mbox{ om $\abs{x-1} < \delta$, vi vill hitta $\delta > 0$}	
    \end{align*}
    Antag att $0 < \delta \le 1$. Kan välje $\delta = \frac{\epsilon}{11}$
\end{Ex}

% section gr_nsv_rden (end)

\end{document}























