\documentclass{article}

\usepackage[utf8]{inputenc}
\usepackage{amsthm}
\usepackage{amssymb}
\usepackage{mathtools}
\usepackage{graphicx}
\usepackage{mdframed}
\usepackage{float}
\usepackage[top=0.75in, bottom=0.75in, left=0.75in, right=0.75in]{geometry}
\usepackage{gauss}

\usepackage{array}
\allowdisplaybreaks

\makeatletter
\newcounter{elimination@steps}
\newcolumntype{R}[1]{>{\raggedleft\arraybackslash$}p{#1}<{$}}
\def\elimination@num@rights{}
\def\elimination@num@variables{}
\def\elimination@col@width{}
\newenvironment{elimination}[4][0]
{
    \setcounter{elimination@steps}{0}
    \def\elimination@num@rights{#1}
    \def\elimination@num@variables{#2}
    \def\elimination@col@width{#3}
    \renewcommand{\arraystretch}{#4}
    \start@align\@ne\st@rredtrue\m@ne
}
{
    \endalign
    \ignorespacesafterend
}
\newcommand{\step}[2]
{
    \ifnum\value{elimination@steps}>0\sim\quad\fi
    \left[
        \ifnum\elimination@num@rights>0
            \begin{array}
            {@{}*{\elimination@num@variables}{R{\elimination@col@width}}
            |@{}*{\elimination@num@rights}{R{\elimination@col@width}}}
        \else
            \begin{array}
            {@{}*{\elimination@num@variables}{R{\elimination@col@width}}}
        \fi
            #1
        \end{array}
    \right]
    & 
    \begin{array}{l}
        #2
    \end{array}
    \addtocounter{elimination@steps}{1}
}
\makeatother

\DeclarePairedDelimiter{\abs}{\lvert}{\rvert}
\DeclarePairedDelimiter{\norm}{\lvert \lvert}{\rvert \rvert}

\newtheoremstyle{break}% name
  {}%         Space above, empty = `usual value'
  {}%         Space below
  {\itshape}% Body font
  {}%         Indent amount (empty = no indent, \parindent = para indent)
  {\bfseries}% Thm head font
  {.}%        Punctuation after thm head
  {\newline}% Space after thm head: \newline = linebreak
  {}%         Thm head spec

\newtheorem{Def}{Definition}[section]

\theoremstyle{break}

\newtheorem{innerEx}{Exempel}[section]
\newtheorem{sats}{Sats}[section]
\newtheorem{Rem}{Anmärkning}[]

\newenvironment{Ex}
{\begin{mdframed} \begin{innerEx} \vspace{3pt}}
{\vspace{3pt} \end{innerEx} \end{mdframed}}  

\newenvironment{bevis}
{\begin{mdframed} \begin{proof} \vspace{3pt}}
{\vspace{3pt} \end{proof} \end{mdframed}}


\title{
     Analys\\
     Föreläsning 5
    \author{Erik Sjöström}
}
\begin{document}
\maketitle

\section{Förre föreläsningen} % (fold)
\label{sec:f_rre_f_rel_sningen}

\subsection{Gränsvärden} % (fold)
\label{sub:gr_nsv_rden}
\[
\lim\limits_{x \to a}f(x) = L
\]
$f$ är en reell funktion\\
$a \in \mathbb{R}$\\
\[
D_f \cap ((a - \delta, a) \cup (a, a + \delta)) \neq \emptyset
\]
för alla $\delta > 0$
\begin{center}
	FIGUR
\end{center}
% subsection gr_nsv_rden (end)
% section f_rre_f_rel_sningen (end)

\section{Kontinuitet} % (fold)
\label{sec:kontinuitet}
$f(x)$ är en reell funktion\\
$a \in \mathbb{R}$\\
Vi säger att $f(x)$ kontinuerlig i $a$ om:
\begin{itemize}
	\item $a \in D_f$
	\item $\lim\limits_{x \to a}f(x)$ existerar
	\item $f(a) = \lim\limits_{x \to a}f(x)$
\end{itemize}
Vi säger att $f(x)$ är en kontinuerlig funktion om $f(x)$ är kontinuerlig i $a$ för varje $a \in D_f$

\begin{Rem}
	Polynomfunktioner, rationella funktioner, trigonometriska funktioner är kontinuerliga funktioner.
\end{Rem}
\begin{Ex}
    \begin{align*}
    	&f(x) = \frac{1}{x} &&D_f = \{x \in \mathbb{R}: x \neq 0\}
    \end{align*}
    \begin{center}
    	FIGUR
    \end{center}
\end{Ex}
\begin{Rem}
	Summor, produkter och sammansättningar av kontinuerliga funktioner är också kontinuerliga.
\end{Rem}
\begin{Ex}
    $f,g$ kontinuerliga funktioner, $D_f = D_g$
    \[
    (f + g)(x) = f(x) + g(x)
    \]
    \textbf{Påstår:} $f+g$ kontinuerlig funktion\\
    Ska visa att $f+g$ är kontinuerlig i $a$ för varje $a \in D_f = D_g$\\
    Vi ser att:
    \begin{itemize}
    	\item $a \in D_f = D_g$
    	\item $\lim\limits_{x \to a}(f + g)(x)$ existerar ty:
    	\begin{align*}
    		\lim\limits_{x \to a}\overbrace{(f + g)(x)}^{f(x) + g(x)} &= \mbox{f,g kontinuerliga i a och alltså existerar $\lim\limits_{x \to a}f(x)$ och $\lim\limits_{x \to a}g(x)$} \\
    		&=\{\lim\limits_{x \to a}(f(x) + g(x)) \\
    		&= \lim\limits_{x \to a}f(x) + \lim\limits_{x \to a}g(x)\}
    	\end{align*}
    	\item $f(a) = \lim\limits_{x \to a}f(x)$ ty:
    	\begin{align*}
    		(f + g)(a) &= f(a) + g(a) \\
    		&= \{f(a) = \lim\limits_{x \to a}f(x), g(a) = \lim\limits_{x \to a}g(x)\}\\
    		&= \lim\limits_{x \to a}f(x) + \lim\limits_{x \to a}g(x) \\
    		&= \lim\limits_{x \to a}(f+g)(x)
    	\end{align*}
    \end{itemize}
\end{Ex}
\begin{Ex}
    Polynomfunktionen:
    \[
    f(x) = a_mx^m + a_{m-1}x^{m-1} + ... + a_1x + a_0
    \]
    Term: $ax^k$, k är ett positivt heltal\\
    $ax^k$ kan ses som $\underbrace{axx...x}_\text{k st}$
\end{Ex}
\begin{sats}
    \textbf{Största och minsta värde}\\
    $f$ reell funktion, $[a, b] \subset D_f$, $f$ är kontinuerlig.\\
    \textbf{Slutsats:} $\alpha,\beta \in [a,b]$ sådana att:
    \begin{align*}
    		&f(\alpha) \le f(x) \le f(\beta) \mbox{ } \forall x \in [a,b]
	\end{align*}
	$\alpha, \beta$ behöver inte vara entydigt bestämda.
	\begin{center}
		FIGUR
	\end{center}
\end{sats}
\begin{Rem}
	Vad kan hända om $f$ inte är kontinuerlig på $[a,b]$?
	\begin{center}
		FIGUR
	\end{center}
	Då gäller ej satsen.
\end{Rem}
\begin{Ex}
    \begin{center}
    	Figur
    \end{center}
\end{Ex}
\begin{sats}
    \textbf{Mellanliggande värden}\\
    $f$ är en reell funktion\\
    $[a,b] \subset D_f$\\
    $f$ är kontinuerlig funktion. 
    \[
    c \in (min(f(a), f(b)), max(f(a), f(b)))
    \]
    \begin{center}
    	FIGUR
    \end{center}
    \textbf{Slutsats:} Det finns $\xi \in (a,b)$ sådant att $f(\xi) = c$
\end{sats}
\begin{Ex}
    \[
    f(x) = x^3 - x - 1
    \]
    \textbf{Påstår:} $f(x) = 0$ har en rot i intervallet $[1,2]$
    \begin{Rem}
    	$f(x)$ är kontinuerlig.
    	\begin{align*}
    		&f(1) = -1 < 0 && f(2) = 5 > 0
    	\end{align*}
    	Med $c = 0$ ger satsen om mellanliggande värden att det finns $\xi \in (1,2)$ sådant att $f(\xi) = 0$, så $\xi$ är en rot till $f(x) = 0$
    \end{Rem}
\end{Ex}

\section{Derivata} % (fold)
\label{sec:derivata}

$f$ är en reell funktion\\
$a \in \mathbb{R}$\\
Antag att $(a - \delta, a + \delta) \subset D_f$
\begin{center}
	FIGUR
\end{center}
$h \neq 0$ kan vara $> 0$ eller $< 0$
\[
\lim\limits_{h \to 0}\frac{f(a+h) - f(a)}{h}
\]
Om gränsvärdet existerar så kallas vi det för derivatan av $f(x)$ i $x=a$\\
\textbf{Beteckning:} $f\textprime(a) = Df(a)=\frac{d}{dx}f(a)$\\
\textbf{Geometrisk tolkning:} $f\prime(a) = $ lutningen för tangenten till kurvan $y=f(x)$ i punkten $(a, f(a))$\\
Vi säger att $f$ är deriverbar i $a$ om:
\[
\lim\limits_{h \to 0}\frac{f(a + h)- f(a)}{h}
\]
existerar.

\begin{Ex}
    \[
    f(x) = x^m, \mbox{ m positivt heltal}
    \]
    Fixera $a \in \mathbb{R}$. Betrakta:
    \begin{align*}
    	\frac{f(a + h) - f(a)}{h} &= \frac{(a + h)^m - a^m}{h} \\
    	&= \{b^m - c^m = (b-c)(b^{m-1} + b^{m-2} \cdot c + b^{m-3}\cdot c^2 + ... + c^{m-1})\}\\
    	&= \frac{h}{h}(\underbrace{(a + h)^{m-1}}_{\to a^{m-1} \mbox{ då } h \to 0} + \underbrace{(a+h)^{m-2}}_{\to a^{m-1} \mbox{ då } h \to 0} \cdot a + ... + \underbrace{a^{m-1}}_{\to a^{m-1} \mbox{ då } h \to 0}) \to ma^{m-1}
    \end{align*}
\end{Ex}

% section derivata (end)

% section kontinuitet (end)

\end{document}























