\documentclass{article}

\usepackage[utf8]{inputenc}
\usepackage{amsthm}
\usepackage{amssymb}
\usepackage{mathtools}
\usepackage{graphicx}
\usepackage{mdframed}
\usepackage{float}
\usepackage[top=1in, bottom=1.25in, left=1.25in, right=1.25in]{geometry}

\DeclarePairedDelimiter{\abs}{\lvert}{\rvert}
\DeclarePairedDelimiter{\norm}{\lvert \lvert}{\rvert \rvert}

\newtheoremstyle{break}% name
  {}%         Space above, empty = `usual value'
  {}%         Space below
  {\itshape}% Body font
  {}%         Indent amount (empty = no indent, \parindent = para indent)
  {\bfseries}% Thm head font
  {.}%        Punctuation after thm head
  {\newline}% Space after thm head: \newline = linebreak
  {}%         Thm head spec

\newtheorem{Def}{Definition}[section]

\theoremstyle{break}

\newtheorem{innerEx}{Exempel}[section]
\newtheorem{sats}{Sats}[section]
\newtheorem{Rem}{Anmärkning}[section]

\newenvironment{Ex}
{\begin{mdframed} \begin{innerEx} \vspace{3pt}}
{\vspace{3pt} \end{innerEx} \end{mdframed}}  

\newenvironment{bevis}
{\begin{mdframed} \begin{proof} \vspace{3pt}}
{\vspace{3pt} \end{proof} \end{mdframed}}

\usepackage{listings}
\usepackage{color} %red, green, blue, yellow, cyan, magenta, black, white
\definecolor{mygreen}{RGB}{28,172,0} % color values Red, Green, Blue
\definecolor{mylilas}{RGB}{170,55,241}

\title{
     Linjär Algebra\\
     Föreläsning 20 - Om tentan
    \author{Erik Sjöström}
}
\begin{document}
\maketitle

\lstset{language=Matlab,%
    %basicstyle=\color{red},
    breaklines=true,%
    morekeywords={matlab2tikz},
    keywordstyle=\color{blue},%
    morekeywords=[2]{1}, keywordstyle=[2]{\color{black}},
    identifierstyle=\color{black},%
    stringstyle=\color{mylilas},
    commentstyle=\color{mygreen},%
    showstringspaces=false,%without this there will be a symbol in the places where there is a space
    numbers=left,%
    numberstyle={\tiny \color{black}},% size of the numbers
    numbersep=9pt, % this defines how far the numbers are from the text
    emph=[1]{for,end,break},emphstyle=[1]\color{red}, %some words to emphasise
    %emph=[2]{word1,word2}, emphstyle=[2]{style},
}

\section{Att kunna} % (fold)
\label{sec:att_kunna}

\begin{itemize}
    \item Läsa enkel matlabkod
    \item Frågor relaterade till lab3-lab6 kan kommma på tentan
\end{itemize}
\begin{Ex}
    Låt \textbf{A} vara en matris med 1000 rader och 1000 kolumner, och låt $\vec{x}$ vara en $1000 \times 1$ vektor. Ungefär hur många multiplikationer och additioner av matriselement behövs göras för att beräkna $\vec{v}_1$ respektive $\vec{v}_2$ i matlabsekvensen nedan?\\
    \begin{lstlisting}
        v1 = (A * A) * v;
        v2 = A * (A * v);
    \end{lstlisting}
    \textbf{Lösning: }\\
    Skalärprodukt 2 st $1000 \times 1$ vektorer ca:
    \[
        \begin{bmatrix}
            v_1\\
            v_2\\
            \vdots\\
            v_{1000}
        \end{bmatrix}
        \cdot
        \begin{bmatrix}
            u_1\\
            u_2\\
            \vdots\\
            u_{1000}
        \end{bmatrix}
         = v_1 \cdot u_1 + v_2 \cdot u_2 + ... + v_{1000} \cdot u_{1000} \approx 1000 \mbox{addition och multiplikation}
    \]
    \[
        \mathbf{A} \cdot \mathbf{A} \approx 1000^2 \mbox{ att beräkna, dvs } 1000 \cdot 1000^2 \mbox{ oberationer, totalt } 1000^3
    \]
    \[
        \vec{v}_1 = \overbrace{\overbrace{(\mathbf{A} \cdot \mathbf{A})}^{1000^3} \cdot \vec{v}}^{1000^2} \approx 1000^3 + 1000^2 \mbox{ op }
    \]
    \[
        \vec{v}_2 = \overbrace{\mathbf{A} \cdot \overbrace{(\mathbf{A} \cdot \vec{v})}^{1000^2}}^{1000^2} \approx 1000^2 + 1000^2 \mbox{ op}
    \]

\end{Ex}
\begin{Ex}
    \begin{center}
        Fråga i kod och algebra
    \end{center}
    \textbf{Lösning: } Vi ska lösa:
    \[
        \mathbf{A}^T \mathbf{A} \cdot \vec{x} = \mathbf{A}^T \cdot \vec{b}
    \]
    \begin{gather*}
        \mathbf{A}^T \mathbf{A} =
        \begin{bmatrix}
            1 & 2 & 3\\
            1 & 1 & 1
        \end{bmatrix} \cdot
        \begin{bmatrix}
            1 & 1\\
            2 & 1\\
            3 & 1
        \end{bmatrix} =
        \begin{bmatrix}
            14 & 6\\
            6 & 3
        \end{bmatrix} \\
        \mathbf{A}^T \vec{b} =
        \begin{bmatrix}
            1 & 2 & 3\\
            1 & 1 & 1
        \end{bmatrix}
        \begin{bmatrix}
            4 \\
            6 \\
            14
        \end{bmatrix} =
        \begin{bmatrix}
            58 \\
            24
        \end{bmatrix}
        \end{gather*}
        \[
            \begin{bmatrix}
            \begin{array}{cc|c}
                14 & 6 & 58\\
                6 & 3 & 24
            \end{array}
            \end{bmatrix}
            \sim ... \sim
            \begin{bmatrix}
            \begin{array}{cc|c}
                1 & 0 & 5\\
                0 & 1 & -2
            \end{array}
            \end{bmatrix}
            \Rightarrow \mbox{ svar } \vec{x} =
            \begin{bmatrix}
                5\\
                -2
            \end{bmatrix}
        \]

\end{Ex}

\begin{Ex}
    Bestäm lösningen till det överbestämda ekvationssystemet $\mathbf{A} \cdot \vec{x} = \vec{b}$ där:
    \begin{align*}
    &\mathbf{A} =
    \begin{bmatrix}
        1 & 1 \\
        2 & 1\\
        3 & 1
    \end{bmatrix}
    \end{align*}

    \textbf{Lösning: } Beräkna ortogonal projektion av $\vec{b}$ på planet $\vec{a} x + \vec{b} y+ \vec{x}z$
    \[
        \vec{b}_p = \vec{b} + \alpha \cdot \vec{n}
    \]
    där:
    \[
        \alpha = \frac{d - \vec{n} \cdot \vec{b}}{\vec{n} \cdot \vec{n}}
    \]
    Måste uttrycka planet på normalform (F3)\\
    Vi vet att vi har 2 vektorer:
    \begin{align*}
    &\begin{bmatrix} 1\\2\\3 \end{bmatrix}
    &&\begin{bmatrix} 1\\1\\1 \end{bmatrix}
    \end{align*}
    och en punkt, t.ex $x_0 = \begin{bmatrix} 0\\0\\0 \end{bmatrix}$\\
    En normalvektor:
    \[
        \begin{bmatrix} 1\\2\\3 \end{bmatrix} \times \begin{bmatrix} 1\\1\\1 \end{bmatrix} = \begin{bmatrix} -1\\2\\-1 \end{bmatrix}
    \]
    Minnesregel för kryssprodukt:
    \[
    \begin{vmatrix}
        e_1 & e_2 & e_3\\
        1 & 2 & 3\\
        1 & 1 & 1
    \end{vmatrix}
    = e_1
    \begin{vmatrix}
        2 & 3\\
        1 & 1
    \end{vmatrix}
     - e_2
     \begin{vmatrix}
         1 & 3\\
         1 & 1
     \end{vmatrix}
     + e_3
     \begin{vmatrix}
         1 & 2\\
         1 & 1
     \end{vmatrix}
    \]
    Vi får planets ekvations på normalform:
    \[
        \vec{a}(x -0) + \vec{b}(y -0) + \vec{c}(z -0) \Leftrightarrow \vec{a}x + \vec{b}y + \vec{c}z = 0 \Leftrightarrow -x + 2y -z = 0
    \]
    Projicera $\vec{b} = \begin{bmatrix} 4\\6\\14 \end{bmatrix}$
    \[
        \alpha = \frac{0 - \begin{bmatrix} -1\\2\\-1 \end{bmatrix} \cdot \begin{bmatrix} 4\\6\\14 \end{bmatrix}}{\begin{bmatrix} -1\\2\\-1 \end{bmatrix} \cdot \begin{bmatrix} -1\\2\\-1 \end{bmatrix}} = \frac{6}{6} = 1
    \]
    \[
        \vec{b}_p = \begin{bmatrix} 4\\6\\14 \end{bmatrix} + 1 \cdot \begin{bmatrix} -1\\2\\-1 \end{bmatrix} = \begin{bmatrix} 3\\8\\13 \end{bmatrix}
    \]
    Vi ska lösa:
    \[
        \begin{bmatrix}
            1 & 1\\
            2 & 1\\
            3 & 1
        \end{bmatrix}
        \begin{bmatrix} x_1 \\ x_2 \end{bmatrix} = \begin{bmatrix} 3 \\8\\13 \end{bmatrix}
    \]
    \[
        \begin{bmatrix}
        \begin{array}{cc|c}
            1 & 1 & 3\\
            2 & 1 & 8\\
            3 & 1 & 13
        \end{array}
        \end{bmatrix}
        \sim ... \sim
        \begin{bmatrix}
        \begin{array}{cc|c}
            1 & 1 & 3\\
            0 & -1 & 2\\
            0 & 0 & 0
        \end{array}
        \end{bmatrix}
        \Rightarrow
        \begin{cases}
            x_1 = 3 + 2 = 5\\
            x_2 = -2
        \end{cases}
    \]
\end{Ex}

% section att_kunna (end)

\section{Tenta 150115} % (fold)
\label{sec:tenta_150115}

\subsection{Fråga 1}
\textbf{Lösning: }
Bestäm  $\pi$:s ekvation på normalform:
\begin{gather*}
    \vec{n} = \begin{bmatrix} a\\b\\c \end{bmatrix} = \overrightarrow{AB} \times \overrightarrow{AC}
\end{gather*}
Planet ges då av:
\[
    a(x -x_0) + b(y - y_0) + c(z - z_0) = 0 \Rightarrow \pi: x -3y -3z = 3
\]
Linjens ekvation på parameterform:
\begin{gather*}
    \vec{v} = \overrightarrow{DE} = \begin{bmatrix} -1\\4\\3 \end{bmatrix}\\
    \vec{x} = D + \begin{bmatrix} -1\\3\\3 \end{bmatrix} \cdot t \Rightarrow
    \begin{cases}
        x = -t\\
        y = 1 + 4t\\
        z = 1 + 3t
    \end{cases}
\end{gather*}
Vilken t uppfyller planets ekvation?
\begin{gather*}
    (-t) -3 (1 + 4t) - 3(1 + 3t) = 3 \Leftrightarrow t = \frac{-9}{22}
\end{gather*}
Vilken punkt är det?
\begin{gather*}
    \begin{cases}
        x = 9/22\\
        y = 1 - 4 \cdot 9/22 = -7/11\\
        z = 1 - 3 \cdot 9/22 = 16/22
    \end{cases}
\end{gather*}
\subsection{Fråga 2}
\subsubsection{(a)}
För att anpassa en rät linje så måste detta gälla:
\[
    \begin{cases}
        4 = k \cdot 1 + m\\
        6 = k \cdot 2 + m\\
        14 = k \cdot 3 + m
    \end{cases}
\]
Man får då:
\[
    \begin{bmatrix} 4\\6\\14 \end{bmatrix} =
    \begin{bmatrix}
        1 & 1\\
        2 & 1\\
        3 & 1
    \end{bmatrix}
    \begin{bmatrix} k\\m \end{bmatrix}
\]
Man ser att detta inte har någon lösning.
\subsubsection{(b)}
Lös:
\[
    \mathbf{A}^T \mathbf{A} \cdot \vec{x} = \mathbf{A}^T \vec{b}
\]
\subsection{Fråga 3}
Lös:
\begin{gather*}
    \vec{x} \cdot \mathbf{A} = \mathbf{B} + 2 \vec{x}
\end{gather*}
där:
\begin{align*}
&A =
\begin{pmatrix}
    1 & -1 & 0\\
    1 & 3 & 1\\
    0 & 2 & 1
\end{pmatrix}
&&B =
\end{align*}
\textbf{Lösning:}
\begin{gather*}
    \vec{x} \cdot \mathbf{A} = \mathbf{B} + 2 \vec{x} \Leftrightarrow\\
    \vec{x} \cdot \mathbf{A} - 2 \vec{x} = \mathbf{B} \Leftrightarrow\\
    \vec{x} \cdot \mathbf{A} - \vec{x} \cdot 2 = \mathbf{B} \Leftrightarrow\\
    \vec{x} \cdot \overbrace{(\mathbf{A} - 2 \cdot \mathbf{I})}^\text{om inverterbar} = \mathbf{B} \Leftrightarrow\\
    \vec{x} = \mathbf{B} (\mathbf{A} - 2 \cdot \mathbf{I})^{-1}
\end{gather*}
Bestäm inversen:
\[
    \begin{bmatrix}
    \begin{array}{c|c}
        \mathbf{A} - 2 \cdot \mathbf{I} & \mathbf{I}
    \end{array}
    \end{bmatrix}
    \sim ... \sim
    \begin{bmatrix}
    \begin{array}{c|c}
        \mathbf{I} & (\mathbf{A} - 2 \cdot \mathbf{I})^{-1}
    \end{array}
    \end{bmatrix}
\]
\subsection{Fråga 4}
\subsubsection{(a)}
Egenvärden till \textbf{A}:
\[
    \mathbf{det}(\mathbf{A} - \mathbf{I} \cdot \lambda) = 0
\]
dvs
\[
    \begin{vmatrix}
        1 - \lambda & -1 & 0\\
        0 & 3 - \lambda & 1\\
        0 & 0 & 2 - \lambda
    \end{vmatrix}
    = (1 - \lambda) \cdot (3 - \lambda) \cdot (2 - \lambda) = 0 \Rightarrow
    \begin{cases}
        \lambda_1 = 1\\
        \lambda_2 = 2\\
        \lambda_3 = 3
    \end{cases}
\]
Bestäm egenvektorerna $\vec{v}_1, \vec{v}_2, \vec{v}_3$ till egenvärdena:
\begin{align*}
&(\mathbf{A} - \lambda_i \cdot \mathbf{I}) \vec{v}_i = \emptyset
&& \leftarrow \mbox{ lös för $\lambda_1, \lambda_2, \lambda_3$}
\end{align*}
\subsubsection{(b)}
Visa egenvektorerna linjärt obereoende (ty 3 linjärt oberoende vektorer i $\mathbb{R}^3$ utgör en bas för $\mathbb{R}^3$)\\
\textbf{Lösning:} Visa att
\[
    x_1 \vec{v}_1 + x_2 \vec{v}_2 + x_3 \vec{v}_3 = \emptyset
\]
Har bara lösningen:
\[
    x_1 0 x_2 = x_3 = 0
\]
\[
    \begin{bmatrix} \vec{v}_1 & \vec{v}_2 & \vec{v}_3 \end{bmatrix}
    \begin{bmatrix} x_1\\x_2\\x_3 \end{bmatrix} =
    \begin{bmatrix} 0\\0\\0 \end{bmatrix}
\]
\subsubsection{(c)}
$\vec{u} = \begin{bmatrix} 1 & 2 & 3 \end{bmatrix}^T$ Beräkna $\mathbf{A}^n \cdot \vec{u}$ där $n>0 \in \mathbb{R}$\\
\textbf{Lösning:} \textbf{A} diagonaliserbar, dvs
\begin{gather*}
    \mathbf{A} = \vec{v} \cdot \mathbf{D} \cdot \vec{v}^{-1}\\
    \mathbf{D} =
    \begin{bmatrix}
        \lambda_1 & 0 & 0\\
        0 & \lambda_2 & 0\\
        0 & 0 & \lambda_3
    \end{bmatrix}
\end{gather*}
\begin{align*}
& \vec{v} =
\begin{bmatrix}
    1 & 1 & 1\\
    0 & -2 & -1\\
    0 & 0 & 1
\end{bmatrix}
\vec{v}^{-1} =
\begin{bmatrix}
    1 & 1/2 & -1/2\\
    0 & -1/2 & -1/2\\
    0 & 0 & 1
\end{bmatrix}
\end{align*}
\begin{gather*}
    \mathbf{A}^n \vec{u} = (\vec{v} \cdot \mathbf{D} \cdot \vec{v}^{-1}) \cdot \vec{u} \\
    \vec{v} \cdot \mathbf{D}^n \cdot \vec{v}^{-1} \vec{u} =
\end{gather*}
\begin{align*}
&\begin{bmatrix} 1/2 & -5/2 & 3 \end{bmatrix}
&& \mathbf{D} = \begin{bmatrix} 1 & 0 & 0\\ 0 & 3 & 0\\0 & 0 & 2 \end{bmatrix}\\
&
\begin{bmatrix}
    1/2\\
    -5/2 \cdot 3^n\\
    3 \cdot 2^n
\end{bmatrix}
&& \mathbf{D}^n =
\begin{bmatrix}
    1^n & 0 & 0 \\
    0 & 3^n & 0\\
    0 & 0 & 2^n
\end{bmatrix}\\
&
\begin{bmatrix}
    1/2 - 5/2 \cdot 3^n + 3 \cdot 2^n\\
    5 \cdot 3^n/2 - 3 \cdot 2^n\\
    3 \cdot 2^n
\end{bmatrix}
\end{align*}


% section tenta_150115 (end)


\end{document}