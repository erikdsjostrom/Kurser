\documentclass{article}

\usepackage[utf8]{inputenc}
\usepackage{amsthm}
\usepackage{amssymb}
\usepackage{mathtools}
\usepackage{graphicx}
\usepackage{mdframed}
\usepackage{float}
\usepackage[top=0.75in, bottom=0.75in, left=0.75in, right=0.75in]{geometry}
\usepackage{gauss}

\usepackage{array}
\allowdisplaybreaks

\makeatletter
\newcounter{elimination@steps}
\newcolumntype{R}[1]{>{\raggedleft\arraybackslash$}p{#1}<{$}}
\def\elimination@num@rights{}
\def\elimination@num@variables{}
\def\elimination@col@width{}
\newenvironment{elimination}[4][0]
{
    \setcounter{elimination@steps}{0}
    \def\elimination@num@rights{#1}
    \def\elimination@num@variables{#2}
    \def\elimination@col@width{#3}
    \renewcommand{\arraystretch}{#4}
    \start@align\@ne\st@rredtrue\m@ne
}
{
    \endalign
    \ignorespacesafterend
}
\newcommand{\step}[2]
{
    \ifnum\value{elimination@steps}>0\sim\quad\fi
    \left[
        \ifnum\elimination@num@rights>0
            \begin{array}
            {@{}*{\elimination@num@variables}{R{\elimination@col@width}}
            |@{}*{\elimination@num@rights}{R{\elimination@col@width}}}
        \else
            \begin{array}
            {@{}*{\elimination@num@variables}{R{\elimination@col@width}}}
        \fi
            #1
        \end{array}
    \right]
    & 
    \begin{array}{l}
        #2
    \end{array}
    \addtocounter{elimination@steps}{1}
}
\makeatother

\DeclarePairedDelimiter{\abs}{\lvert}{\rvert}
\DeclarePairedDelimiter{\norm}{\lvert \lvert}{\rvert \rvert}

\newtheoremstyle{break}% name
  {}%         Space above, empty = `usual value'
  {}%         Space below
  {\itshape}% Body font
  {}%         Indent amount (empty = no indent, \parindent = para indent)
  {\bfseries}% Thm head font
  {.}%        Punctuation after thm head
  {\newline}% Space after thm head: \newline = linebreak
  {}%         Thm head spec

\newtheorem{Def}{Definition}[section]

\theoremstyle{break}

\newtheorem{innerEx}{Exempel}[section]
\newtheorem{sats}{Sats}[section]
\newtheorem{Rem}{Anmärkning}[]

\newenvironment{Ex}
{\begin{mdframed} \begin{innerEx} \vspace{3pt}}
{\vspace{3pt} \end{innerEx} \end{mdframed}}  

\newenvironment{bevis}
{\begin{mdframed} \begin{proof} \vspace{3pt}}
{\vspace{3pt} \end{proof} \end{mdframed}}

\title{
	 Linjär Algebra\\
	 Föreläsning 5
    \author{Erik Sjöström}
}
\begin{document}
\maketitle

\section{Matrisvektorprodukt} % (fold)
\label{sec:matrisvektorprodukt}
\begin{Def}
    Låt $\mathbf{A} = \begin{bmatrix} a_1,a_2, \dots, a_n \end{bmatrix}$ och $\vec{v} = \begin{bmatrix} v_1\\v_2\\ \vdots \\v_n \end{bmatrix}$ Då defineras matrisvektorprodukten som:
    \[
        \mathbf{A} \cdot \vec{v} = v_1 \cdot a_1 + v_2 \cdot a_2 + \dots + v_n \cdot a_n
    \]
\end{Def}
\begin{Ex}
    \begin{align*}
    &\mathbf{A} = \begin{bmatrix} 1&2&-1\\0&-5&3 \end{bmatrix} &&\vec{v} = \begin{bmatrix} 4\\3\\7 \end{bmatrix}
    \end{align*}

    \[
        \mathbf{A} \cdot \vec{v} = \begin{bmatrix} 1&2&-1\\0&-5&3 \end{bmatrix} \begin{bmatrix} 4\\3\\7 \end{bmatrix} = 4 \begin{bmatrix} 1\\0 \end{bmatrix} + 3 \begin{bmatrix} 2\\-5 \end{bmatrix} + 7 \begin{bmatrix} -1\\3 \end{bmatrix} = \begin{bmatrix} 4\\0 \end{bmatrix} + \begin{bmatrix} 6\\-15 \end{bmatrix} + \begin{bmatrix} -7\\21 \end{bmatrix} = \begin{bmatrix} 3\\6 \end{bmatrix}
    \]
\end{Ex}
\begin{Rem}
    Man får samma svar om man beräknar skalärprodukten mellan raderna i \textbf{A} och kolmunen i $\vec{v}$
\end{Rem}
Räkneregler för matrisvektorprodukten:\\
\textbf{A} och \textbf{B} är $(m \times n)$ matriser och $\vec{v}$, $\vec{u}$ $(nx1)$ kolumnvektorer, $k \in \mathbb{R}$
\begin{itemize}
	\item $\mathbf{A}(\vec{v} + \vec{u}) = \mathbf{A} \cdot \vec{u} + \mathbf{A} \cdot \vec{v}$
	\item $(\mathbf{A} + \mathbf{B}) \cdot \vec{v} = \mathbf{A} \cdot \vec{v} + \mathbf{B} \cdot \vec{v}$
	\item $\mathbf{A}(k \cdot \vec{v}) = k \cdot (\mathbf{A} \cdot \vec{v}) = (k \cdot \mathbf{A}) \cdot \vec{v}$
\end{itemize}
% section matrisvektorprodukt (end)

\section{Matrismultiplikation} % (fold)
\label{sec:matrismultiplikation}
\begin{Def}
    Låt en matris \textbf{A} vara av storlek $(m \times n)$, och en matris \textbf{B} vara av storlek $(n \times p)$ med kolumnerna $b_1,b_2, \dots ,b_p$ då är matrismultiplikationen $\mathbf{A} \cdot \mathbf{B}$ den $(m \times p$ matris vars kolumner är, $\mathbf{A} \cdot b_1, \mathbf{A} \cdot b_2, \dots, \mathbf{A} \cdot b_p$ dvs:
    \[
        \mathbf{A} \cdot \mathbf{B} = \mathbf{A} \cdot \begin{bmatrix} b_1,b_2, \dots, b_p \end{bmatrix} = \begin{bmatrix} \mathbf{A}b_1,\mathbf{A}b_2, \dots, \mathbf{A}b_p \end{bmatrix}
    \]
\end{Def}
\newpage
\begin{Ex}
   \begin{align*}
   &\mathbf{A} = \begin{bmatrix} 2&3\\1&-5 \end{bmatrix}_{2 \times 2} &\mathbf{B} = \begin{bmatrix} 4&3&6\\1&-2&3 \end{bmatrix}_{2 \times 3}
   \end{align*}
   \[
       \mathbf{A} \cdot \mathbf{B} = \begin{bmatrix} 2&3\\1&-5 \end{bmatrix} \cdot \begin{bmatrix} 4&3&6\\1&-2&3 \end{bmatrix} = \begin{bmatrix} \begin{bmatrix} 2&3\\1&-5 \end{bmatrix}\begin{bmatrix} 4\\1 \end{bmatrix} &\begin{bmatrix} 2&3\\1&-5 \end{bmatrix}\begin{bmatrix} 3\\-2 \end{bmatrix} &\begin{bmatrix} 2&3\\1&-5 \end{bmatrix} \begin{bmatrix} 6\\3 \end{bmatrix} \end{bmatrix}
   \]
   Problemt blir nu att räkna ut tre matrisvektorprodukter:
   \[
       \begin{bmatrix} 2 \cdot 4 + 3 \cdot 1 & 2 \cdot 3 + 3 \cdot (-2) & 2 \cdot 6	+ 3 \cdot 3 \\ 1 \cdot 4 - 5 \cdot 1 & 1 \cdot 3 + (-5)(-2) & 1 \cdot 6 - 5 \cdot 3 \end{bmatrix} = \begin{bmatrix} 11&0&21\\-1&13&-9 \end{bmatrix}
   \]
\end{Ex}
Det är oftast enklare att beräkna skalärprodukten mellan raderna i \textbf{A} och kolumnerna i \textbf{B}.
\begin{Rem}
    Storlekarna (typerna) måste stämma överens för att matrismultiplikation skall var definierad. Dvs om \textbf{A} är en $(m \times n)$ matris, och \textbf{B} en $(i \times j)$ matris, så är matrismultiplikationen:
    \[
        \mathbf{A}_{m \times n} \cdot \mathbf{B}_{i \times j}
    \]
    definierad omm $n = i$. Storleken på den resulterande matrisen blir $m \times j$
\end{Rem}
\subsection{Räkneregler} % (fold)
\label{sub:r_kneregler}
Låt \textbf{A},\textbf{B},\textbf{C} vara matriser, och $\vec{v}$ en kolumnvektor definierade så att följande operationer är giltiga. Då gäller:
\begin{itemize}
	\item $\mathbf{A} \cdot (k \cdot \mathbf{B}) = k \cdot (\mathbf{A} \cdot \mathbf{B}) = (k \cdot \mathbf{A}) \cdot \mathbf{B}$ \mbox{ för } $k \in \mathbb{R}$
	\item $\mathbf{A} \cdot (\mathbf{B} + \mathbf{C}) = \mathbf{A} \cdot \mathbf{B} + \mathbf{A} \cdot \mathbf{C}$
	\item $(\mathbf{B} + \mathbf{C}) \cdot \mathbf{A} = \mathbf{B} \cdot \mathbf{A} + \mathbf{C} \cdot \mathbf{A}$
	\item $\mathbf{A} \cdot (\mathbf{B} \cdot \vec{v}) = (\mathbf{A} \cdot \mathbf{B}) \cdot \vec{v}$
	\item $\mathbf{A} \cdot (\mathbf{B} \cdot \mathbf{C}) = (\mathbf{A} \cdot \mathbf{B}) \cdot \mathbf{C}$
\end{itemize}
\begin{Rem}
\textbf{   O\textbf{B}S!}
    \begin{itemize}
    	\item Det gäller oftast att $\mathbf{A} \cdot \mathbf{B} \neq \mathbf{B} \cdot \mathbf{A}$
    	\item Om $\mathbf{A} \cdot \mathbf{B} = \mathbf{A} \cdot \mathbf{C}$ kan man inte dra slutsatsen att $\mathbf{B} = \mathbf{C}$
    	\item Om $\mathbf{A} \cdot \mathbf{B} = \mathbb{O}$, kan man ej dra slutsatsen att $\mathbf{A} = \mathbb{O}$ eller $\mathbf{B} = \mathbb{O}$
    \end{itemize}
\end{Rem}

\begin{Ex}
    \begin{align*}
    &\mathbf{A} = \begin{bmatrix} -1&1\\1&-1 \end{bmatrix} &\mathbf{B} = \begin{bmatrix} 2&2\\2&2 \end{bmatrix}
    \end{align*}
    \[
        \mathbf{A} \cdot \mathbf{B} = \begin{bmatrix} -1&1\\1&-1 \end{bmatrix}\begin{bmatrix} 2&2\\2&2 \end{bmatrix} = \begin{bmatrix} 0&0\\0&0 \end{bmatrix} = \mathbb{O}
    \]
\end{Ex}
\begin{Ex}
    \begin{align*}
    &\mathbf{A} = \begin{bmatrix} 5&1\\3&-1 \end{bmatrix} &\mathbf{B} = \begin{bmatrix} 2&0\\4&3 \end{bmatrix}
    \end{align*}
    Då är:
    \begin{gather*}
    	\mathbf{A} \cdot \mathbf{B} = \begin{bmatrix} 5&1\\3&-1 \end{bmatrix} \begin{bmatrix} 2&0\\4&3 \end{bmatrix} = \begin{bmatrix} 14&3\\2&-3 \end{bmatrix}\\
    	\mathbf{B} \cdot \mathbf{A} = \begin{bmatrix} 2&0\\4&3 \end{bmatrix} \begin{bmatrix} 5&1\\3&-1 \end{bmatrix} = \begin{bmatrix} 10&2\\29&1 \end{bmatrix}
    \end{gather*}
\end{Ex}
% subsection r_kneregler (end)
% section matrismultiplikation (end)

\section{Transponat} % (fold)
\label{sec:transponat}
\begin{Def}
    Transponatet av en matris \textbf{A} ges av $\mathbf{A}^T$ vars kolumner är raderna i \textbf{A}.
\end{Def}
\begin{Ex}
    \[
        \mathbf{A} = \begin{bmatrix} -5&2\\1&-3\\0&4 \end{bmatrix} \mbox{ så är } \mathbf{A}^T = \begin{bmatrix} -5&1&0\\2&-3&4 \end{bmatrix}
    \]
\end{Ex}
\begin{Ex}
    Låt $\vec{u} = \begin{bmatrix} u_1\\u_2 \end{bmatrix}$ vara en kolumnvektor. Då är:
    \begin{align*}
    	&\mbox{ Den inre produkten: } \vec{u}^T \cdot \vec{u} = \begin{bmatrix} u_1&u_2 \end{bmatrix} \cdot \begin{bmatrix} u_1\\u_2 \end{bmatrix} = u^2_1 + u^2_2\\
    	&\mbox{Den yttre produkten: } \vec{u} \cdot \vec{u}^T = \begin{bmatrix} u_1\\u_2 \end{bmatrix} \cdot \begin{bmatrix} u_1&u_2 \end{bmatrix} = \begin{bmatrix} u^2_1&u_1u_2\\u_2u_1&u^2_2 \end{bmatrix}
    \end{align*}
\end{Ex}
\subsection{Räkneregler} % (fold)
\label{sub:r_kneregler2}
Låt \textbf{A},\textbf{B} vara matriser så att följande operationer är giltiga, $k \in \mathbb{R}$.
\begin{itemize}
	\item $(\mathbf{A}^T)^T = \mathbf{A}$
	\item $(\mathbf{A} + \mathbf{B})^T = \mathbf{A}^T + \mathbf{B}^T$
	\item $(k \cdot \mathbf{A})^T = k \cdot \mathbf{A}^T$
	\item $(\mathbf{A} \cdot \mathbf{B})^T = \mathbf{B}^T \cdot \mathbf{A}^T$ $\leftarrow$ observera ordningen.
\end{itemize}
% subsection r_kneregler (end)
\begin{Def}
    En $\mathbf{A}_{n \times n}$ matris är symetrisk om $\mathbf{A} = \mathbf{A}^T$
\end{Def}
\begin{Ex}
    Om $\vec{u} = \begin{bmatrix} u_1\\u_2 \end{bmatrix}$ så är $\vec{u} \cdot \vec{u}^T$ symertrisk.
\end{Ex}
% section transponat (end)

\section{Inverterbara matriser} % (fold)
\label{sec:inverterbara_matriser}
\begin{Def}
    En $(n \times n)$ matris vars diagonalelement är 1 och alla andra element är 0 kallas identitetsmatris, betecknas $\mathbb{I}_n$
    \[
        \mathbb{I}_n = \begin{bmatrix} 1&0&\cdots&0&0\\ 
        0&1&\cdots&0&0 \\ 
        \vdots&\vdots&\ddots&\vdots&\vdots \\ 
        0&0&\cdots&1&0 \\ 
        0&0&\cdots&0&1 \end{bmatrix}
    \]
\end{Def}
\begin{Ex}
    \[
        \mathbb{I}_2 = \begin{bmatrix} 1&0\\0&1 \end{bmatrix}
    \]
\end{Ex}

\begin{Def}
    En $(n \times n)$ matris \textbf{A} är inverterbar om det finns en matris $\mathbf{C}-{n \times m}$ så att:
    \begin{align*}
    &\mathbf{C} \cdot \mathbf{A} = \mathbb{I}_n \\ &\mathbf{A} \cdot \mathbf{C} = \mathbb{I}_n
    \end{align*}
    \textbf{C} kallas då för inversen till \textbf{A}. \textbf{B}etecknas $\mathbf{A}^{-1}$
\end{Def}

\begin{Ex}
    \begin{align*}
    &\mathbf{A} = \begin{bmatrix} 2&5\\-3&-7 \end{bmatrix} &\mathbf{C} = \begin{bmatrix} -7&-5\\3&2 \end{bmatrix}
    \end{align*}
    Då är:
    \begin{align*}
    &\mathbf{C} \cdot \mathbf{A} = \begin{bmatrix} -7&-5\\3&2 \end{bmatrix} \begin{bmatrix} 2&5\\-3&-7 \end{bmatrix} = \begin{bmatrix} 1&0\\0&1 \end{bmatrix} = \mathbb{I}_2\\
    &\mathbf{A} \cdot \mathbf{C} = \begin{bmatrix} 2&5\\-3&-7 \end{bmatrix} \begin{bmatrix} -7&-5\\3&2 \end{bmatrix} = \begin{bmatrix} 1&0\\0&1 \end{bmatrix} = \mathbb{I}_2
    \end{align*}
    dvs $\mathbf{C} = \mathbf{A}^{-1}$
\end{Ex}
\newpage
\begin{sats}
    Låt $\mathbf{A}_{2\times 2} = \begin{bmatrix} a_{11}&a_{21}\\a_{21}&a_{22} \end{bmatrix}$. Om $d = a_{11} \cdot a_{22} - a_{21} \cdot a_{12} \neq 0$. Så är \textbf{A} inverterbar, och inversen ges av:
    \[
         \mathbf{A}^{-1} = \frac{1}{d}\begin{bmatrix} a_{22}&-a_{12}\\-a_{21}&a_{11} \end{bmatrix}
     \] 
\end{sats}
\begin{Ex}
    \begin{align*}
    &\mathbf{A} = \begin{bmatrix} 2&5\\-3&-7 \end{bmatrix} &d = 2(-7) - (-3)5 = 1 \neq 0
    \end{align*}
    \[
        \mathbf{A}^{-1} = \frac{1}{1} \cdot \begin{bmatrix} -7&5\\3&2 \end{bmatrix} = \begin{bmatrix} -7&5\\3&2 \end{bmatrix}
    \]
    
\end{Ex}
\textit{d} kallas för determinanten till \textbf{A}.
% section inverterbara_matriser (end)
\end{document}