\documentclass{article}

\usepackage[utf8]{inputenc}
\usepackage{amsthm}
\usepackage{amssymb}
\usepackage{mathtools}
\usepackage{graphicx}
\usepackage{mdframed}
\usepackage{float}
\usepackage[top=0.75in, bottom=0.75in, left=0.75in, right=0.75in]{geometry}
\usepackage{gauss}

\usepackage{array}
\allowdisplaybreaks

\makeatletter
\newcounter{elimination@steps}
\newcolumntype{R}[1]{>{\raggedleft\arraybackslash$}p{#1}<{$}}
\def\elimination@num@rights{}
\def\elimination@num@variables{}
\def\elimination@col@width{}
\newenvironment{elimination}[4][0]
{
    \setcounter{elimination@steps}{0}
    \def\elimination@num@rights{#1}
    \def\elimination@num@variables{#2}
    \def\elimination@col@width{#3}
    \renewcommand{\arraystretch}{#4}
    \start@align\@ne\st@rredtrue\m@ne
}
{
    \endalign
    \ignorespacesafterend
}
\newcommand{\step}[2]
{
    \ifnum\value{elimination@steps}>0\sim\quad\fi
    \left[
        \ifnum\elimination@num@rights>0
            \begin{array}
            {@{}*{\elimination@num@variables}{R{\elimination@col@width}}
            |@{}*{\elimination@num@rights}{R{\elimination@col@width}}}
        \else
            \begin{array}
            {@{}*{\elimination@num@variables}{R{\elimination@col@width}}}
        \fi
            #1
        \end{array}
    \right]
    & 
    \begin{array}{l}
        #2
    \end{array}
    \addtocounter{elimination@steps}{1}
}
\makeatother

\DeclarePairedDelimiter{\abs}{\lvert}{\rvert}
\DeclarePairedDelimiter{\norm}{\lvert \lvert}{\rvert \rvert}

\newtheoremstyle{break}% name
  {}%         Space above, empty = `usual value'
  {}%         Space below
  {\itshape}% Body font
  {}%         Indent amount (empty = no indent, \parindent = para indent)
  {\bfseries}% Thm head font
  {.}%        Punctuation after thm head
  {\newline}% Space after thm head: \newline = linebreak
  {}%         Thm head spec

\newtheorem{Def}{Definition}[section]

\theoremstyle{break}

\newtheorem{innerEx}{Exempel}[section]
\newtheorem{sats}{Sats}[section]
\newtheorem{Rem}{Anmärkning}[]

\newenvironment{Ex}
{\begin{mdframed} \begin{innerEx} \vspace{3pt}}
{\vspace{3pt} \end{innerEx} \end{mdframed}}  

\newenvironment{bevis}
{\begin{mdframed} \begin{proof} \vspace{3pt}}
{\vspace{3pt} \end{proof} \end{mdframed}}

\title{
	 Linjär Algebra\\
	 Föreläsning 5
    \author{Erik Sjöström}
}
\begin{document}
\maketitle

\section{Matrisvektorprodukt} % (fold)
\label{sec:matrisvektorprodukt}
\begin{Def}
    Låt $A = \begin{bmatrix} a_1,a_2, \dots, a_n \end{bmatrix}$ och $\vec{v} = \begin{bmatrix} v_1\\v_2\\ \vdots \\v_n \end{bmatrix}$ Då defineras matrisvektorprodukten som:
    \[
        A \cdot \vec{v} = v_1 \cdot a_1 + v_2 \cdot a_2 + \dots + v_n \cdot a_n
    \]
\end{Def}
\begin{Ex}
    \begin{align*}
    &A = \begin{bmatrix} 1&2&-1\\0&-5&3 \end{bmatrix} &&\vec{v} = \begin{bmatrix} 4\\3\\7 \end{bmatrix}
    \end{align*}

    \[
        A \cdot \vec{v} = \begin{bmatrix} 1&2&-1\\0&-5&3 \end{bmatrix} \begin{bmatrix} 4\\3\\7 \end{bmatrix} = 4 \begin{bmatrix} 1\\0 \end{bmatrix} + 3 \begin{bmatrix} 2\\-5 \end{bmatrix} + 7 \begin{bmatrix} -1\\3 \end{bmatrix} = \begin{bmatrix} 4\\0 \end{bmatrix} + \begin{bmatrix} 6\\-15 \end{bmatrix} + \begin{bmatrix} -7\\21 \end{bmatrix} = \begin{bmatrix} 3\\6 \end{bmatrix}
    \]
\end{Ex}
\begin{Rem}
    Man får samma svar om man beräknar skalärprodukten mellan raderna i A och kolmunen i $\vec{v}$
\end{Rem}
Räkneregler för matrisvektorprodukten:\\
A och B är $(m \times n)$ matriser och $\vec{v}$, $\vec{u}$ $(nx1)$ kolumnvektorer, $k \in \mathbb{R}$
\begin{itemize}
	\item $A(\vec{v} + \vec{u}) = A \cdot \vec{u} + A \cdot \vec{v}$
	\item $(A + B) \cdot \vec{v} = A \cdot \vec{v} + B \cdot \vec{v}$
	\item $A(k \cdot \vec{v}) = k \cdot (A \cdot \vec{v}) = (k \cdot A) \cdot \vec{v}$
\end{itemize}
% section matrisvektorprodukt (end)

\section{Matrismultiplikation} % (fold)
\label{sec:matrismultiplikation}
\begin{Def}
    Låt en matris A vara av storlek $(m \times n)$, och en matris B vara av storlek $(n \times p)$ med kolumnerna $b_1,b_2, \dots ,b_p$ då är matrismultiplikationen $A \cdot B$ den $(m \times p$ matris vars kolumner är, $A \cdot b_1, A \cdot b_2, \dots, A \cdot b_p$ dvs:
    \[
        A \cdot B = A \cdot \begin{bmatrix} b_1,b_2, \dots, b_p \end{bmatrix} = \begin{bmatrix} Ab_1,Ab_2, \dots, Ab_p \end{bmatrix}
    \]
\end{Def}
\begin{Ex}
   \begin{align*}
   &A = \begin{bmatrix} 2&3\\1&-5 \end{bmatrix}_{2 \times 2} &B = \begin{bmatrix} 4&3&6\\1&-2&3 \end{bmatrix}_{2 \times 3}
   \end{align*}
   \[
       A \cdot B = \begin{bmatrix} 2&3\\1&-5 \end{bmatrix} \cdot \begin{bmatrix} 4&3&6\\1&-2&3 \end{bmatrix} = \begin{bmatrix} \begin{bmatrix} 2&3\\1&-5 \end{bmatrix}\begin{bmatrix} 4\\1 \end{bmatrix} &\begin{bmatrix} 2&3\\1&-5 \end{bmatrix}\begin{bmatrix} 3\\-2 \end{bmatrix} &\begin{bmatrix} 2&3\\1&-5 \end{bmatrix} \begin{bmatrix} 6\\3 \end{bmatrix} \end{bmatrix}
   \]
   Problemt blir nu att räkna ut tre matrisvektorprodukter:
   \[
       \begin{bmatrix} 2 \cdot 4 + 3 \cdot 1 & 2 \cdot 3 + 3 \cdot (-2) & 2 \cdot 6	+ 3 \cdot 3 \\ 1 \cdot 4 - 5 \cdot 1 & 1 \cdot 3 + (-5)(-2) & 1 \cdot 6 - 5 \cdot 3 \end{bmatrix} = \begin{bmatrix} 11&0&21\\-1&13&-9 \end{bmatrix}
   \]
\end{Ex}
Det är oftast enklare att beräkna skalärprodukten mellan raderna i A och kolumnerna i B.
\begin{Rem}
    Storlekarna (typerna) måste stämma överens för att matrismultiplikation skall var definierad. Dvs om A är en $(m \times n)$ matris, och B en $(i \times j)$ matris, så är matrismultiplikationen:
    \[
        A_{m \times n} \cdot B_{i \times j}
    \]
    definierad omm $n = i$. Storleken på den resulterande matrisen blir $m \times j$
\end{Rem}
\subsection{Räkneregler} % (fold)
\label{sub:r_kneregler}
Låt A,B,C vara matriser, och $\vec{v}$ en kolumnvektor definierade så att följande operationer är giltiga. Då gäller:
\begin{itemize}
	\item $A \cdot (k \cdot B) = k \cdot (A \cdot B) = (k \cdot A) \cdot B$ \mbox{ för } $k \in \mathbb{R}$
	\item $A \cdot (B + C) = A \cdot B + A \cdot C$
	\item $(B + C) \cdot A = B \cdot A + C \cdot A$
	\item $A \cdot (B \cdot \vec{v}) = (A \cdot B) \cdot \vec{v}$
	\item $A \cdot (B \cdot C) = (A \cdot B) \cdot C$
\end{itemize}
\begin{Rem}
\textbf{   OBS!}
    \begin{itemize}
    	\item Det gäller oftast att $A \cdot B \neq B \cdot A$
    	\item Om $A \cdot B = A \cdot C$ kan man inte dra slutsatsen att $B = C$
    	\item Om $A \cdot B = \mathbb{O}$, kan man ej dra slutsatsen att $A = \mathbb{O}$ eller $B = \mathbb{O}$
    \end{itemize}
\end{Rem}
\newpage
\begin{Ex}
    \begin{align*}
    &A = \begin{bmatrix} -1&1\\1&-1 \end{bmatrix} &B = \begin{bmatrix} 2&2\\2&2 \end{bmatrix}
    \end{align*}
    \[
        A \cdot B = \begin{bmatrix} -1&1\\1&-1 \end{bmatrix}\begin{bmatrix} 2&2\\2&2 \end{bmatrix} = \begin{bmatrix} 0&0\\0&0 \end{bmatrix} = \mathbb{O}
    \]
\end{Ex}
\begin{Ex}
    \begin{align*}
    &A = \begin{bmatrix} 5&1\\3&-1 \end{bmatrix} &B = \begin{bmatrix} 2&0\\4&3 \end{bmatrix}
    \end{align*}
    Då är:
    \begin{gather*}
    	A \cdot B = \begin{bmatrix} 5&1\\3&-1 \end{bmatrix} \begin{bmatrix} 2&0\\4&3 \end{bmatrix} = \begin{bmatrix} 14&3\\2&-3 \end{bmatrix}\\
    	B \cdot A = \begin{bmatrix} 2&0\\4&3 \end{bmatrix} \begin{bmatrix} 5&1\\3&-1 \end{bmatrix} = \begin{bmatrix} 10&2\\29&1 \end{bmatrix}
    \end{gather*}
\end{Ex}
% subsection r_kneregler (end)
% section matrismultiplikation (end)

\section{Transponat} % (fold)
\label{sec:transponat}
\begin{Def}
    Transponatet av en matris A ges av $A^T$ vars kolumner är raderna i A.
\end{Def}
\begin{Ex}
    \[
        A = \begin{bmatrix} -5&2\\1&-3\\0&4 \end{bmatrix} \mbox{ så är } A^T = \begin{bmatrix} -5&1&0\\2&-3&4 \end{bmatrix}
    \]
\end{Ex}
\begin{Ex}
    Låt $\vec{u} = \begin{bmatrix} u_1\\u_2 \end{bmatrix}$ vara en kolumnvektor. Då är:
    \begin{align*}
    	&\mbox{ Den inre produkten: } \vec{u}^T \cdot \vec{u} = \begin{bmatrix} u_1&u_2 \end{bmatrix} \cdot \begin{bmatrix} u_1\\u_2 \end{bmatrix} = u^2_1 + u^2_2\\
    	&\mbox{Den yttre produkten: } \vec{u} \cdot \vec{u}^T = \begin{bmatrix} u_1\\u_2 \end{bmatrix} \cdot \begin{bmatrix} u_1&u_2 \end{bmatrix} = \begin{bmatrix} u^2_1&u_1u_2\\u_2u_1&u^2_2 \end{bmatrix}
    \end{align*}
\end{Ex}
\subsection{Räkneregler} % (fold)
\label{sub:r_kneregler2}
Låt A,B vara matriser så att följande operationer är giltiga, $k \in \mathbb{R}$.
\begin{itemize}
	\item $(A^T)^T = A$
	\item $(A + B)^T = A^T + B^T$
	\item $(k \cdot A)^T = k \cdot A^T$
	\item $(A \cdot B)^T = B^T \cdot A^T$ $\leftarrow$ observera ordningen.
\end{itemize}
% subsection r_kneregler (end)
\begin{Def}
    En $A_{n \times n}$ matris är symetrisk om $A = A^T$
\end{Def}
\begin{Ex}
    Om $\vec{u} = \begin{bmatrix} u_1\\u_2 \end{bmatrix}$ så är $\vec{u} \cdot \vec{u}^T$ symertrisk.
\end{Ex}
% section transponat (end)

\section{Inverterbara matriser} % (fold)
\label{sec:inverterbara_matriser}
\begin{Def}
    En $(n \times n)$ matris vars diagonalelement är 1 och alla andra element är 0 kallas identitetsmatris, betecknas $\mathbb{I}_n$
    \[
        \mathbb{I}_n = \begin{bmatrix} 1&0&\cdots&0&0\\ 
        0&1&\cdots&0&0 \\ 
        \vdots&\vdots&\ddots&\vdots&\vdots \\ 
        0&0&\cdots&1&0 \\ 
        0&0&\cdots&0&1 \end{bmatrix}
    \]
\end{Def}
\begin{Ex}
    \[
        \mathbb{I}_2 = \begin{bmatrix} 1&0\\0&1 \end{bmatrix}
    \]
\end{Ex}

\begin{Def}
    En $(n \times n)$ matris A är inverterbar om det finns en matris $C-{n \times m}$ så att:
    \begin{align*}
    &C \cdot A = \mathbb{I}_n \\ &A \cdot C = \mathbb{I}_n
    \end{align*}
    C kallas då för inversen till A. Betecknas $A^{-1}$
\end{Def}
\newpage
\begin{Ex}
    \begin{align*}
    &A = \begin{bmatrix} 2&5\\-3&-7 \end{bmatrix} &C = \begin{bmatrix} -7&-5\\3&2 \end{bmatrix}
    \end{align*}
    Då är:
    \begin{align*}
    &C \cdot A = \begin{bmatrix} -7&-5\\3&2 \end{bmatrix} \begin{bmatrix} 2&5\\-3&-7 \end{bmatrix} = \begin{bmatrix} 1&0\\0&1 \end{bmatrix} = \mathbb{I}_2\\
    &A \cdot C = \begin{bmatrix} 2&5\\-3&-7 \end{bmatrix} \begin{bmatrix} -7&-5\\3&2 \end{bmatrix} = \begin{bmatrix} 1&0\\0&1 \end{bmatrix} = \mathbb{I}_2
    \end{align*}
    dvs $C = A^{-1}$
\end{Ex}

\begin{sats}
    Låt $A_{2\times 2} = \begin{bmatrix} a_{11}&a_{21}\\a_{21}&a_{22} \end{bmatrix}$. Om $d = a_{11} \cdot a_{22} - a_{21} \cdot a_{12} \neq 0$. Så är A inverterbar, och inversen ges av:
    \[
         A^{-1} = \frac{1}{d}\begin{bmatrix} a_{22}&-a_{12}\\-a_{21}&a_{11} \end{bmatrix}
     \] 
\end{sats}
\begin{Ex}
    \begin{align*}
    &A = \begin{bmatrix} 2&5\\-3&-7 \end{bmatrix} &d = 2(-7) - (-3)5 = 1 \neq 0
    \end{align*}
    \[
        A^{-1} = \frac{1}{1} \cdot \begin{bmatrix} -7&5\\3&2 \end{bmatrix} = \begin{bmatrix} -7&5\\3&2 \end{bmatrix}
    \]
    
\end{Ex}
d kallas för determinanten till A.
% section inverterbara_matriser (end)
\end{document}