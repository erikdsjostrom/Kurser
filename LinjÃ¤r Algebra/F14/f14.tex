\documentclass{article}

\usepackage[utf8]{inputenc}
\usepackage{amsthm}
\usepackage{amssymb}
\usepackage{mathtools}
\usepackage{graphicx}
\usepackage{mdframed}
\usepackage{float}
\usepackage[top=0.75in, bottom=0.75in, left=0.75in, right=0.75in]{geometry}
\usepackage{gauss}

\usepackage{array}
\allowdisplaybreaks

\makeatletter
\newcounter{elimination@steps}
\newcolumntype{R}[1]{>{\raggedleft\arraybackslash$}p{#1}<{$}}
\def\elimination@num@rights{}
\def\elimination@num@variables{}
\def\elimination@col@width{}
\newenvironment{elimination}[4][0]
{
    \setcounter{elimination@steps}{0}
    \def\elimination@num@rights{#1}
    \def\elimination@num@variables{#2}
    \def\elimination@col@width{#3}
    \renewcommand{\arraystretch}{#4}
    \start@align\@ne\st@rredtrue\m@ne
}
{
    \endalign
    \ignorespacesafterend
}
\newcommand{\step}[2]
{
    \ifnum\value{elimination@steps}>0\sim\quad\fi
    \left[
        \ifnum\elimination@num@rights>0
            \begin{array}
            {@{}*{\elimination@num@variables}{R{\elimination@col@width}}
            |@{}*{\elimination@num@rights}{R{\elimination@col@width}}}
        \else
            \begin{array}
            {@{}*{\elimination@num@variables}{R{\elimination@col@width}}}
        \fi
            #1
        \end{array}
    \right]
    & 
    \begin{array}{l}
        #2
    \end{array}
    \addtocounter{elimination@steps}{1}
}
\makeatother

\DeclarePairedDelimiter{\abs}{\lvert}{\rvert}
\DeclarePairedDelimiter{\norm}{\lvert \lvert}{\rvert \rvert}

\newtheoremstyle{break}% name
  {}%         Space above, empty = `usual value'
  {}%         Space below
  {\itshape}% Body font
  {}%         Indent amount (empty = no indent, \parindent = para indent)
  {\bfseries}% Thm head font
  {.}%        Punctuation after thm head
  {\newline}% Space after thm head: \newline = linebreak
  {}%         Thm head spec

\newtheorem{Def}{Definition}[section]

\theoremstyle{break}

\newtheorem{innerEx}{Exempel}[section]
\newtheorem{sats}{Sats}[section]
\newtheorem{Rem}{Anmärkning}[]

\newenvironment{Ex}
{\begin{mdframed} \begin{innerEx} \vspace{3pt}}
{\vspace{3pt} \end{innerEx} \end{mdframed}}  

\newenvironment{bevis}
{\begin{mdframed} \begin{proof} \vspace{3pt}}
{\vspace{3pt} \end{proof} \end{mdframed}}

\usepackage{gauss}

\title{
	 Linjär Algebra\\
	 Föreläsning 14 - Baser och linjärt beroende
    \author{Erik Sjöström}
}
\begin{document}
\maketitle
\section{Linjärt beroende} % (fold)
\label{sec:linj_rt_beroende}
\paragraph{Obs:} % (fold)
\label{par:obs_}
\begin{align*}
&&\begin{cases}
	x_1 + x_2 + 2x_3 = 0\\
	x_1 + 2x_2 + x_3 = 0\\
	x_1 + x_2 + x_3 = 0
\end{cases}
& \Leftrightarrow
&& \begin{bmatrix} 1\\1\\1 \end{bmatrix} \cdot x_1 + 
\begin{bmatrix} 1\\2\\1 \end{bmatrix} \cdot x_2 + 
\begin{bmatrix} 2\\1\\1 \end{bmatrix} \cdot x_3 = 
\begin{bmatrix} 0\\0\\0 \end{bmatrix}\\
&&
& \Leftrightarrow 
&&\begin{bmatrix} 1&1&2\\1&2&1\\1&1&1 \end{bmatrix} \cdot \begin{bmatrix} x_1\\x_2\\x_3 \end{bmatrix} = \begin{bmatrix} 0\\0\\0 \end{bmatrix}\\
&&
& \Leftrightarrow
&& \begin{bmatrix}
	\begin{array}{ccc|c}
	1 & 1 & 2 & 0\\
	1 & 2 & 1 & 0\\
	1 & 1 & 1 & 0\\
\end{array} 
\end{bmatrix}
\end{align*}
% paragraph obs_ (end)
\begin{Def}
	Vektorerna $\vec{v}_1, \vec{v}_2,...,\vec{v}_n$ är \underline{linjärt oberoende} om enda lösningen till ekvationen:
	\begin{align*}
	&&x_1 \cdot \vec{v}_1 + x_2 \cdot \vec{v}_2 + ... + x_n \cdot \vec{v}_n	= \emptyset
	&&\mbox{är}
	&&x_1 = x_2 = ... = x_n = 0
	&&x_i \in \mathbb{R}
	\end{align*}
	
\end{Def}
\begin{Def}
	Vektorerna $\vec{v}_1, \vec{v}_2,...,\vec{v}_n$ är \underline{linjärt beroende} om enda lösningen till ekvationen:
	\[
	x_1 \cdot \vec{v}_1 + x_2 \cdot \vec{v}_2 + ... + x_n \cdot \vec{v}_n	= \emptyset
	\]
	har en lösning där åtminstonde något $x_i \neq 0$
\end{Def}
\begin{Ex}
	\begin{align*}
	&\vec{e}_1 = \begin{bmatrix} 1\\0\\0 \end{bmatrix}
	&&\vec{e}_2 = \begin{bmatrix} 0\\1\\0 \end{bmatrix}
	&&\vec{e}_3	= \begin{bmatrix} 0\\0\\1 \end{bmatrix}
	\end{align*}
	är \underline{linjärt oberoende} ty:
	\[
	x_1 \cdot \vec{e}_1 + x_2 \cdot \vec{e}_2 + x_3 \cdot \vec{e}_3 = x_1 \begin{bmatrix} 1\\0\\0 \end{bmatrix} + x_2 \begin{bmatrix} 0\\1\\0 \end{bmatrix} + x_3 \begin{bmatrix} 0\\0\\1 \end{bmatrix} \Leftrightarrow \begin{bmatrix} 1&0&0\\0&1&0\\0&0&1 \end{bmatrix} = \begin{bmatrix} 0\\0\\0 \end{bmatrix} \Rightarrow x_1=x_2=x_3=0
	\]
\end{Ex}
\begin{Ex}
	\begin{align*}
	&\vec{v}_1 = \begin{bmatrix} 1\\1\\1 \end{bmatrix}
	&&\vec{v}_2	= \begin{bmatrix} 1\\2\\1 \end{bmatrix}
	&&\vec{v}_3 = \begin{bmatrix} 2\\1\\1 \end{bmatrix}
	\end{align*}
	är \underline{linjärt oberoende} ty:
	\[
	x_1 \cdot \vec{v}_1 + x_2 \cdot \vec{v}_2 + x_3 \cdot \vec{v}_3 = \emptyset \Leftrightarrow \begin{bmatrix} 1 & 1 & 2\\1 & 2 & 1\\1&1&1 \end{bmatrix}\begin{bmatrix} x_1\\x_2\\x_3 \end{bmatrix} = \begin{bmatrix} 0\\0\\0 \end{bmatrix}
	\]
	Gausseliminering ger:
	\[
	\begin{bmatrix}
		\begin{array}{ccc|c}
			1 & 1 & 2 & 0\\
			1 & 2 & 1 & 0\\
			1 & 1 & 1 & 0
		\end{array}
	\end{bmatrix} \sim ... \sim
	\begin{bmatrix}
		\begin{array}{ccc|c}
			1 & 0 & 0 & 0\\
			0 & 1 & 0 & 0\\
			0 & 0 & 1 & 0
		\end{array}
	\end{bmatrix} \Rightarrow x_1 = x_2 = x_3 = 0
	\]
\end{Ex}
\begin{Ex}
	\begin{align*}
	&\vec{v}_1 = \begin{bmatrix} 1\\1\\1 \end{bmatrix}
	&&\vec{v}_2 = \begin{bmatrix} 1\\2\\1 \end{bmatrix}
	&&\vec{v}_3 = \begin{bmatrix} 1\\3\\1 \end{bmatrix}
	\end{align*}
	Är \underline{linjärt beroende} ty:
	\[
	x_1 \cdot \vec{v}_1 0 x_2 \cdot \vec{v}_2 + x_3 + \vec{v}_3 =
	\begin{bmatrix}
		1 & 1 & 1\\
		1 & 2 & 3\\
		1 & 1 & 1\\
	\end{bmatrix}
	\begin{bmatrix} x_1\\x_2\\x_3 \end{bmatrix} = \begin{bmatrix} 0\\0\\0 \end{bmatrix}
	\]
	Gausseliminering ger:
	\[
	\begin{bmatrix}
		\begin{array}{ccc|c}
			1 & 1 & 1 & 0\\
			1 & 2 & 3 & 0\\
			1 & 1 & 1 & 0
		\end{array}
	\end{bmatrix} \sim ... \sim
	\begin{bmatrix}
		\begin{array}{ccc|c}
			1 & 1 & 1 & 0\\
			0 & 1 & 2 & 0\\
			0 & 0 & 0 & 0\\
		\end{array}
	\end{bmatrix}
	\]
	Vi har en fri kolumn, och en fri variabel $x_3$, vilket ger lösningarna:
	\[
	\begin{cases}
		x_1 = -(-2t) - t = t\\
		x_2 = -2t\\
		x_3 = t
	\end{cases}
	\]
	Vi får:
	\[
	t \cdot \vec{v}_1 - 2t \cdot \vec{v}_2 + t \cdot \vec{v}_3 = \emptyset
	\]
	\textit{t} är ett godtyckligt reellt tal. Om vi sätter $t = 1$ får vi:
	\[
	\begin{cases}
		x_1 = 1, \\ 
		x_2 = -2, \\
		x_3 = 1
	\end{cases}
	\]
	Vi har då:
	\[
	\vec{v}_1 = 2 \vec{v}_2 - \vec{v}_3
	\]
\end{Ex}
\newpage
\begin{itemize}
	\item Två vektorer $\vec{v}_1$ och $\vec{v}_2$ är \underline{linjärt beroende} om de är parallella, ty $\vec{v}_1 = t \cdot \vec{v}_2$, $t \in \mathbb{R}$
	\item \textit{n} stycken vektorer är \underline{linjärt beroende} om minst en av dom kan uttryckas som en linjärkombination av dom andra vektorerna.\\
	(dvs: denna vektor är "överflödig" i den mening att man kan uttrycka lika många linjärkombinationer om man tar bort denna vektor)
\end{itemize}
\begin{mdframed}
\paragraph{Exempel: 1.3.} % (fold)
\label{par:ex_1_3_}
\textbf{(igen)}\\
Ta bort $\vec{v}_3$, vi kan fortfarande uttrycka lika många linjärkombinationer med $\vec{v}_1$ och $\vec{v}_2$
% paragraph ex_1_3_ (end)
\end{mdframed}
% section linj_rt_beroende (end)

\section{Baser} % (fold)
\label{sec:baser}
\begin{Def}
	Vektorerna $\vec{v}_1, \vec{v}_2,..., \vec{v}_n$ utgör en bas för en mängd $\mathbb{V}$ om de är \underline{linjärt oberoende} och om varje vektor $\vec{v} \in \mathbb{V}$ kan skrivas som en linjärkombination av vektorerna $\vec{v}_1, \vec{v}_2,..., \vec{v}_n$.
\end{Def}
\paragraph{-} % (fold)
\label{par:dimension_}
\textbf{Dimensionen} för $\mathbb{V}$ är antalet vektorer i basen.
% paragraph dimension_ (end)
\begin{Ex}
	\begin{align*}
	&\vec{e}_1 = \begin{bmatrix} 1\\0 \end{bmatrix}
	&&\vec{e}_2 = \begin{bmatrix} 0\\1 \end{bmatrix}
	\end{align*}
	är en bas i $\mathbb{R}^2$ eftersom de är \underline{linjärt oberoende} och alla vektorer i $\mathbb{R}^2$ kan skrivas som en linjärkombination av $\vec{e}_1$ och $\vec{e}_2$
\end{Ex}
\begin{Ex}
	Låt $\vec{v} = \begin{bmatrix} 2\\1 \end{bmatrix}$. Då är $\vec{e}_1 = \begin{bmatrix} 1\\0 \end{bmatrix}$ och $\vec{v}$ ej parallella.
	\begin{center}
		INFOGA FIGUR HÄR
	\end{center}
	så vi vet att de är \underline{linjärt oberoende}
	Alla andra vektorer i $\mathbb{R}^2$ kan skrivas so en linjärkombination ($\vec{e}_1$, $\vec{v}$), t.ex.:
	\[
	\vec{e}_2 = \begin{bmatrix} 0\\1 \end{bmatrix} = -2 \vec{e}_1 + \vec{v} = -2 \begin{bmatrix} 1\\0 \end{bmatrix} + \begin{bmatrix} 2\\1 \end{bmatrix} = \begin{bmatrix} 0\\1 \end{bmatrix}
	\]
	Eller som en gausselimination:
	\[
	\begin{bmatrix}
		\begin{array}{cc|c}
			1 & 2 & 0\\
			0 & 1 & 0
		\end{array}
	\end{bmatrix}
	\Rightarrow
	\begin{cases}
		x_1 = 1\\
		x_2 = -2
	\end{cases}
	\]
\end{Ex}
\begin{mdframed}
	\paragraph{Exempel 1.2.} % (fold)
	\label{par:ex_}
	\textbf{(igen)}\\
	\begin{align*}
	&\vec{e}_1 = \begin{bmatrix} 1\\1\\1 \end{bmatrix}
	&&\vec{e}_2 = \begin{bmatrix} 1\\2\\1 \end{bmatrix}
	&&\vec{e}_3	= \begin{bmatrix} 2\\1\\1 \end{bmatrix}
	\end{align*}
	är tre stycken \underline{linjärt oberoende} vektorer i $\mathbb{R}^3$, och utgör därmed en bas för $\mathbb{R}^3$.
	% paragraph ex_ (end)
\end{mdframed}
\begin{mdframed}
	\paragraph{Exempel 1.3.} % (fold)
	\label{par:exempel_1_3_}
	\textbf{(igen)}\\
	\begin{align*}
	& \vec{v}_1 = \begin{bmatrix} 1\\1\\1 \end{bmatrix}
	&& \vec{v}_2 = \begin{bmatrix} 1\\2\\1 \end{bmatrix}
	&& \vec{v}_3 = \begin{bmatrix} 1\\3\\1 \end{bmatrix}
	\end{align*}
	är inte en bas för $\mathbb{R}^3$ eftersom de är \underline{linjärt beroende}.
	% paragraph exempel_1_3_ (end)
\end{mdframed}
Vi måste alltså ha tre stycken \underline{linjärt oberoende} vektorer i $\mathbb{R}^3$ för att få en bas för $\mathbb{R}^3$, det finns alltså en vektor $\vec{b} \in \mathbb{R}^3$ som ej kan skrivas som en linjärkombination av $\vec{v}_1, \vec{v}_2, \vec{v}_3$, t.ex. $\vec{b} = \begin{bmatrix} 0\\0\\1 \end{bmatrix}$\\
I exempel \textbf{1.3.} fick vi en gausseliminering som gav:
\[
... \sim
\begin{bmatrix}
	\begin{array}{ccc|c}
		1 & 1 & 1 & 0\\
		0 & 1 & 2 & 0\\
		0 & 0 & 0 & 0
	\end{array}
\end{bmatrix}
\]
Om det hade stått $\begin{bmatrix} 0\\0\\1 \end{bmatrix}$ i högerledet så hade vi fått ett system som saknar lösningar, ty vi hade då fått en ekvation:
\[
0x_1 + 0x_2 + 0x_3 = 1
\]
Vilket saknar lösning. Alltså har vi funnit en vektor som ej är en linjärkombinationen av $\vec{v}_1, \vec{v}_2, \vec{v}_3$\\
\paragraph{-} % (fold)
\label{par:_1}
Om $\vec{v}_1, \vec{v}_2,..., \vec{v}_n$ är \underline{linjärt beroende} så har ekvationen:
\[
\vec{v}_1 \cdot x_1 + \vec{v}_2 \cdot x_2 + ... + \vec{v}_n \cdot x_n = \vec{b}
\]
\begin{itemize}
	\item oändligt många lösningar om $\vec{b}$ kan skrivas som en linjärkombination av $\vec{v}_1, \vec{v}_2,..., \vec{v}_n$.
	\item inga lösningar om $\vec{b}$ ej kan skrivas som en linjärkombination av $\vec{v}_1, \vec{v}_2,..., \vec{v}_n$.
\end{itemize}
% paragraph _1 (end)
Eller uttryckt i matrisform:\\
Om $\mathbf{A} = \begin{bmatrix} \vec{v}_1&\vec{v}_2&\vec{v}_3 \end{bmatrix}$ har linjärt beroende kolumner så har:
\[
\mathbf{A} \cdot \vec{x} = \vec{b}
\]
\begin{itemize}
	\item oändligt många lösningar om $\vec{b}$ kan skrivas som en linjärkombination av kolumnerna.
	\item inga lösningar om $\vec{b}$ ej kan skrivas som en linjärkombination av kolumnerna.
\end{itemize}
Om vektorerna $\vec{v}_1, \vec{v}_2,..., \vec{v}_n \in \mathbb{R}^n$ är \underline{linjärt oberoende} så har:
\[
\vec{v}_1 \cdot x_1 + \vec{v}_2 \cdot x_2 +...+ \vec{v}_n \cdot x_n = \emptyset
\]
Bara lösningen $x_1 = x_2 = ... = x_n = 0$\\
Vilket betyder att:
\begin{align*}
&\vec{v}_1 \cdot x_1 + \vec{v}_2 \cdot x_2 +...+ \vec{v}_n \cdot x_n = \emptyset
&&\vec{b} \in \mathbb{R}^n
\end{align*}
har en entydig lösning.\\
Eller uttryckt i matriform:
Om $\mathbf{A} = \begin{bmatrix} \vec{v}_1 & \vec{v}_2 &...&\vec{v}_n \end{bmatrix}$ har \underline{linjärt oberoende} kolumner så har $\mathbf{A} \cdot \vec{x} = \vec{b}$ en entydigt bestämd lösning. Dvs den homogena ekvationen:
\[
\mathbf{A} \cdot \vec{x} = \emptyset
\]
Vilket har den triviala lösningen: $\vec{x} = \emptyset$
\begin{sats}
	$\vec{v}_1, \vec{v}_2,..., \vec{v}_r$ är en bas för $\mathbb{R}^n$ oom $r = n$ (och $\vec{v}_1,..., \vec{v}_r \in \mathbb{R}^n$) och dom är \underline{linjärt oberoende}
\end{sats}
\begin{bevis}
	Låt $\mathbf{A} = \begin{bmatrix} \vec{v}_1&\vec{v}_2&...&\vec{v}_r \end{bmatrix}$ och gausseliminera \textbf{A} till \textbf{T} (trappstgsform). Om:
	\begin{itemize}
		\item $r > n$, då har vi fler obekanta än ekvationer. Vilket betyder att vi får fria kolumner i lösningen, dvs vi får oändligt många lösningar, dvs kolumnerna i \textbf{A} är linjärt beroende. De utgör alltså ej en bas för $\mathbb{R}^n$.
		\item $r < n$, få har vi fler ekvationer än obekanta. Vilket betyder att minst en rad ej kommer innehålla ett pivotelement. Eftersom $\vec{b}$ kan väljas fritt, jämför exempel \textbf{1.3.}. Välj $\vec{b}$:
		\[
		\vec{b} = \begin{bmatrix} 0\\0\\\vdots\\0\\1 \end{bmatrix}
		\]
		detta är ej lösbart av samma anledning som i exempel \textbf{1.3.}. Dvs $\vec{b}$ kan ej skrivas som en linjärkombination av $\vec{v}_1, \vec{v}_2,..., \vec{v}_r$. Dvs $\vec{v}_1, \vec{v}_2,..., \vec{v}_r$ är ej en bas för $\mathbb{R}^n$
		\item $r = n$, är det enda kvarstående alternativet.
	\end{itemize}
\end{bevis}
% section baser (end)
\end{document}