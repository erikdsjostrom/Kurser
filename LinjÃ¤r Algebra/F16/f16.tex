\documentclass{article}

\usepackage[utf8]{inputenc}
\usepackage{amsthm}
\usepackage{amssymb}
\usepackage{mathtools}
\usepackage{graphicx}
\usepackage{mdframed}
\usepackage{float}
\usepackage[top=0.75in, bottom=0.75in, left=0.75in, right=0.75in]{geometry}
\usepackage{gauss}

\usepackage{array}
\allowdisplaybreaks

\makeatletter
\newcounter{elimination@steps}
\newcolumntype{R}[1]{>{\raggedleft\arraybackslash$}p{#1}<{$}}
\def\elimination@num@rights{}
\def\elimination@num@variables{}
\def\elimination@col@width{}
\newenvironment{elimination}[4][0]
{
    \setcounter{elimination@steps}{0}
    \def\elimination@num@rights{#1}
    \def\elimination@num@variables{#2}
    \def\elimination@col@width{#3}
    \renewcommand{\arraystretch}{#4}
    \start@align\@ne\st@rredtrue\m@ne
}
{
    \endalign
    \ignorespacesafterend
}
\newcommand{\step}[2]
{
    \ifnum\value{elimination@steps}>0\sim\quad\fi
    \left[
        \ifnum\elimination@num@rights>0
            \begin{array}
            {@{}*{\elimination@num@variables}{R{\elimination@col@width}}
            |@{}*{\elimination@num@rights}{R{\elimination@col@width}}}
        \else
            \begin{array}
            {@{}*{\elimination@num@variables}{R{\elimination@col@width}}}
        \fi
            #1
        \end{array}
    \right]
    & 
    \begin{array}{l}
        #2
    \end{array}
    \addtocounter{elimination@steps}{1}
}
\makeatother

\DeclarePairedDelimiter{\abs}{\lvert}{\rvert}
\DeclarePairedDelimiter{\norm}{\lvert \lvert}{\rvert \rvert}

\newtheoremstyle{break}% name
  {}%         Space above, empty = `usual value'
  {}%         Space below
  {\itshape}% Body font
  {}%         Indent amount (empty = no indent, \parindent = para indent)
  {\bfseries}% Thm head font
  {.}%        Punctuation after thm head
  {\newline}% Space after thm head: \newline = linebreak
  {}%         Thm head spec

\newtheorem{Def}{Definition}[section]

\theoremstyle{break}

\newtheorem{innerEx}{Exempel}[section]
\newtheorem{sats}{Sats}[section]
\newtheorem{Rem}{Anmärkning}[]

\newenvironment{Ex}
{\begin{mdframed} \begin{innerEx} \vspace{3pt}}
{\vspace{3pt} \end{innerEx} \end{mdframed}}  

\newenvironment{bevis}
{\begin{mdframed} \begin{proof} \vspace{3pt}}
{\vspace{3pt} \end{proof} \end{mdframed}}


\title{
	 Linjär Algebra\\
	 Föreläsning 16
    \author{Erik Sjöström}
}
\begin{document}
\maketitle

\section{Basbyten} % (fold)
\label{sec:basbyten}
\paragraph{Hur utnyttjar man basbyten?} % (fold)
\label{par:hur_utnyttjar_man_basbyten_}
Låt:
\[
\vec{x} = \begin{bmatrix} x_1\\x_2\\x_2 \end{bmatrix} \in \mathbb{R}^3
\]
Antag att vi vill rotera $\vec{x}$ runt axeln vars riktning är $\vec{v}$ (som ej är $\vec{e}_1, \vec{e}_2, \vec{e}_3$-axlarna)\\
Rotationen ska ske med vinkeln $\theta$ moturs. Relativt $\vec{v}$:s riktning.
Vi kan via basbyte återföra denna rotation till en rotation runt t.ex. $x_1$-axeln (som vi ju har en rotationsmatris för,se lab 4).\\
Vi bildar en bas runt $\vec{v}$. Låt $\mathbf{G} = (\vec{g}_1, \vec{g}_2, \vec{g}_3)$ vara en ortonormal bas för $\mathbb{R}^3$ och $\vec{g}_1 = \frac{\vec{v}}{\norm{\vec{v}}}$\\
Låt $\mathbf{G} = \begin{bmatrix} \vec{g}_1&\vec{g}_2&\vec{g}_3 \end{bmatrix}$ vara basbytesmatrisen, dvs:
\[
\mathbf{G} \cdot \vec{x}_{\mathbf{G}} = \vec{x} \Leftrightarrow \vec{x}_{\mathbf{G}} = \mathbf{G}^{-1} \cdot \vec{x}
\]
Rotera $\vec{x}_{\mathbf{G}}$ runt $\vec{g}_1$ axeln med standardmatrisen:
\begin{align*}
&\mathbf{A} = \begin{bmatrix} 1&0&0\\0&\sin(\theta)&-\sin(\theta)\\0&\sin(\theta)&\cos(\theta) \end{bmatrix}
\end{align*}
Vi får $\mathbf{A} \cdot \vec{x}_{\mathbf{G}}$ (rotationen uttryckt i basen ($\vec{g}_1, \vec{g}_2, \vec{g}_3$)). Uttryck rotationen i standardbasen:
\[
\mathbf{G} \cdot (\mathbf{A} \cdot \vec{x}_{\mathbf{G}})
\]
Vi har alltså beräknat:
\[
\overbrace{\mathbf{G} \cdot \overbrace{\mathbf{A} \cdot \overbrace{\mathbf{G}^{-1} \cdot \vec{x}}^{\vec{x}_{\mathbf{G}}}}^\text{roterad $\vec{x}_{\mathbf{G}}$ i basen \textbf{G}}}^\text{roterad $\vec{x}$ ($\vec{x}_{\mathbf{G}}$ uttryckt i ($\vec{e}_1, \vec{e}_2, \vec{e}_3$))}
\]
Låt en linjär avbildning $f: \mathbb{R}^n \rightarrow \mathbb{R}^m$ ha matrisen $\mathbf{A}_{\mathbf{G}}$ relativt basen $\mathbf{G} = (\vec{g}_1, \vec{g}_2, ..., \vec{g}_n$\\
Då har avbildningen relativt standardbasen, matrisen:
\[
\mathbf{A}_{\mathbf{E}} = \mathbf{G} \cdot \mathbf{A}_{\mathbf{G}} \cdot \mathbf{G}^{-1}
\]
% paragraph hur_utnyttjar_man_basbyten_ (end)
% section basbyten (end)
\section{Linjära avbildningar} % (fold)
\label{sec:linj_ra_avbildningar}
Till en linjär avbildning $f: \mathbb{R}^n \rightarrow \mathbb{R}^m$ hör en $(m \times n)$-matris \textbf{A} sådan att:
\begin{align*}
&f(\vec{x})_{\mathbf{E}} = \mathbf{A}_{\mathbf{E}} \cdot \vec{x}_{\mathbf{E}}
&&\mbox{och}
&&\mathbf{A} = \begin{bmatrix} f(\vec{e}_1)&f(f(\vec{e}_2))&...&f(\vec{e}_n) \end{bmatrix}
\end{align*}
(i standardbasen \textbf{E}). Vi har antagit förut att vi har opererat i standardbasen.
\paragraph{Den allmänna formulering av Bassatsen:} % (fold)
\label{par:den_allm_nna_formulering_av_bassatsen_}
Låt $f:\mathbb{R}^n \rightarrow \mathbb{R}^m$ vara en linjär avbildning och låt $\mathbf{G} = (\vec{g_1}, \vec{g}_2, ..., \vec{g}_m)$ vara en bas i $\mathbb{R}^m$ och låt $\mathbf{H} = (\vec{h}_1, \vec{h}_2, ..., \vec{h}_n)$ vara en bas för $\mathbb{R}^n$.\\
Då gäller att standardmatrisen för \textit{f} relativt baserna \textbf{G} och \textbf{H} ges av:
\[
 \mathbf{A}_{\mathbf{H} \rightarrow \mathbf{G}} = \begin{bmatrix} f(\vec{h}_1)_{\mathbf{G}} & f(\vec{h}_2)_{\mathbf{G}}&...& f(\vec{h}_n)_{\mathbf{G}} \end{bmatrix}
 \]
 där $f(\vec{h}_i)_{\mathbf{G}}$ är $f(\vec{h}_i)_{\mathbf{G}}$ i basen \textbf{G}.
% paragraph den_allm_nna_formulering_av_bassatsen_ (end)
\begin{Ex}
	I rotationsexemplet har vi $m=n=3$, $\mathbf{G}=\mathbf{H}$. Dvs:
	\[
	\mathbf{A} =
	\begin{bmatrix}
	1 & 0 & 0\\
	0 & \cos(\theta) & -\sin(\theta)\\
	0 & \sin(\theta) & \cos{t\theta}
	\end{bmatrix}
	\]
	är standardmatris: $f: \mathbf{G} \rightarrow \mathbf{G}: f(\vec{x}_{\mathbf{G}}) = \mathbf{A} \cdot \vec{x}_{\mathbf{G}}$
\end{Ex}
% section linj_ra_avbildningar (end)
\section{Egenvärden och egenvektorer} % (fold)
\label{sec:egenv_rden_och_egenvektorer}
Givet en kvadratisk matris $A_{(n \times m)}$, bestäm en vektor $\vec{v} \neq 0$ och tillhörande tal $\lambda$ så att:
\[
\mathbf{A} \cdot \vec{v} = \lambda \cdot \vec{v}
\]
$\vec{v}$ är en egenvektor till \textbf{A} om $\vec{v}$ och $\mathbf{A} \cdot \vec{v}$ är parallella.
\begin{Ex}
	Givet en matris och en vektor:
	\[
	A = \begin{bmatrix} 3&-2\\1&0 \end{bmatrix}
	\]
	Så är vektorn $\vec{v} = \begin{bmatrix} 2\\1 \end{bmatrix}$ en egenvektor och $\lambda = 2$ egenvärde ty:
	\[
	\mathbf{A} \cdot \vec{v} = \begin{bmatrix} 3&-2\\1&0 \end{bmatrix} \begin{bmatrix} 2\\1 \end{bmatrix} = \begin{bmatrix} 4\\2 \end{bmatrix} = 2 \cdot \begin{bmatrix} 2\\1 \end{bmatrix}
	\]
	Vektor $\vec{v} = \begin{bmatrix} 1\\1 \end{bmatrix}$ är en egenvektor och $\lambda = 1$ är det tillhörande egenvärdet ty:
	\[
	\mathbf{A} \cdot \vec{v} = \begin{bmatrix} 3&-2\\1&0 \end{bmatrix}\begin{bmatrix} 1\\1 \end{bmatrix} = \begin{bmatrix} 1\\1 \end{bmatrix} = 1 \cdot \begin{bmatrix} 1\\1 \end{bmatrix}
	\]
	Men $\vec{v} = \begin{bmatrix} -1\\1 \end{bmatrix}$ är inte en egenvektor, eftersom:
	\[
	\mathbf{A} \cdot \vec{v} = \begin{bmatrix} 3&-2\\1&0 \end{bmatrix}\begin{bmatrix} -1\\1 \end{bmatrix} = \begin{bmatrix} -5\\-1 \end{bmatrix}
	\]
	är ej parallell med $\vec{v}$.\\
\end{Ex}
\newpage
\noindent
Om $\vec{v}$ är en egenvektor så är den en lösning till:
\[
(\mathbf{A} - \lambda \cdot \mathbf{I}) \cdot \vec{v} = \emptyset
\]
ty:
\begin{align*}
\mathbf{A} \cdot \vec{v} = \lambda \cdot \vec{v} &\Leftrightarrow \mathbf{A} \cdot \vec{v} - \lambda \cdot \vec{v} = \emptyset\\
&\Leftrightarrow A \cdot \vec{v} - \lambda \cdot \mathbf{I} \cdot \vec{v} = \emptyset\\
&\Leftrightarrow(\mathbf{A} - \lambda \cdot \mathbf{I}) \cdot \vec{v} = \emptyset
\end{align*}

\paragraph{-} % (fold)
\label{par:_}
Matrisen $(\mathbf{A} - \lambda \cdot \mathbf{I})$ är inte inverterbar om:
\[
\overbrace{\mathbf{det}(\mathbf{A} - \lambda \cdot \mathbf{I} = 0)}^\text{karakteristiska ekvationen}
\]
\begin{Ex}
	För:
	\[
	A = \begin{bmatrix} 1&4\\1&1 \end{bmatrix}
	\]
	har vi att:
	\begin{align*}
	\mathbf{det}(\mathbf{A} - \lambda \cdot \mathbf{I}) = \mathbf{det}\begin{pmatrix} \begin{bmatrix} 1&4\\1&1 \end{bmatrix} - \lambda \begin{bmatrix} 1&0\\0&1 \end{bmatrix} \end{pmatrix} &= \mathbf{det}\begin{pmatrix} \begin{bmatrix} 1 - \lambda & 4\\1 & 1 - \lambda \end{bmatrix} \end{pmatrix} \\
	&= (1 - \lambda)(1 - \lambda) - 1 \cdot 4 = (1 - \lambda)^2 - 4 = 0
	\end{align*}
	Som har lösningen:
	\[
	\begin{cases}
		\lambda_1 = -1\\
		\lambda_2 = 3
	\end{cases}
	\]
	Egenvektorn som hör till egenvärdet $\lambda_1 = -1$ får vi genom att lösa ekvationen:
	\[
	(\mathbf{A} - \lambda \cdot \mathbf{I}) \cdot \vec{v} = (\mathbf{A} + \mathbf{I}) \cdot \vec{v} = \emptyset
	\]
	Dvs:
	\begin{align*}
		\begin{pmatrix}
			\begin{bmatrix}
				1 & 4\\
				1 & 1 
			\end{bmatrix} + 
			\begin{bmatrix} 
				1 & 1\\
				0 & 1 
			\end{bmatrix}
		\end{pmatrix} \cdot 
		\begin{bmatrix}
		v_1\\
		v_2 
		\end{bmatrix} = 
		\begin{bmatrix}
			0\\
			0 
		\end{bmatrix}
		&\Leftrightarrow
		\begin{bmatrix}
		2 & 4\\
		1 & 2 
		\end{bmatrix}
		\begin{bmatrix} 
			v_1\\
			v_2 
		\end{bmatrix} = 
		\begin{bmatrix} 
			0\\
			0 
		\end{bmatrix}\\
		&\Leftrightarrow 
		\begin{bmatrix}
			\begin{array}{cc|c}
			    2 & 4 & 0\\
			    1 & 2 & 0
			\end{array}
		\end{bmatrix}
		\sim
		\begin{bmatrix}
			\begin{array}{cc|c}
			    2 & 4 & 0\\
			    0 & 0 & 0
			\end{array}
		\end{bmatrix}
	\end{align*}
	Låt $\vec{v}_2 = t$, vi får då $\vec{v}_1 = -2t$, dvs:
	\begin{align*}
	&\vec{v} = 
	\begin{bmatrix}
	    1\\
	    -2
	\end{bmatrix} \cdot t
	&&\mbox{där $t \neq 0$ kan väljas godtyckligt}
	\end{align*}
	Egenvektorn som hör ihop med $\lambda_2 = 3$
	\begin{gather*}
		\begin{pmatrix}
			\begin{bmatrix}
				1 & 4\\
				1 & 1	
			\end{bmatrix}
			- 3 \cdot
			\begin{bmatrix}
				1 & 0\\
				0 & 1
			\end{bmatrix}
		\end{pmatrix}
		\begin{bmatrix}
		    v_1\\
		    v_2
		\end{bmatrix} = 
		\begin{bmatrix}
		    0 \\
		    0
		\end{bmatrix}
		\Leftrightarrow
		\begin{bmatrix}
		    -2 & 4\\
		    1 & -2
		\end{bmatrix}
		\begin{bmatrix}
		    v_1\\
		    v_2
		\end{bmatrix} = 
		\begin{bmatrix}
		    0\\
		    0
		\end{bmatrix} \Rightarrow 
		\begin{bmatrix}
		    v_1\\
		    v_2
		\end{bmatrix}
		= t \cdot 
		\begin{bmatrix}
		    2\\
		    1
		\end{bmatrix}
	\end{gather*}
	där $t \neq 0$ kan väljas godtyckligt.\\
\end{Ex}
% paragraph _ (end)
Vi ser att om $\vec{v}$ är en egenvektor så är $t \cdot \vec{v}$ ($t \neq 0$) det också. (Ofta väljs \textit{t} så att $\norm{\vec{v}} = 1$)
\paragraph{-} % (fold)
\label{par:_1}
Ett egenvärdesproblem för en ($n \times m$)-matris har \textit{n} stycken egenvärden. Med tillhörande egenvektorer. (Egenvärdena kan vara multipla och behöver ej vara reella).
% paragraph _ (end)

% section egenv_rden_och_egenvektorer (end)
\end{document}
