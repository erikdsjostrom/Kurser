\documentclass{article}

\usepackage[utf8]{inputenc}
\usepackage{amsthm}
\usepackage{amssymb}
\usepackage{mathtools}
\usepackage{graphicx}
\usepackage{mdframed}
\usepackage{float}
\usepackage[top=0.75in, bottom=0.75in, left=0.75in, right=0.75in]{geometry}
\usepackage{gauss}

\usepackage{array}
\allowdisplaybreaks

\makeatletter
\newcounter{elimination@steps}
\newcolumntype{R}[1]{>{\raggedleft\arraybackslash$}p{#1}<{$}}
\def\elimination@num@rights{}
\def\elimination@num@variables{}
\def\elimination@col@width{}
\newenvironment{elimination}[4][0]
{
    \setcounter{elimination@steps}{0}
    \def\elimination@num@rights{#1}
    \def\elimination@num@variables{#2}
    \def\elimination@col@width{#3}
    \renewcommand{\arraystretch}{#4}
    \start@align\@ne\st@rredtrue\m@ne
}
{
    \endalign
    \ignorespacesafterend
}
\newcommand{\step}[2]
{
    \ifnum\value{elimination@steps}>0\sim\quad\fi
    \left[
        \ifnum\elimination@num@rights>0
            \begin{array}
            {@{}*{\elimination@num@variables}{R{\elimination@col@width}}
            |@{}*{\elimination@num@rights}{R{\elimination@col@width}}}
        \else
            \begin{array}
            {@{}*{\elimination@num@variables}{R{\elimination@col@width}}}
        \fi
            #1
        \end{array}
    \right]
    & 
    \begin{array}{l}
        #2
    \end{array}
    \addtocounter{elimination@steps}{1}
}
\makeatother

\DeclarePairedDelimiter{\abs}{\lvert}{\rvert}
\DeclarePairedDelimiter{\norm}{\lvert \lvert}{\rvert \rvert}

\newtheoremstyle{break}% name
  {}%         Space above, empty = `usual value'
  {}%         Space below
  {\itshape}% Body font
  {}%         Indent amount (empty = no indent, \parindent = para indent)
  {\bfseries}% Thm head font
  {.}%        Punctuation after thm head
  {\newline}% Space after thm head: \newline = linebreak
  {}%         Thm head spec

\newtheorem{Def}{Definition}[section]

\theoremstyle{break}

\newtheorem{innerEx}{Exempel}[section]
\newtheorem{sats}{Sats}[section]
\newtheorem{Rem}{Anmärkning}[]

\newenvironment{Ex}
{\begin{mdframed} \begin{innerEx} \vspace{3pt}}
{\vspace{3pt} \end{innerEx} \end{mdframed}}  

\newenvironment{bevis}
{\begin{mdframed} \begin{proof} \vspace{3pt}}
{\vspace{3pt} \end{proof} \end{mdframed}}


\title{
	 Linjär Algebra\\
	 Föreläsning 10
    \author{Erik Sjöström}
}
\begin{document}
\maketitle
\section{System av linjär ekvationer} % (fold)
\label{sec:system_av_linj_r_ekvationer}
Ett system av linjära ekvationer med \textit{n} obekanta skrivs på detta sätt:
\[
\left\{ \begin{array}{ccccccccc}
	a_{11}x_1 &+& a_{12}x_2 &+ &...& +& a_{1n}x_n &=& b_1\\
	a_{21}x_1 &+& a_{22}x_2 &+ &...& +& a_{2n}x_n &=& b_2\\
	\vdots && \vdots && \vdots && \vdots && \vdots       \\
	a_{m1}x_1 &+& a_{m2}x_2 &+&...&+& a_{mn}x_n &=& b_m 
\end{array}
\]
Där $a_{ij}$, $b_j$ är reella tal och $x_j$ är obekant.\\
Man söker en lösning sådan att alla ekvationer är uppfyllda samtidigt.\\
Med matris-vektor beteckning skrivs ekvationssystemet som:
\[
\overbrace{
\begin{bmatrix}
a_{11}&a_{12}&...&a_{1n}\\
a_{21}&a_{22}&...&a_{2n}\\
\vdots & \vdots &\ddots& \vdots\\
a_{m1}&a_{m2}&...&a_{mn} \end{bmatrix}}^\text{Koefficientmatris} \cdot \overbrace{\begin{bmatrix} x_1\\x_2\\\vdots\\x_n \end{bmatrix}}^\text{Obekanta} = \overbrace{\begin{bmatrix} b_1\\b_2\\\vdots\\b_m \end{bmatrix}}^\text{Högerled}
\] 
Två fundamentala frågor:
\begin{enumerate}
	\item Finns det en lösning? (Existens)
	\item Om det finns en lösning, finns det flera? (Entydighet)
\end{enumerate}

\begin{Ex}
    \[
        \begin{cases}
        	x_1 - 2x_2 = -1 &\text{Två ekvationer}\\ 
        	-x_1 + 3x_2 = 3 &\text{Två obekanta}
        \end{cases}
    \]
    Dessa två ekvationer kan tolkas som två linjer:
    \begin{align*}
    &x_2 = \frac{x_1 + 1}{2} = l_1 & x_1 = 3x_2 -3 = l_2
    \end{align*}
    \begin{center}
    	Här ska det vara en figur
    \end{center}
    Lösningen ges av skärningen mellan linjerna:
    \[
        \begin{bmatrix} x_1\\x_2 \end{bmatrix} = \begin{bmatrix} 3\\2 \end{bmatrix}
    \]
    Det finns inga fler lösningar eftersom det bara finns en skärningspunkt.
\end{Ex}
% section system_av_linj_r_ekvationer (end)
\section{$\mathbb{R}^n$} % (fold)
\label{sec: R^n}
I $\mathbb{R}^n$ kan vi ej rita linjer!\\
Det allmänna tillvägagångssättet är att bestämma en lösning med Gauseliminering på totalmatrisen som består av.
\[
    \begin{bmatrix}
    \begin{array}{c|c}
    	\mathbf{A} & \vec{b}
    \end{array}
    \end{bmatrix}
\]
Där \textbf{A} är koefficientmatrisen och $\vec{b}$ är högerledet.
Tillåtna elementära radoperationer i gauseliminering är:
\begin{itemize}
	\item Addition: Addera till en rad en multipel av en annan rad
	\item Platsbyte: Låt två rader byta plats
	\item Skalning: Multiplicera en rad med en konstant $\neq 0$
\end{itemize}
Dessa operationer bibehåller lösningsmängden.
\begin{Ex}
    \[
        \begin{cases}
        	x_1 - 2x_2 = -1 &\text{Två ekvationer}\\ 
        	-x_1 + 3x_2 = 3 &\text{Två obekanta}
        \end{cases}
    \]
    Med matris-vektor beteckning:
    \[
        \begin{bmatrix} 1&-2\\-1&3 \end{bmatrix} \cdot \begin{bmatrix} x_1\\x_2 \end{bmatrix} = \begin{bmatrix} -1\\3 \end{bmatrix}
    \]
    Och om vi nu skriver om det som totalmatrisen ovan:
    \[
    \begin{bmatrix}
    \begin{array}{cc|c}
    1 & -2 & -1 \\
    -1 & 3 & 3
    \end{array}
    \end{bmatrix}
    \]
    Vi vill ha noll i det nedre vänstra hörnet. Så vi använder additionsoperationen, och adderar rad 1 till rad 2:
    \begin{elimination}[1]{2}{1.75em}{1.1}
    \eliminationstep
    {
        1 & -2 & -1\\
        -1 & 3 & 3
    }
    {
               \\
        +R_{1} \\
    }
    \eliminationstep
    {
        1 & -2 & -1\\
        0 & 1 & 2
    }
    {
                \\
                \\
    }
\end{elimination}
\noindent
Nu vill vi sätta det första elementet i den andra kolumnen (\textit{-2}) till 0. Det gör vi genom att kombinera skalnings- och additions-operationen. Vi adderar alltså 2 $\cdot$ Rad 2 till Rad 1.
\begin{elimination}[1]{2}{1.75em}{1.1}
\eliminationstep
{
    1 & -2 & -1 \\
    0 & 1 & 2
}
{
    + 2 \cdot R_2\\
    \\
}
\eliminationstep
{
    1 & 0 & 3\\
    0 & 1 & 2
}
{
    \\
    \\
}
\end{elimination}
Om vi nu stoppar tillbaka det i ekvationssystemet får vi:
\[
    \begin{cases}
        1x_1 + 0x_2 = 3 \\
        0x_1 + 1x_2 = 2
    \end{cases}
\]
Dvs:
\[
    \begin{cases}
        x_1 = 3\\
        x_2 = 2
    \end{cases}
\]
Vi har nu löst ekvationssystemet med gauseliminering och:
    \[
    \begin{bmatrix}
    \begin{array}{cc|c}
    1 & 0 & 3\\
    0 & 1 & 2
    \end{array}
    \end{bmatrix}
    \]
    Är skriven på reducerad trappstegsform.
\end{Ex}

\begin{Rem}
    Man använder '$\sim$' för att marker att matriserna är radekvivalenta, samma lösningsmäng.\\
    Obs: De är ej lika.
\end{Rem}

Nu tittar vi på systemet \textbf{A}
% section R^n (end)

\end{document}